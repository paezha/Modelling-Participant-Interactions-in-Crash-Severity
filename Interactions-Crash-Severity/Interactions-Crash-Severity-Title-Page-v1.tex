\documentclass[]{elsarticle} %review=doublespace preprint=single 5p=2 column
%%% Begin My package additions %%%%%%%%%%%%%%%%%%%
\usepackage[hyphens]{url}

  \journal{Some Journal} % Sets Journal name


\usepackage{lineno} % add
\providecommand{\tightlist}{%
  \setlength{\itemsep}{0pt}\setlength{\parskip}{0pt}}

\usepackage{graphicx}
\usepackage{booktabs} % book-quality tables
%%%%%%%%%%%%%%%% end my additions to header

\usepackage[T1]{fontenc}
\usepackage{lmodern}
\usepackage{amssymb,amsmath}
\usepackage{ifxetex,ifluatex}
\usepackage{fixltx2e} % provides \textsubscript
% use upquote if available, for straight quotes in verbatim environments
\IfFileExists{upquote.sty}{\usepackage{upquote}}{}
\ifnum 0\ifxetex 1\fi\ifluatex 1\fi=0 % if pdftex
  \usepackage[utf8]{inputenc}
\else % if luatex or xelatex
  \usepackage{fontspec}
  \ifxetex
    \usepackage{xltxtra,xunicode}
  \fi
  \defaultfontfeatures{Mapping=tex-text,Scale=MatchLowercase}
  \newcommand{\euro}{€}
\fi
% use microtype if available
\IfFileExists{microtype.sty}{\usepackage{microtype}}{}
\bibliographystyle{elsarticle-harv}
\ifxetex
  \usepackage[setpagesize=false, % page size defined by xetex
              unicode=false, % unicode breaks when used with xetex
              xetex]{hyperref}
\else
  \usepackage[unicode=true]{hyperref}
\fi
\hypersetup{breaklinks=true,
            bookmarks=true,
            pdfauthor={},
            pdftitle={An empirical assessment of strategies to model opponent effects in crash severity analysis},
            colorlinks=false,
            urlcolor=blue,
            linkcolor=magenta,
            pdfborder={0 0 0}}
\urlstyle{same}  % don't use monospace font for urls

\setcounter{secnumdepth}{5}
% Pandoc toggle for numbering sections (defaults to be off)


% Pandoc header
\usepackage[margin=1in]{geometry}
\usepackage{lineno}
\linenumbers



\begin{document}
\begin{frontmatter}

  \title{An empirical assessment of strategies to model opponent effects in crash
severity analysis}
    \author[McMaster University]{Antonio Paez\corref{Corresponding Author}}
   \ead{paezha@mcmaster.ca} 
    \author[Louisiana State University]{Hany Hassan}
   \ead{hassan1@lsu.edu} 
    \author[McMaster University]{Mark Ferguson}
   \ead{fergumr@mcmaster.ca} 
    \author[McMaster University]{Saiedeh Razavi}
   \ead{razavi@mcmaster.ca} 
      \address[McMaster University]{McMaster Institute for Transportation and Logistics, McMaster
University, 1280 Main Street West, Hamilton, Ontario, Canada L8S 4K1}
    \address[Louisiana State University]{Department of Civil and Environmental Engineering, Louisiana State
University, Baton Rouge, Louisiana, USA 70803}
    
  \begin{abstract}
  Road accidents impose an important burden on health and the economy.
  Numerous efforts to understand the factors that affect road collisions
  have been undertaken. One stream of research focus on modelling the
  severity of crashes. Crash severity research is useful to clarify the
  way different factors can influence the outcome of an event. The
  objective of this paper is to assess different strategies to model the
  interactions between participants in a crash in the context of crashes
  involving two parties. Towards this objective, a series of models are
  estimated using data from Canada's National Collision Database. Three
  levels of crash severity (no injury/injury/fatality) are analyzed using
  ordered logit models and covariates for the participants in the crash
  and the conditions of the crash. Modelling strategies include different
  ways of introducing the covariates (e.g., in a single-level or
  multi-level form), as well as by subsetting the dataset. The models are
  assessed using predicted shares and classes of outcomes, and the results
  highlight the importance of considering opponent effects in crash
  severity analysis. The study also suggests that hierarchical (i.e.,
  multi-level) specifications and subsetting do not necessarily perform
  better than a relatively simple single-level model with opponent
  effects. The results of this study provide insights regarding the
  performace of different modelling strategies, and should be informative
  to researchers working with crash severity models.
  \end{abstract}
  
 \end{frontmatter}




\end{document}


