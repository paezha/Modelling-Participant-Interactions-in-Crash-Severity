\documentclass[]{elsarticle} %review=doublespace preprint=single 5p=2 column
%%% Begin My package additions %%%%%%%%%%%%%%%%%%%
\usepackage[hyphens]{url}

  \journal{Some Journal} % Sets Journal name


\usepackage{lineno} % add
\providecommand{\tightlist}{%
  \setlength{\itemsep}{0pt}\setlength{\parskip}{0pt}}

\usepackage{graphicx}
\usepackage{booktabs} % book-quality tables
%%%%%%%%%%%%%%%% end my additions to header

\usepackage[T1]{fontenc}
\usepackage{lmodern}
\usepackage{amssymb,amsmath}
\usepackage{ifxetex,ifluatex}
\usepackage{fixltx2e} % provides \textsubscript
% use upquote if available, for straight quotes in verbatim environments
\IfFileExists{upquote.sty}{\usepackage{upquote}}{}
\ifnum 0\ifxetex 1\fi\ifluatex 1\fi=0 % if pdftex
  \usepackage[utf8]{inputenc}
\else % if luatex or xelatex
  \usepackage{fontspec}
  \ifxetex
    \usepackage{xltxtra,xunicode}
  \fi
  \defaultfontfeatures{Mapping=tex-text,Scale=MatchLowercase}
  \newcommand{\euro}{€}
\fi
% use microtype if available
\IfFileExists{microtype.sty}{\usepackage{microtype}}{}
\bibliographystyle{elsarticle-harv}
\ifxetex
  \usepackage[setpagesize=false, % page size defined by xetex
              unicode=false, % unicode breaks when used with xetex
              xetex]{hyperref}
\else
  \usepackage[unicode=true]{hyperref}
\fi
\hypersetup{breaklinks=true,
            bookmarks=true,
            pdfauthor={},
            pdftitle={Modelling participant interactions in crash severity analysis},
            colorlinks=false,
            urlcolor=blue,
            linkcolor=magenta,
            pdfborder={0 0 0}}
\urlstyle{same}  % don't use monospace font for urls

\setcounter{secnumdepth}{5}
% Pandoc toggle for numbering sections (defaults to be off)
% Pandoc header



\begin{document}
\begin{frontmatter}

  \title{Modelling participant interactions in crash severity analysis}
    \author[Some University]{Author 1\corref{c1}}
   \ead{a1@example.com} 
   \cortext[c1]{Corresponding Author}
    \author[Some Institute of Technology]{Author 2}
   \ead{a2@example.com} 
  
      \address[Some University]{Department, Street, City, State, Zip}
    \address[Some Institute of Technology]{Department, Street, City, State, Zip}
  
  \begin{abstract}
  Road safety continues to impose an important burden on health and the
  economy. Numerous efforts to understand the factors that affect road
  safety have been undertaken. One stream of research focus on modelling
  the severity of crashes. Crash severity research is useful to clarify
  the way different factors can influence an outcome where there are no
  injuries, or injuries, or fatalities. The objective of this paper is to
  describe an approach to model the interactions between participants in a
  crash, in the context of crashes involving two parties. Using data from
  Canada's National Collision Database we demonstrate the approach. Three
  levels of crash severity (no injury/injury/fatality) are analyzed using
  ordered logit models and covariates for the participants in the crash
  and the conditions of the crash. The results suggest that there are
  important interactions between the various descriptors of the
  participants in the crash. More generally, the example shows how
  modelling the interactions between participants explicitly leads to
  richer insights into the covariates of crash severity, in addition to
  better model fits. The approach described in this paper is intuitive,
  simple to implement, and can be adapted to advanced modelling methods
  with ease.
  \end{abstract}
  
 \end{frontmatter}

\hypertarget{introduction}{%
\section{Introduction}\label{introduction}}

Road safety continues to be a concern world-wide. According to a recent
report from the World Health Organization (2019), road accidents are the
8th leading cause of death for all ages, and the number one cause of
death for children and young people between the ages of 5 to 29. Of all
leading causes of death, road accidents are the only non-external cause,
unrelated to disease, disorder, or infection. Road accidents impose a
heavy burden on individuals and society as a whole. Gobally, there has
been an increasing trend in number of vehicles and road accident-related
deaths, even if the rate of death per 100,000 population and 100,000
vehicles have both fallen (World Health Organization, 2019, Figs. 1 and
2). These gains, although they are to be celebrated, cannot distract
from the crippling economic cost of premature death (e.g., Symons et
al., 2019; Wijnen et al., 2019), not to mention the long-term
consequences for survivors, measured in sometimes crushing emotional and
physical pain (e.g., Merlin et al., 2007; Devlin et al., 2019; Pelissier
et al., n.d.).

As the statistics suggest, the burden of road accidents is also not
borne evenly. There are important disparities at the international
level, where the odds of death due to a road crash are three times
higher in low-income countries compared to high-income countries; in
fact, no reductions in road accident-related fatalities were appreciated
in low-income countries between 2013 and 2016 (World Health
Organization, 2019). In the case of high-income countries, where
substantial gains in road safety have been observed for years, said
gains have been unevenly distributed; thus, while fatal crashes
involving older adults in the United States and Great Britain declined
between 1997 and 2010 (despite an increase in the number of older
people), the trend remained stable or increased slightly in Australia in
roughly the same period (Thompson et al., 2018). There are also
systematic differences in the impact of road accidents. For example, in
a study in the United States, Obeng (2011) reported that the impact of
covariates of crash severity varied in magnitude by gender, and in some
cases were not even the same. More recently, Regev et al.~(2018) used
adjusted crash risk to find that the risk of crashes in Great Britain
peaked for people 21 to 29 years of age; on the other hand, the risk of
fatal injuries for older drivers was constant, irrespective of the
seriousness of the crash - which highlights the perils of accidents at
older ages. Other studies have concentrated on the consequences of road
accidents on the young (e.g., Peek-Asa et al., 2010), the old (e.g.,
Rakotonirainy et al., 2012), as well as pedestrians and cyclists (e.g.,
Hanson et al., 2013; McArthur et al., 2014).

Given the relevance and cost of the matter, as well as the important
variations of the impacts among different population segments, numerous
efforts efforts exist to better understand the factors that affect road
safety - including the probable consequences of the crash. Consequently,
a stream of research in the analysis of road accidents is concerned with
the severity of crashes. In particular, multivariate analysis of crash
severity is a useful way to clarify the way various factors can affect
the outcome of an incident, to discriminate between various levels of
injury, from no injury (i.e., property damage only), to different
degrees of injury up to and including fatality. This is an active area
of research (Savolainen et al., 2011), and one where methodological
developments have aimed at improving the reliability, accuracy, and
precision of models.

This paper aspires to contribute to the literature on crash severity by
describing an approach to model the interactions between participants in
a crash, in the context of incidents involving two parties. The
importance of these interactions has been recognized in the existing
literature (e.g., Chiou et al., 2013; Lee and Li, 2014; Li et al.,
2017), and in this paper we present a systematic approach to model the
way participants in a crash interact to influence the severity of the
accident for each party. By way of application, we demonstrate the
approach through an empirical analysis of data drawn from Canada's
National Collision Database. Three levels of crash severity (no
injury/injury/fatality) are analyzed using ordered logit models and
covariates for the participants in the crash and the conditions of the
crash. The results suggest that there are important interactions between
the various descriptors of the participants in the crash. More
generally, the example shows how modelling the interactions between
participants explicitly leads to richer insights into the covariates of
crash severity, in addition to better model fits. The approach described
in this paper is intuitive, simple to implement, and can be adapted to
advanced modelling methods with ease.

The rest of this paper is structure as follows. In Section
\ref{sec:review-of-methods} we present a concise review of the methods
used to analyze crash severity, with a particular focus on techniques
that consider the interactions between participants in a crash.

\hypertarget{sec:review-of-methods}{%
\section{Methodological approaches in crash severity
analysis}\label{sec:review-of-methods}}

Methods:

Iranitalab and Khattak (2017); Khan et al. (2015); Chang and Wang
(2006): machine learning Quddus et al. (2010): ordinal models Kadilar
(2016): conditional logistic regression Mooradian et al. (2013): partial
proportional odds models Ma et al. (2008): counts

Interplay between participants:

Chiou et al. (2013)

Lee and Li (2014) consider effect on crash severity of interactions
between different types of vehicles. This they do by subsetting the
dataset and estimating independent models for each subset of data. Since
they consider three types of vehicles, namely cars (C), light trucks
(L), and heavy trucks (H), they work with nine datasets, for each type
of interactions (i.e., C-C, C-L, C-H, and so on).

Li et al. (2017)

\hypertarget{methods}{%
\section{Methods}\label{methods}}

Words go here.

\hypertarget{illustrative-application}{%
\section{Illustrative application}\label{illustrative-application}}

Words go here.

\hypertarget{further-considerations}{%
\section{Further considerations}\label{further-considerations}}

Words go here.

\hypertarget{concluding-remarks}{%
\section{Concluding remarks}\label{concluding-remarks}}

Words go here.

\hypertarget{references}{%
\section*{References}\label{references}}
\addcontentsline{toc}{section}{References}

\hypertarget{refs}{}
\leavevmode\hypertarget{ref-Chang2006analysis}{}%
Chang, L.Y., Wang, H.W., 2006. Analysis of traffic injury severity: An
application of non-parametric classification tree techniques. Accident
Analysis and Prevention 38, 1019--1027.
doi:\href{https://doi.org/10.1016/j.aap.2006.04.009}{10.1016/j.aap.2006.04.009}

\leavevmode\hypertarget{ref-Chiou2013modeling}{}%
Chiou, Y.C., Hwang, C.C., Chang, C.C., Fu, C., 2013. Modeling
two-vehicle crash severity by a bivariate generalized ordered probit
approach. Accident Analysis and Prevention 51, 175--184.
doi:\href{https://doi.org/10.1016/j.aap.2012.11.008}{10.1016/j.aap.2012.11.008}

\leavevmode\hypertarget{ref-Devlin2019road}{}%
Devlin, A., Beck, B., Simpson, P.M., Ekegren, C.L., Giummarra, M.J.,
Edwards, E.R., Cameron, P.A., Liew, S., Oppy, A., Richardson, M., Page,
R., Gabbe, B.J., 2019. The road to recovery for vulnerable road users
hospitalised for orthopaedic injury following an on-road crash. Accident
Analysis and Prevention 132, 10.
doi:\href{https://doi.org/10.1016/j.aap.2019.105279}{10.1016/j.aap.2019.105279}

\leavevmode\hypertarget{ref-Hanson2013severity}{}%
Hanson, C.S., Noland, R.B., Brown, C., 2013. The severity of pedestrian
crashes: An analysis using google street view imagery. Journal of
Transport Geography 33, 42--53.
doi:\href{https://doi.org/10.1016/j.jtrangeo.2013.09.002}{10.1016/j.jtrangeo.2013.09.002}

\leavevmode\hypertarget{ref-Iranitalab2017comparison}{}%
Iranitalab, A., Khattak, A., 2017. Comparison of four statistical and
machine learning methods for crash severity prediction. Accident
Analysis and Prevention 108, 27--36.
doi:\href{https://doi.org/10.1016/j.aap.2017.08.008}{10.1016/j.aap.2017.08.008}

\leavevmode\hypertarget{ref-Kadilar2016effect}{}%
Kadilar, G.O., 2016. Effect of driver, roadway, collision, and vehicle
characteristics on crash severity: A conditional logistic regression
approach. International Journal of Injury Control and Safety Promotion
23, 135--144.
doi:\href{https://doi.org/10.1080/17457300.2014.942323}{10.1080/17457300.2014.942323}

\leavevmode\hypertarget{ref-Khan2015exploring}{}%
Khan, G., Bill, A.R., Noyce, D.A., 2015. Exploring the feasibility of
classification trees versus ordinal discrete choice models for analyzing
crash severity. Transportation Research Part C-Emerging Technologies 50,
86--96.
doi:\href{https://doi.org/10.1016/j.trc.2014.10.003}{10.1016/j.trc.2014.10.003}

\leavevmode\hypertarget{ref-Lee2014analysis}{}%
Lee, C., Li, X.C., 2014. Analysis of injury severity of drivers involved
in single- and two-vehicle crashes on highways in ontario. Accident
Analysis and Prevention 71, 286--295.
doi:\href{https://doi.org/10.1016/j.aap.2014.06.008}{10.1016/j.aap.2014.06.008}

\leavevmode\hypertarget{ref-Li2017interplay}{}%
Li, L., Hasnine, M.S., Habib, K.M.N., Persaud, B., Shalaby, A., 2017.
Investigating the interplay between the attributes of at-fault and
not-at-fault drivers and the associated impacts on crash injury
occurrence and severity level. Journal of Transportation Safety \&
Security 9, 439--456.
doi:\href{https://doi.org/10.1080/19439962.2016.1237602}{10.1080/19439962.2016.1237602}

\leavevmode\hypertarget{ref-Ma2008multivariate}{}%
Ma, J.M., Kockelman, K.M., Damien, P., 2008. A multivariate
poisson-lognormal regression model for prediction of crash counts by
severity, using bayesian methods. Accident Analysis and Prevention 40,
964--975.
doi:\href{https://doi.org/10.1016/j.aap.2007.11.002}{10.1016/j.aap.2007.11.002}

\leavevmode\hypertarget{ref-McArthur2014spatial}{}%
McArthur, A., Savolainen, P.T., Gates, T.J., 2014. Spatial analysis of
child pedestrian and bicycle crashes development of safety performance
function for areas adjacent to schools. Transportation Research Record
57--63. doi:\href{https://doi.org/10.3141/2465-08}{10.3141/2465-08}

\leavevmode\hypertarget{ref-Merlin2007stress}{}%
Merlin, E.P.R., Gonzalez-Forteza, C., Lira, L.R., Tapia, J.A.J., 2007.
Post-traumatic stress disorder in patients with non intentional injuries
caused by road traffic accidents. Salud Mental 30, 43--48.

\leavevmode\hypertarget{ref-Mooradian2013analysis}{}%
Mooradian, J., Ivan, J.N., Ravishanker, N., Hu, S., 2013. Analysis of
driver and passenger crash injury severity using partial proportional
odds models. Accident Analysis and Prevention 58, 53--58.
doi:\href{https://doi.org/10.1016/j.aap.2013.04.022}{10.1016/j.aap.2013.04.022}

\leavevmode\hypertarget{ref-Obeng2011gender}{}%
Obeng, K., 2011. Gender differences in injury severity risks in crashes
at signalized intersections. Accident Analysis and Prevention 43,
1521--1531.
doi:\href{https://doi.org/10.1016/j.aap.2011.03.004}{10.1016/j.aap.2011.03.004}

\leavevmode\hypertarget{ref-Peek-Asa2010teenage}{}%
Peek-Asa, C., Britton, C., Young, T., Pawlovich, M., Falb, S., 2010.
Teenage driver crash incidence and factors influencing crash injury by
rurality. Journal of Safety Research 41, 487--492.
doi:\href{https://doi.org/10.1016/j.jsr.2010.10.002}{10.1016/j.jsr.2010.10.002}

\leavevmode\hypertarget{ref-Pelissier2019medical}{}%
Pelissier, C., Fort, E., Fontana, L., Hours, M., n.d. Medical and
socio-occupational predictive factors of psychological distress 5 years
after a road accident: A prospective study. Social Psychiatry and
Psychiatric Epidemiology 13.
doi:\href{https://doi.org/10.1007/s00127-019-01780-0}{10.1007/s00127-019-01780-0}

\leavevmode\hypertarget{ref-Quddus2010road}{}%
Quddus, M.A., Wang, C., Ison, S.G., 2010. Road traffic congestion and
crash severity: Econometric analysis using ordered response models.
Journal of Transportation Engineering-Asce 136, 424--435.
doi:\href{https://doi.org/10.1061/(asce)te.1943-5436.0000044}{10.1061/(asce)te.1943-5436.0000044}

\leavevmode\hypertarget{ref-Rakotonirainy2012older}{}%
Rakotonirainy, A., Steinhardt, D., Delhomme, P., Darvell, M., Schramm,
A., 2012. Older drivers' crashes in queensland, australia. Accident
Analysis and Prevention 48, 423--429.
doi:\href{https://doi.org/10.1016/j.aap.2012.02.016}{10.1016/j.aap.2012.02.016}

\leavevmode\hypertarget{ref-Regev2018crash}{}%
Regev, S., Rolison, J.J., Moutari, S., 2018. Crash risk by driver age,
gender, and time of day using a new exposure methodology. Journal of
Safety Research 66, 131--140.
doi:\href{https://doi.org/10.1016/j.jsr.2018.07.002}{10.1016/j.jsr.2018.07.002}

\leavevmode\hypertarget{ref-Savolainen2011statistical}{}%
Savolainen, P.T., Mannering, F., Lord, D., Quddus, M.A., 2011. The
statistical analysis of highway crash-injury severities: A review and
assessment of methodological alternatives. Accident Analysis and
Prevention 43, 1666--1676.
doi:\href{https://doi.org/10.1016/j.aap.2011.03.025}{10.1016/j.aap.2011.03.025}

\leavevmode\hypertarget{ref-Symons2019reduced}{}%
Symons, J., Howard, E., Sweeny, K., Kumnick, M., Sheehan, P., 2019.
Reduced road traffic injuries for young people: A preliminary investment
analysis. Journal of Adolescent Health 65, S34--S43.
doi:\href{https://doi.org/10.1016/j.jadohealth.2019.01.009}{10.1016/j.jadohealth.2019.01.009}

\leavevmode\hypertarget{ref-Thompson2018trends}{}%
Thompson, J.P., Baldock, M.R.J., Dutschke, J.K., 2018. Trends in the
crash involvement of older drivers in australia. Accident Analysis and
Prevention 117, 262--269.
doi:\href{https://doi.org/10.1016/j.aap.2018.04.027}{10.1016/j.aap.2018.04.027}

\leavevmode\hypertarget{ref-Wijnen2019analysis}{}%
Wijnen, W., Weijermars, W., Schoeters, A., Berghe, W. van den, Bauer,
R., Carnis, L., Elvik, R., Martensen, H., 2019. An analysis of official
road crash cost estimates in european countries. Safety Science 113,
318--327.
doi:\href{https://doi.org/10.1016/j.ssci.2018.12.004}{10.1016/j.ssci.2018.12.004}

\leavevmode\hypertarget{ref-WHO2019global}{}%
World Health Organization, 2019. Global status report on road safety
2018 (2018). Geneva.


\end{document}


