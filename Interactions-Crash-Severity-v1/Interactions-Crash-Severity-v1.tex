\documentclass[]{elsarticle} %review=doublespace preprint=single 5p=2 column
%%% Begin My package additions %%%%%%%%%%%%%%%%%%%
\usepackage[hyphens]{url}

  \journal{Some Journal} % Sets Journal name


\usepackage{lineno} % add
\providecommand{\tightlist}{%
  \setlength{\itemsep}{0pt}\setlength{\parskip}{0pt}}

\usepackage{graphicx}
\usepackage{booktabs} % book-quality tables
%%%%%%%%%%%%%%%% end my additions to header

\usepackage[T1]{fontenc}
\usepackage{lmodern}
\usepackage{amssymb,amsmath}
\usepackage{ifxetex,ifluatex}
\usepackage{fixltx2e} % provides \textsubscript
% use upquote if available, for straight quotes in verbatim environments
\IfFileExists{upquote.sty}{\usepackage{upquote}}{}
\ifnum 0\ifxetex 1\fi\ifluatex 1\fi=0 % if pdftex
  \usepackage[utf8]{inputenc}
\else % if luatex or xelatex
  \usepackage{fontspec}
  \ifxetex
    \usepackage{xltxtra,xunicode}
  \fi
  \defaultfontfeatures{Mapping=tex-text,Scale=MatchLowercase}
  \newcommand{\euro}{€}
\fi
% use microtype if available
\IfFileExists{microtype.sty}{\usepackage{microtype}}{}
\bibliographystyle{elsarticle-harv}
\ifxetex
  \usepackage[setpagesize=false, % page size defined by xetex
              unicode=false, % unicode breaks when used with xetex
              xetex]{hyperref}
\else
  \usepackage[unicode=true]{hyperref}
\fi
\hypersetup{breaklinks=true,
            bookmarks=true,
            pdfauthor={},
            pdftitle={An assessment of strategies to model opponent effects in crash severity analysis},
            colorlinks=false,
            urlcolor=blue,
            linkcolor=magenta,
            pdfborder={0 0 0}}
\urlstyle{same}  % don't use monospace font for urls

\setcounter{secnumdepth}{5}
% Pandoc toggle for numbering sections (defaults to be off)
% Pandoc header
\usepackage{booktabs}
\usepackage{longtable}
\usepackage{array}
\usepackage{multirow}
\usepackage{wrapfig}
\usepackage{float}
\usepackage{colortbl}
\usepackage{pdflscape}
\usepackage{tabu}
\usepackage{threeparttable}
\usepackage{threeparttablex}
\usepackage[normalem]{ulem}
\usepackage{makecell}
\usepackage{xcolor}

\usepackage[margin=1in]{geometry}



\begin{document}
\begin{frontmatter}

  \title{An assessment of strategies to model opponent effects in crash severity
analysis}
    \author[Some University]{Author 1\corref{c1}}
   \ead{a1@example.com} 
   \cortext[c1]{Corresponding Author}
    \author[Some Institute of Technology]{Author 2}
   \ead{a2@example.com} 
  
      \address[Some University]{Department, Street, City, State, Zip}
    \address[Some Institute of Technology]{Department, Street, City, State, Zip}
  
  \begin{abstract}
  Road accidents impose an important burden on health and the economy.
  Numerous efforts to understand the factors that affect road collisions
  have been undertaken. One stream of research focus on modelling the
  severity of crashes. Crash severity research is useful to clarify the
  way different factors can influence the outcome of the event. The
  objective of this paper is to assess different strategies to model the
  interactions between participants in a crash, in the context of crashes
  involving two parties. Towards this objective, a series of models are
  estimated using data from Canada's National Collision Database. Three
  levels of crash severity (no injury/injury/fatality) are analyzed using
  ordered logit models and covariates for the participants in the crash
  and the conditions of the crash. Modelling strategies include different
  ways of introducing the covariates (e.g., in a single-level or
  multi-level form), as well as by subsetting the dataset. The models are
  assessed using predicted shares and outcomes, and the results highlight
  the importance of considering opponent effects in crash severity
  analysis. On the other hand, the study suggests that hierarchical (i.e.,
  multi-level) specifications and subsetting do not perform necessarily
  better than a relatively simple single-level model with opponent
  effects.
  \end{abstract}
  
 \end{frontmatter}

\hypertarget{introduction}{%
\section{Introduction}\label{introduction}}

Road safety continues to be a concern world-wide. According to a recent
report from the World Health Organization (2019), road accidents are the
8th leading cause of death for all ages, and the number one cause of
death for children and young people between the ages of 5 to 29. Of all
leading causes of death, road accidents are the only cause of death
unrelated to disease, disorder, or infection. Road accidents impose a
heavy burden on individuals and society as a whole. Gobally, the rate of
road collision-related deaths per 100,000 population and 100,000
vehicles have both fallen, even as the number of vehicles has grown
(World Health Organization, 2019, Figs. 1 and 2). These gains, although
they are to be celebrated, cannot distract from the crushing economic
cost of premature death (e.g., Symons et al., 2019; Wijnen et al.,
2019), not to mention the long-term consequences for survivors, measured
in sometimes crippling emotional and physical pain (e.g., Merlin et al.,
2007; Devlin et al., 2019; Pelissier et al., n.d.).

Evidence from across the world suggests that the burden of road
accidents is not borne evenly. There are important disparities at the
international level, where the odds of death due to road crashes are
three times higher in low-income countries compared to high-income
countries; in fact, no reductions in road accident-related fatalities
were appreciated in low-income countries between 2013 and 2016 (World
Health Organization, 2019). In the case of high-income countries, where
substantial gains in road safety have been observed for years, said
gains have also been unevenly distributed; thus, while fatal crashes
involving older adults in the United States and Great Britain declined
between 1997 and 2010 (despite the graying of the population), the trend
remained stable or increased slightly in Australia in roughly the same
period (Thompson et al., 2018). There are also systematic differences in
the impact of road accidents. For example, in a study in the United
States, Obeng (2011) reported that the impact of covariates of crash
severity varied substantially in magnitude by gender. More recently,
Regev et al.~(2018) used adjusted crash risk to find that the risk of
crashes in Great Britain peaked for people 21 to 29 years of age; on the
other hand, the risk of fatal injuries for older drivers was constant,
irrespective of the seriousness of the crash - which highlights the
perils of accidents at older ages. Other studies have concentrated on
the consequences of road accidents for the young (e.g., Peek-Asa et al.,
2010), the old (e.g., Rakotonirainy et al., 2012), as well as
pedestrians and cyclists (e.g., Hanson et al., 2013; McArthur et al.,
2014).

Given the relevance and cost of this matter, as well as the important
variations of the impacts among different population segments, numerous
efforts efforts have been conducted to better understand the factors
that affect road safety - including the probable consequences of
crashes. Consequently, a stream of research in the analysis of road
accidents is concerned with the severity of crashes. In particular,
multivariate analysis of crash severity is a useful way to clarify the
way various factors can affect the outcome of an incident, to
discriminate between various levels of injury, from no injury (i.e.,
property damage only), to different degrees of injury up to and
including fatality. This is an active area of research (e.g., Savolainen
et al., 2011), and one where methodological developments have aimed at
improving the reliability, accuracy, and precision of models.

This paper aspires to contribute to the literature on crash severity by
assessing different modelling strategies useful to incorporate opponent
effects in crash analysis, in the context of incidents involving two
parties. The importance of these interactions has been recognized in the
existing literature (e.g., Chiou et al., 2013; Lee and Li, 2014; Li et
al., 2017; Tarrao et al., 2014), and a number of different modelling
strategies have been proposed. In this paper we present a systematic
assessment of several relevant modelling strategies, ranging from the
way variables are defined in single-level models, in multi-level models
(i.e., hierarchical models), as well as using data subsetting
approaches. For the assessment we use data from Canada's National
Collision Database, a database that collects all police-reported
collisions in the country. Using the most recent version of the dataset
(2017), three levels of crash severity (no injury/injury/fatality) are
analyzed using ordered logit models and covariates for the participants
in the crash and the conditions of the crash. For model assessment, we
conduct an in-sample prediction exercise using the estimation sample
(i.e., \emph{nowcasting}), and also an out-of-sample prediction exercise
using the dataset corresponding to 2016 (i.e., \emph{backcasting}). The
models are assessed using predicted shares and predicted individual
outcomes using an extensive array of verification statistics. The
results highlight the importance of considering opponent effects in
crash severity analysis to improve the goodness-of-fit and predictive
performance o. On the other hand, the study suggests that hierarchical
variable specifications and subsetting do not perform necessarily better
than a relatively simple single-level model with opponent effects.

The rest of this paper is structured as follows. In Section
\ref{sec:review-of-methods} we present a concise review of the methods
used to analyze crash severity, with a particular focus on techniques
that consider the interactions between participants in a crash. Section
\ref{sec:methods} describes the data requirements, data preprocessing,
and the modelling strategy. Model estimation is presented in
\ref{sec:application} and the assessment of models is in Section
\ref{sec:assessment}. We then present some additional thoughts about the
applicability of this approach in Section
\ref{sec:further-considerations} before offering some concluding remarks
in Section \ref{sec:concluding-remarks}.

\hypertarget{sec:review-of-methods}{%
\section{Methodological approaches in crash severity
analysis}\label{sec:review-of-methods}}

Modelling the outcomes of crashes in terms of the severity of injuries
to participants has been a preoccupation of transportation researchers,
planners, auto insurance companies, governments and the public for
decades. One of the earliests studies to investigate the severity of
injuries conditional on an accident having occurred was by White and
Clayton (1972). Kim et al.~(1995) later stated that the ``linkages
between severity of injury and driver characteristics and behaviors have
not been thoroughly investigated'' (p.~470). Nowadays, there is a
burgeoning literature on this subject, including methodological
developments, case studies, and more niche research with a focus on
particular situations (e.g., crashes at intersections, Mussone et al.,
2017; crashes in rural roads, Gong and Fan, 2017) or special populations
(e.g., crashes involving motorcyclists or active travelers; see Shaheed
et al., 2013; Salon and McIntyre, 2018).

Crash severity is often modelled using models for discrete outcomes,
including classification techniques from machine learning (e.g.,
Iranitalab and Khattak, 2017; Chang and Wang, 2006; Effati et al., 2015;
Khan et al., 2015), Poisson models for counts (e.g., Ma et al., 2008),
unordered logit/probit models {[}{]}, as well as ordered logit/probit
models (e.g., Rifaat and Chin, 2007), with numerous variants, such as
random parameters/mixed logit (e.g., Aziz et al., 2013; Haleem and Gan,
2013), partial proportional odds models (e.g., Mooradian et al., 2013;
Sasidharan and Menendez, 2014), and the use of copulas (e.g., Wang et
al., 2015). Recent reviews of methods include Savolainen et al.~(2011)
and Shamsunnahar and Eluru (2013). Shamsunnahar and Eluru (2013) in
particular conducted an extensive comparison of models for discrete
outcomes and found that while the difference between the performance of
unordered models and ordered models was so small as to make not
difference, ordered models are usually more parsimonious since only one
latent functions needs to be estimated for all outcomes, as opposed to
one for each outcome in unordered modelling mechanisms.

Irrespective of the modelling framework employed, models of crash
severity often include variables in several categories, as shown with
examples in Table \ref{tab:table-variable-categories} (also see Montella
et al., 2013). Many crash databases (but not by any means all) also
account for the multievent nature of many crashes. In this way, there
are crashes that involve a single traffic unit (e.g., Kim et al., 2013;
Gong and Fan, 2017), others that involve two traffic units (e.g., Tarrao
et al., 2014; Wang et al., 2015), and more rarely there are
multi-traffic unit crashes (e.g., Wu et al., 2014; Bogue et al., 2017).
Likewise, each traffic unit can possibly involve more than one person
(e.g., driver and passengers). Thus, depending on the incident, each
record in a crash database may include unique identifier for the crash,
as well as identifiers for the traffic units (or identifiers for dummy
objects such as a light pole that was hit by a vehicle), and the people
involved in the crash.

\begin{table}

\caption{\label{tab:table-variable-categories}\label{tab:table-variable-categories}Categories of variables used in the analysis of crash severity with examples}
\centering
\begin{tabular}[t]{l>{\raggedright\arraybackslash}p{22em}}
\toprule
Category & Examples\\
\midrule
Person-related & Attributes of participants in the crash, e.g., injury status, age, gender, licensing status, professional driver status\\
Traffic unit-related & Attributes of the traffic unit, e.g., type of traffic unit (car, motorcycle, etc.), maneouver, etc.\\
Crash-related & Attributes of the crash, e.g., location, weather conditions, light conditions, number of parties, etc.\\
Road-related & Attributes of the road, e.g., surface condition, grade, geometry, etc.\\
\bottomrule
\end{tabular}
\end{table}

Interplay between participants:

Chiou et al. (2013)

Lee and Li (2014) consider effect on crash severity of interactions
between different types of vehicles. This they do by subsetting the
dataset and estimating independent models for each subset of data. Since
they consider three types of vehicles, namely cars (C), light trucks
(L), and heavy trucks (H), they work with nine datasets, for each type
of interactions (i.e., C-C, C-L, C-H, and so on). Tarrao et al. (2014)

Mannering et al. (2016)

Li et al. (2017)

Salon and McIntyre (2018)

Chen et al. (2019)

Wu et al. (2014) two-vehicle crashes do not really consider the
interactions.

Many models use a latent-variable approach, whereby the severity of the
crash (observed) is linked to an underlying latent variable that is a
function of the variables: \[
y_{ik}^*=\sum_{l=1}\beta_lp_{ikl} + \sum_{m=1}\gamma_mc_{km} + \epsilon_{ik}
\]

The left-hand side of the expression above (\(y_{ik}^*\)) is a latent
(unobservable) variable that is associated with the severity of crash
\(k\) (\(k=1,\cdots,K\)) for participant \(i\) (\(i = 1,\cdots,n\)). The
right-hand side of the expression is split in three parts. The first
part collects \(l=1,\cdots,L\) individual attributes \(p\) for
participant \(i\) in crash \(k\); these could relate to the person
(e.g., age and gender) or be individual attributes of the traffic unit
(e.g., maneuver or vehicle type). The second part collects
\(m=1,\cdots,M\) attributes \(c\) related to the crash \(k\), including
crash-related and road-related data. The last part is a random term
specific to participant \(i\) in crash \(k\).

The latent variable is not observed directly, but it is possible to
posit a probabilistic relationship with the outcome \(y_{ik}\) (the
severity of crash \(k\) for participant \(i\)). Depending on the
characteristics of the data and the assumptions made about the random
component of the latent function different models can be obtained. For
example, if crash severity is coded as a binary variable (e.g.,
non-fatal/fatal), we can relate the latent variable to the outcome as
follows: \[
y_{ik} = 
\begin{cases}
\text{fatal} & \text{if } y_{ik}^*>0\\
\text{non-fatal} & \text{if } y_{ik}^*\leq0
\end{cases}
\]

Due to the stochastic nature of the latent function, the outcome of the
crash is not fully determined. However, we can make the following
probability statement: \[
P(y_i = \text{fatal}) = P(y_i^* > 0)
\] In other words, the probability of a fatal accident equals the
probability that the latent variable is greater than zero. This implies
(see Maddala, 1986, p. 22): \[
\begin{array}{rl}\
P(y_i = \text{fatal}) &= P(\sum_{l=1}\beta_lp_{ikl} + \sum_{m=1}\gamma_mc_{km} + \epsilon_{ik} > 0)\\ 
&=P(\epsilon_{ik} > -\sum_{l=1}\beta_lp_{ikl} - \sum_{m=1}\gamma_mc_{km})
\end{array}
\]

It the random terms \(\epsilon_i\) are assumed to follow the logistic
distribution, the the binary logit model is obtained; if they are
assumed to follow the normal distribution, the binary probit model is
obtained.

More often, though, the outcome is recorded using more categories, for
example property damage only (PDO)/injury/fatality. A similar approach
can be adopted, with a latent variable that relates to the outcome as
follows: \[
y_i = 
\begin{cases}
\text{fatality} & \text{if } y_i^*> k_2\\
\text{injury} & \text{if } k_1< y_i^*<k_2\\
\text{PDO} & \text{if } y_i^*<k_1
\end{cases}
\] where \(k_1\) and \(k_2\) are estimable thresholds. In this case, the
associated probability statements are as follows: \[
\begin{array}{rcl}\
P(y_{ik} = \text{PDO}) &=& 1 - P(y_{ik} = \text{injury}) - P(y_{ik} = \text{fatality})\\ 
P(y_{ik} = \text{injury}) &=& P(k_1 - \sum_{l=1}\beta_lp_{ikl} - \sum_{m=1}\gamma_mc_{km} < \epsilon_{ik} < k_2 - \sum_{l=1}\beta_lp_{ikl} - \sum_{m=1}\gamma_mc_{km} )\\
P(y_{ik} = \text{fatality}) &=& P(\epsilon_{ik} < k_1 - \sum_{l=1}\beta_lp_{ikl} - \sum_{m=1}\gamma_mc_{km} )
\end{array}
\]

If the random terms are assumed to follow the logistic distribution, the
ordered logit model is obtained; if the normal distribution, then the
ordered probit model. Estimation methods for these models are very
well-established (e.g., Maddala, 1986; Train, 2009)

There are numerous variations of the basic modelling framework above,
including hierarchical models, bivariate models, multinomial models, and
Bayesian models, among others (see Savolainen et al., 2011 for a review
of methods).

\hypertarget{sec:methods}{%
\section{Methods}\label{sec:methods}}

\hypertarget{data-considerations}{%
\subsection{Data considerations}\label{data-considerations}}

Words go here

Montella et al. (2013)

\hypertarget{model-specification}{%
\subsection{Model specification}\label{model-specification}}

More words go here

\hypertarget{sec:application}{%
\section{Application}\label{sec:application}}

\begin{quote}
\textbf{Note:} that this paper presents reproducible research. The
source file is an R Markdown document. All code and data necessary to
reproduce the analysis are available from the following anonymous Drive
folder:
\end{quote}

\begin{quote}
https://drive.google.com/open?id=12aJtVBaQ4Zj0xa7mtfqxh0E48hKCb\_XV
\end{quote}

\begin{quote}
The source files, code, and data will be publicly available in a GitHub
repository upon acceptance of the paper for publication
\end{quote}

\hypertarget{data-for-the-application}{%
\subsection{Data for the application}\label{data-for-the-application}}

To assess the performance of the modelling strategies discussed in
Section \ref{sec:methods}, we use data from Canada's National Collision
Database (NCDB). This is database contains all motor vehicle collisions
on public roads in Canada as reported by a police service. Data are
collected by provinces and territories, and shared with the federal
government, where data are combined, tracked, and analyzed for reporting
of deaths, injuries, and collisions in Canada at the national level. The
NCDB is provided by Transport Canada, the agency of the federal
government of Canada in charge of transportation policies and programs,
under the Open Government License - Canada version 2.0
{[}https://open.canada.ca/en/open-government-licence-canada{]}.

The NCDB is available from 1999. For the purpose of this paper, we use
the most recent year available as of this writing (2017). Furthermore,
for assessment we also use the data corresponding to 2016. Similar to
databases in other jurisdictions {[}see @{]}, the NCDB contains
information pertaining to the collision, the traffic unit, and the
person(s) involved.

\begin{tabular}{ll}
\toprule
Variable & Details\\
\midrule
Age & Details\\
\bottomrule
\multicolumn{2}{l}{\textit{Note: }}\\
\multicolumn{2}{l}{NCDB available from https://open.canada.ca/data/en/dataset/1eb9eba7-71d1-4b30-9fb1-30cbdab7e63a}\\
\end{tabular}

\hypertarget{model-estimation}{%
\subsection{Model estimation}\label{model-estimation}}

Words go here

Model specification. See Table:

\begin{table}[H]
\centering
\resizebox{\linewidth}{!}{
\begin{tabular}{llllll}
\toprule
Variable & Notes & Model 1 & Model 2 & Model 3 & Model 4\\
\midrule
\rowcolor{gray!6}  \addlinespace[0.3em]
\multicolumn{6}{l}{\textbf{Individual-level variables}}\\
\hspace{1em}Age & In decades & $\checkmark$ & $\checkmark$ & $\checkmark$ & $\checkmark$\\
\hspace{1em}Age Squared &  & $\checkmark$ & $\checkmark$ & $\checkmark$ & $\checkmark$\\
\rowcolor{gray!6}  \hspace{1em}Sex & Reference: Female & $\checkmark$ & $\checkmark$ & $\checkmark$ & $\checkmark$\\
\hspace{1em}Use of Safety Devices & 7 levels; Reference: No Safety Device & $\checkmark$ & $\checkmark$ & $\checkmark$ & $\checkmark$\\
\rowcolor{gray!6}  \addlinespace[0.3em]
\multicolumn{6}{l}{\textbf{Traffic unit-level variables}}\\
\hspace{1em}Passenger & Reference: Driver & $\checkmark$ & $\checkmark$ &  & $\checkmark$\\
\hspace{1em}Pedestrian & Reference: Driver & $\checkmark$ & $\checkmark$ &  & $\checkmark$\\
\rowcolor{gray!6}  \hspace{1em}Bicyclist & Reference: Driver & $\checkmark$ & $\checkmark$ &  & $\checkmark$\\
\hspace{1em}Motorcyclist & Reference: Driver & $\checkmark$ & $\checkmark$ &  & $\checkmark$\\
\rowcolor{gray!6}  \hspace{1em}Light Truck & Reference: Light Duty Vehicle & $\checkmark$ & $\checkmark$ &  & $\checkmark$\\
\hspace{1em}Heavy Vehicle & Reference: Light Duty Vehicle & $\checkmark$ & $\checkmark$ &  & $\checkmark$\\
\rowcolor{gray!6}  \addlinespace[0.3em]
\multicolumn{6}{l}{\textbf{Opponent variables)}}\\
\hspace{1em}Age of Opponent & In decades &  & $\checkmark$ & $\checkmark$ & \\
\hspace{1em}Age of Opponent Squared &  &  & $\checkmark$ & $\checkmark$ & \\
\rowcolor{gray!6}  \hspace{1em}Sex of Opponent & Reference: Female &  & $\checkmark$ & $\checkmark$ & \\
\hspace{1em}Opponent: Light Duty Vehicle & Reference: Pedestrian/Bicyclist/Motorcyclist &  & $\checkmark$ & $\checkmark$ & $\checkmark$\\
\rowcolor{gray!6}  \hspace{1em}Opponent: Light Truck & Reference: Pedestrian/Bicyclist/Motorcyclist &  & $\checkmark$ & $\checkmark$ & $\checkmark$\\
\hspace{1em}Opponent: Heavy Vehicle & Reference: Pedestrian/Bicyclist/Motorcyclist &  & $\checkmark$ & $\checkmark$ & $\checkmark$\\
\rowcolor{gray!6}  \addlinespace[0.3em]
\multicolumn{6}{l}{\textbf{Hierarchical traffic unit variables)}}\\
\hspace{1em}Age:Light Truck Driver &  &  &  & $\checkmark$ & \\
\hspace{1em}Age Squared:Light Truck Driver &  &  &  & $\checkmark$ & \\
\rowcolor{gray!6}  \hspace{1em}Age:Heavy Vehicle Driver &  &  &  & $\checkmark$ & \\
\hspace{1em}Age Squared:Heavy Vehicle Driver &  &  &  & $\checkmark$ & \\
\rowcolor{gray!6}  \hspace{1em}Age:Light Truck Passenger &  &  &  & $\checkmark$ & \\
\hspace{1em}Age Squared:Light Truck Passenger &  &  &  & $\checkmark$ & \\
\rowcolor{gray!6}  \hspace{1em}Age:Heavy Vehicle Passenger &  &  &  & $\checkmark$ & \\
\hspace{1em}Age Squared:Heavy Vehicle Passenger &  &  &  & $\checkmark$ & \\
\rowcolor{gray!6}  \hspace{1em}Age:Pedestrian &  &  &  & $\checkmark$ & \\
\hspace{1em}Age Squared:Pedestrian &  &  &  & $\checkmark$ & \\
\rowcolor{gray!6}  \hspace{1em}Age:Bicyclist &  &  &  & $\checkmark$ & \\
\hspace{1em}Age Squared:Bicyclist &  &  &  & $\checkmark$ & \\
\rowcolor{gray!6}  \hspace{1em}Age:Motorcyclist &  &  &  & $\checkmark$ & \\
\hspace{1em}Age Squared:Motorcyclist &  &  &  & $\checkmark$ & \\
\rowcolor{gray!6}  \addlinespace[0.3em]
\multicolumn{6}{l}{\textbf{Hierarchical opponent variables)}}\\
\hspace{1em}Age:Age of Opponent &  &  &  &  & $\checkmark$\\
\hspace{1em}Age:Age of Female Opponent &  &  &  &  & $\checkmark$\\
\rowcolor{gray!6}  \hspace{1em}Age:Age of Male Opponent Squared &  &  &  &  & $\checkmark$\\
\hspace{1em}Age:Age of Female Opponent Squared &  &  &  &  & $\checkmark$\\
\rowcolor{gray!6}  \hspace{1em}Age Squared:Age of Male Opponent &  &  &  &  & $\checkmark$\\
\hspace{1em}Age Squared:Age of Female Opponent &  &  &  &  & $\checkmark$\\
\rowcolor{gray!6}  \addlinespace[0.3em]
\multicolumn{6}{l}{\textbf{Case-level variables)}}\\
\hspace{1em}Crash Configuration & 19 levels; Reference: Hit a Moving Object & $\checkmark$ & $\checkmark$ & $\checkmark$ & $\checkmark$\\
\hspace{1em}Road Configuration & 12 levels; Reference: Non-Intersection & $\checkmark$ & $\checkmark$ & $\checkmark$ & $\checkmark$\\
\rowcolor{gray!6}  \hspace{1em}Weather & 9 levels; Reference: Clear and Sunny & $\checkmark$ & $\checkmark$ & $\checkmark$ & $\checkmark$\\
\hspace{1em}Surface & 11 levels; Reference: Dry & $\checkmark$ & $\checkmark$ & $\checkmark$ & $\checkmark$\\
\rowcolor{gray!6}  \hspace{1em}Road Alignment & 8 levels; Reference: Straight and Level & $\checkmark$ & $\checkmark$ & $\checkmark$ & $\checkmark$\\
\hspace{1em}Traffic Controls & 19 levels; Reference: Operational Traffic Signals & $\checkmark$ & $\checkmark$ & $\checkmark$ & $\checkmark$\\
\rowcolor{gray!6}  \hspace{1em}Month & 12 levels; Reference: January & $\checkmark$ & $\checkmark$ & $\checkmark$ & $\checkmark$\\
\bottomrule
\end{tabular}}
\end{table}

Notice how there are zero cases of user: BYCICLIST - opponent: Heavy
Vehicle.

Re-estimate model after subsetting by USER Type of person 1. LDxLD:

Summary of models. See Table \ref{tab:model-summary}.

\begin{table}

\caption{\label{tab:model-summary}\label{tab:model-summary}Summary of model estimation results}
\centering
\begin{tabular}[t]{lccc}
\toprule
Model & \makecell[l]{Number of\\observations} & \makecell[l]{Number of\\coefficients} & AIC\\
\midrule
\rowcolor{gray!6}  \addlinespace[0.3em]
\multicolumn{4}{l}{\textbf{Full sample models}}\\
\hspace{1em}Model 1 & 164,511 & 100 & 195,215\\
\hspace{1em}Model 2 & 164,511 & 106 & 178,943\\
\rowcolor{gray!6}  \hspace{1em}Model 3 & 164,511 & 116 & 181,333\\
\hspace{1em}Model 4 & 164,511 & 109 & 179,018\\
\rowcolor{gray!6}  \addlinespace[0.3em]
\multicolumn{4}{l}{\textbf{Model 1 Ensemble (sample subsets by user type vs opponent)}}\\
\hspace{1em}Light duty vehicle vs light duty vehicle & 114,841 & 95 & 145,396\\
\hspace{1em}Light duty vehicle vs light truck & 3,237 & 95 & 3,979\\
\rowcolor{gray!6}  \hspace{1em}Light duty vehicle vs heavy vehicle & 5,013 & 95 & 5,913\\
\hspace{1em}Light truck vs light duty vehicle & 3,121 & 95 & 3,921\\
\rowcolor{gray!6}  \hspace{1em}Light truck vs light truck & 809 & 95 & 1,230\\
\hspace{1em}Light truck vs heavy vehicle & 198 & 95 & 354\\
\rowcolor{gray!6}  \hspace{1em}Heavy vehicle vs light duty vehicle & 4,763 & 95 & 4,362\\
\hspace{1em}Heavy vehicle vs light truck & 180 & 95 & 291\\
\rowcolor{gray!6}  \hspace{1em}Heavy vehicle vs heavy vehicle & 779 & 95 & 1,193\\
\hspace{1em}Pedestrian vs light duty vehicle & 7,176 & 94 & 2,842\\
\rowcolor{gray!6}  \hspace{1em}Pedestrian vs light truck & 328 & 94 & 270\\
\hspace{1em}Pedestrian vs heavy vehicle & 376 & 94 & 473\\
\rowcolor{gray!6}  \hspace{1em}Bicyclist vs light duty vehicle & 3,521 & 94 & 686\\
\hspace{1em}Bicyclist vs light truck & 148 & 94 & 192\\
\rowcolor{gray!6}  \hspace{1em}Bicyclist vs heavy vehicle & NA & NA & \vphantom{1} NA\\
\hspace{1em}Motorcyclist vs light duty vehicle & 2,298 & 94 & 1,403\\
\rowcolor{gray!6}  \hspace{1em}Motorcyclist vs light truck & 127 & 94 & 233\\
\hspace{1em}Motorcyclist vs heavy vehicle & 73 & 94 & 204\\
\rowcolor{gray!6}  \addlinespace[0.3em]
\multicolumn{4}{l}{\textbf{Model 2 Ensemble (sample subsets by user type vs opponent)}}\\
\hspace{1em}Light duty vehicle vs light duty vehicle & 114,841 & 98 & 143,909\\
\hspace{1em}Light duty vehicle vs light truck & 3,237 & 98 & 3,963\\
\rowcolor{gray!6}  \hspace{1em}Light duty vehicle vs heavy vehicle & 5,013 & 98 & 5,896\\
\hspace{1em}Light truck vs light duty vehicle & 3,121 & 98 & 3,913\\
\rowcolor{gray!6}  \hspace{1em}Light truck vs light truck & 809 & 98 & 1,216\\
\hspace{1em}Light truck vs heavy vehicle & 198 & 98 & 347\\
\rowcolor{gray!6}  \hspace{1em}Heavy vehicle vs light duty vehicle & 4,763 & 98 & 4,315\\
\hspace{1em}Heavy vehicle vs light truck & 180 & 98 & 275\\
\rowcolor{gray!6}  \hspace{1em}Heavy vehicle vs heavy vehicle & 779 & 98 & 1,182\\
\hspace{1em}Pedestrian vs light duty vehicle & 7,176 & 98 & 2,839\\
\rowcolor{gray!6}  \hspace{1em}Pedestrian vs light truck & 328 & 98 & 270\\
\hspace{1em}Pedestrian vs heavy vehicle & 376 & 98 & 476\\
\rowcolor{gray!6}  \hspace{1em}Bicyclist vs light duty vehicle & 3,521 & 98 & 693\\
\hspace{1em}Bicyclist vs light truck & 148 & 98 & 200\\
\rowcolor{gray!6}  Bicyclist vs heavy vehicle & NA & NA & \vphantom{1} NA\\
\hspace{1em}Motorcyclist vs light duty vehicle & 2,298 & 98 & 1,411\\
\rowcolor{gray!6}  \hspace{1em}Motorcyclist vs light truck & 127 & 98 & 235\\
\hspace{1em}Motorcyclist vs heavy vehicle & 73 & 98 & 201\\
\bottomrule
\multicolumn{4}{l}{\textit{Note: }}\\
\multicolumn{4}{l}{There are zero cases of BICYCLIST vs HV in the sample}\\
\end{tabular}
\end{table}

\hypertarget{sec:assessment}{%
\section{Model assessment}\label{sec:assessment}}

\hypertarget{outcome-shares-based-on-probabilities}{%
\subsection{Outcome shares based on
probabilities}\label{outcome-shares-based-on-probabilities}}

Words go here.

The datasets are used to predict the probabilities of the outcomes:

The results of calculating the Average Prediction Error appear in Table
\ref{tab:ape-results}.

\begin{landscape}\begin{table}

\caption{\label{tab:table-ape-results}\label{tab:ape-results}Predicted shares and average prediction errors (APE) by model (percentages)}
\centering
\begin{tabular}[t]{lrrrrrrrrrr}
\toprule
\multicolumn{1}{c}{} & \multicolumn{3}{c}{No Injury} & \multicolumn{3}{c}{Injury} & \multicolumn{3}{c}{Fatality} & \multicolumn{1}{c}{} \\
\cmidrule(l{3pt}r{3pt}){2-4} \cmidrule(l{3pt}r{3pt}){5-7} \cmidrule(l{3pt}r{3pt}){8-10}
Model & Observed & Predicted & APE & Observed & Predicted & APE & Observed & Predicted & APE & WAPE\\
\rowcolor{gray!15}
\midrule
\addlinespace[0.3em]
\multicolumn{11}{l}{\textbf{In-sample (nowcasting using 2017 dataset, i.e., estimation dataset)}}\\
\hspace{1em}Model 1 & 78886 & 79029.00 & 0.18 & 84675 & 84533.74 & 0.17 & 950 & 948.26 & 0.18 & 0.17\\
\rowcolor{gray!15}
\hspace{1em}Model 1 Ensemble & 62449 & 62439.99 & 0.01 & 83606 & 83614.28 & 0.01 & 933 & 933.73 & 0.08 & 0.01\\
\hspace{1em}Model 2 & 78886 & 78928.98 & 0.05 & 84675 & 84641.94 & 0.04 & 950 & 940.08 & 1.04 & 0.05\\
\rowcolor{gray!15}
\hspace{1em}Model 2 Ensemble & 62449 & 62438.99 & 0.02 & 83606 & 83615.22 & 0.01 & 933 & 933.80 & 0.09 & 0.01\\
\hspace{1em}Model 3 & 78886 & 79027.29 & 0.18 & 84675 & 84512.50 & 0.19 & 950 & 971.21 & 2.23 & 0.20\\
\rowcolor{gray!15}
\hspace{1em}Model 4 & 78886 & 78939.18 & 0.07 & 84675 & 84622.54 & 0.06 & 950 & 949.28 & 0.08 & 0.06\\
\rowcolor{gray!15}
\addlinespace[0.3em]
\multicolumn{11}{l}{\textbf{Out-of-sample (backcasting using 2016 dataset)}}\\
\hspace{1em}Model 1 & 82812 & 82574.35 & 0.29 & 88586 & 88737.88 & 0.17 & 935 & 1020.77 & 9.17 & 0.28\\
\rowcolor{gray!15}
\hspace{1em}Model 1 Ensemble & 64469 & 64525.65 & 0.09 & 87476 & 87291.39 & 0.21 & 909 & 1036.96 & 14.08 & 0.24\\
\hspace{1em}Model 2 & 82812 & 82900.86 & 0.11 & 88586 & 88432.82 & 0.17 & 935 & 999.32 & 6.88 & 0.18\\
\rowcolor{gray!15}
\hspace{1em}Model 2 Ensemble & 64469 & 64541.74 & 0.11 & 87476 & 87261.66 & 0.25 & 909 & 1050.60 & 15.58 & 0.28\\
\hspace{1em}Model 3 & 82812 & 82948.83 & 0.17 & 88586 & 88340.07 & 0.28 & 935 & 1044.10 & 11.67 & 0.29\\
\rowcolor{gray!15}
\hspace{1em}Model 4 & 82812 & 82878.68 & 0.08 & 88586 & 88446.37 & 0.16 & 935 & 1007.94 & 7.80 & 0.16\\
\bottomrule
\end{tabular}
\end{table}
\end{landscape}

\hypertarget{predicted-outcomes}{%
\subsection{Predicted outcomes}\label{predicted-outcomes}}

Words go here.

Verification statistics used are summarized in Table
\ref{tab:verification-statistics}.

\begin{landscape}\begin{table}

\caption{\label{tab:table-verification-statistics}\label{tab:verification-statistics}Verification statistics}
\centering
\resizebox{\linewidth}{!}{
\begin{tabular}[t]{l>{\raggedright\arraybackslash}p{30em}>{\raggedright\arraybackslash}p{30em}}
\toprule
Statistic & Description & Notes\\
\midrule
\rowcolor{gray!6}  Percent Correct ($PC$) & Total hits and correct rejections divided by number of cases & Strongly influenced by most common category\\
Percent Correct by Class ($PC_c$) & Same as Percent Correct but by category & Strongly influenced by most common category\\
\rowcolor{gray!6}  Bias ($B$) & Total predicted by category, divided by total observed by category & $B>1$: class is overpredicted; $B<1$: class is underpredicted\\
Critical Success Index ($CSI$) & Total hits divided by total hits + false alarms + misses & $CSI = 1$: perfect score; $CSI = 0$: no skill\\
\rowcolor{gray!6}  Probability of False Detection ($F$) & Proportion of no events forecast as yes; sensitive to false alarms but ignores misses & $F = 0$: perfect score\\
\addlinespace
Probability of Detection ($POD$) & Total hits divided by total observed by class & $POD = 1$: perfect score\\
\rowcolor{gray!6}  False Alarm Ratio ($FAR$) & Total false alarms divided by total forecast yes by class; measures fraction of predicted yes that did not occur & $FAR = 0$: perfect score\\
Heidke Skill Score ($HSS$) & Fraction of correct predictions after removing predictions attributable to chance; measures fractional improvement over random; tends to reward conservative forecasts & $HSS = 1$: perfect score; $HSS = 0$: no skill; $HSS < 0$: random is better\\
\rowcolor{gray!6}  Peirce Skill Score ($PSS$) & Combines $POD$ and $F$; measures ability to separate yes events from no events; tends to reward conservative forecasts & $PSS = 1$: perfect score; $PSS = 0$: no skill\\
Gerrity Score ($GS$) & Measures accuracy of predicting the correct category, relative to random; tends to reward correct forecasts of less likely category & $GS = 1$: perfect score; $GS = 0$: no skill\\
\bottomrule
\end{tabular}}
\end{table}
\end{landscape}

We next evaluate the outcomes of the nowcast using an array of
verification statistics. See Table
\ref{tab:nowcast-outcomes-assessment}.

\begin{table}

\caption{\label{tab:nowcast-outcomes-assessment}\label{tab:nowcast-outcomes-assessment}Assessment of in-sample outcomes (nowcasting using 2017 dataset, i.e., estimation dataset)}
\centering
\resizebox{\linewidth}{!}{
\begin{tabular}[t]{lrrrrrrrrrrrrr}
\toprule
\multicolumn{1}{c}{Observed} & \multicolumn{3}{c}{Predicted Outcome} & \multicolumn{10}{c}{Verification Statistics} \\
\cmidrule(l{3pt}r{3pt}){1-1} \cmidrule(l{3pt}r{3pt}){2-4} \cmidrule(l{3pt}r{3pt}){5-14}
Outcome & No Injury & Injury & Fatality & \makecell[l]{Percent\\Correct} & \makecell[l]{Percent\\Correct\\by Class} & Bias$^1$ & \makecell[l]{Critical\\Success\\Index$^2$} & \makecell[l]{Probability of\\False\\Detection$^3$} & \makecell[l]{Probability\\of\\Detection$^4$} & \makecell[l]{False\\Alarm\\Ratio$^5$} & \makecell[l]{Heidke\\Skill\\Score$^6$} & \makecell[l]{Peirce\\Skill\\Score$^7$} & \makecell[l]{Gerrity\\Score$^8$}\\
\midrule
\addlinespace[0.3em]
\multicolumn{14}{l}{\textbf{Model 1}}\\
\hspace{1em}No Injury & 50652 & 22504 & 150 &  & 69.067 & 0.929 & 0.499 & 0.265 & 0.642 & 0.309 &  &  & \\

\hspace{1em}Injury & 28232 & 62120 & 797 &  & 68.644 & 1.076 & 0.546 & 0.364 & 0.734 & 0.318 &  &  & \\

Fatality & 2 & 51 & 3 & \multirow{-3}{*}{\raggedleft\arraybackslash 68.552} & 99.392 & 0.059 & 0.003 & 0.000 & 0.003 & 0.946 & \multirow{-3}{*}{\raggedleft\arraybackslash 0.372} & \multirow{-3}{*}{\raggedleft\arraybackslash 0.370} & \multirow{-3}{*}{\raggedleft\arraybackslash 0.190}\\
\cmidrule{1-14}
\addlinespace[0.3em]
\multicolumn{14}{l}{\textbf{Model 1 Ensemble}}\\
\hspace{1em}No Injury & 34711 & 16429 & 63 &  & 69.909 & 0.820 & 0.440 & 0.195 & 0.556 & 0.322 &  &  & \\

\hspace{1em}Injury & 27738 & 67167 & 830 &  & 69.380 & 1.145 & 0.599 & 0.451 & 0.803 & 0.298 &  &  & \\

Fatality & 0 & 10 & 40 & \multirow{-3}{*}{\raggedleft\arraybackslash 69.338} & 99.386 & 0.054 & 0.042 & 0.000 & 0.043 & 0.200 & \multirow{-3}{*}{\raggedleft\arraybackslash 0.363} & \multirow{-3}{*}{\raggedleft\arraybackslash 0.353} & \multirow{-3}{*}{\raggedleft\arraybackslash 0.202}\\
\cmidrule{1-14}
\addlinespace[0.3em]
\multicolumn{14}{l}{\textbf{Model 2}}\\
\hspace{1em}No Injury & 51530 & 17136 & 85 &  & 72.903 & 0.872 & 0.536 & 0.201 & 0.653 & 0.250 &  &  & \\

\hspace{1em}Injury & 27356 & 67515 & 864 &  & 72.415 & 1.131 & 0.598 & 0.353 & 0.797 & 0.295 &  &  & \\

Fatality & 0 & 24 & 1 & \multirow{-3}{*}{\raggedleft\arraybackslash 72.364} & 99.409 & 0.026 & 0.001 & 0.000 & 0.001 & 0.960 & \multirow{-3}{*}{\raggedleft\arraybackslash 0.447} & \multirow{-3}{*}{\raggedleft\arraybackslash 0.443} & \multirow{-3}{*}{\raggedleft\arraybackslash 0.227}\\
\cmidrule{1-14}
\addlinespace[0.3em]
\multicolumn{14}{l}{\textbf{Model 2 Ensemble}}\\
\hspace{1em}No Injury & 35473 & 16147 & 60 &  & 70.621 & 0.828 & 0.451 & 0.192 & 0.568 & 0.314 &  &  & \\

\hspace{1em}Injury & 26976 & 67446 & 829 &  & 70.089 & 1.139 & 0.605 & 0.439 & 0.807 & 0.292 &  &  & \\

Fatality & 0 & 13 & 44 & \multirow{-3}{*}{\raggedleft\arraybackslash 70.049} & 99.386 & 0.061 & 0.047 & 0.000 & 0.047 & 0.228 & \multirow{-3}{*}{\raggedleft\arraybackslash 0.379} & \multirow{-3}{*}{\raggedleft\arraybackslash 0.368} & \multirow{-3}{*}{\raggedleft\arraybackslash 0.212}\\
\cmidrule{1-14}
\addlinespace[0.3em]
\multicolumn{14}{l}{\textbf{Model 3}}\\
\hspace{1em}No Injury & 51102 & 17297 & 79 &  & 72.549 & 0.868 & 0.531 & 0.203 & 0.648 & 0.254 &  &  & \\

\hspace{1em}Injury & 27784 & 67337 & 868 &  & 72.044 & 1.134 & 0.594 & 0.359 & 0.795 & 0.298 &  &  & \\

Fatality & 0 & 41 & 3 & \multirow{-3}{*}{\raggedleft\arraybackslash 71.996} & 99.399 & 0.046 & 0.003 & 0.000 & 0.003 & 0.932 & \multirow{-3}{*}{\raggedleft\arraybackslash 0.440} & \multirow{-3}{*}{\raggedleft\arraybackslash 0.436} & \multirow{-3}{*}{\raggedleft\arraybackslash 0.224}\\
\cmidrule{1-14}
\addlinespace[0.3em]
\multicolumn{14}{l}{\textbf{Model 4}}\\
\hspace{1em}No Injury & 51574 & 17317 & 84 &  & 72.821 & 0.874 & 0.536 & 0.203 & 0.654 & 0.252 &  &  & \\

\hspace{1em}Injury & 27312 & 67335 & 863 &  & 72.333 & 1.128 & 0.597 & 0.353 & 0.795 & 0.295 &  &  & \\

Fatality & 0 & 23 & 3 & \multirow{-3}{*}{\raggedleft\arraybackslash 72.282} & 99.410 & 0.027 & 0.003 & 0.000 & 0.003 & 0.885 & \multirow{-3}{*}{\raggedleft\arraybackslash 0.446} & \multirow{-3}{*}{\raggedleft\arraybackslash 0.441} & \multirow{-3}{*}{\raggedleft\arraybackslash 0.227}\\
\bottomrule
\multicolumn{14}{l}{\textit{Notes: }}\\
\multicolumn{14}{l}{\textsuperscript{1} $B>1$: class is overpredicted; $B<1$: class is underpredicted; }\\
\multicolumn{14}{l}{\textsuperscript{2} $CSI = 1$: perfect score; $CSI = 0$: no skill; }\\
\multicolumn{14}{l}{\textsuperscript{3} $F = 0$: perfect score; }\\
\multicolumn{14}{l}{\textsuperscript{4} $POD = 1$: perfect score; }\\
\multicolumn{14}{l}{\textsuperscript{5} $FAR = 0$: perfect score; }\\
\multicolumn{14}{l}{\textsuperscript{6} $HSS = 1$: perfect score; $HSS = 0$: no skill; $HSS < 0$: random is better; }\\
\multicolumn{14}{l}{\textsuperscript{7} $PSS = 1$: perfect score; $PSS = 0$: no skill; }\\
\multicolumn{14}{l}{\textsuperscript{8} $GS = 1$: perfect score; $GS = 0$: no skill.}\\
\end{tabular}}
\end{table}

We next evaluate the outcomes of the nowcast using an array of
verification statistics. See Table
\ref{tab:backcast-outcomes-assessment}.

\begin{table}

\caption{\label{tab:backcast-outcomes-assessment}\label{tab:nowcast-outcomes-assessment}Assessment of in-sample outcomes (nowcasting using 2017 dataset, i.e., estimation dataset)}
\centering
\resizebox{\linewidth}{!}{
\begin{tabular}[t]{lrrrrrrrrrrrrr}
\toprule
\multicolumn{1}{c}{Observed} & \multicolumn{3}{c}{Predicted Outcome} & \multicolumn{10}{c}{Verification Statistics} \\
\cmidrule(l{3pt}r{3pt}){1-1} \cmidrule(l{3pt}r{3pt}){2-4} \cmidrule(l{3pt}r{3pt}){5-14}
Outcome & No Injury & Injury & Fatality & \makecell[l]{Percent\\Correct} & \makecell[l]{Percent\\Correct\\by Class} & Bias$^1$ & \makecell[l]{Critical\\Success\\Index$^2$} & \makecell[l]{Probability of\\False\\Detection$^3$} & \makecell[l]{Probability\\of\\Detection$^4$} & \makecell[l]{False\\Alarm\\Ratio$^5$} & \makecell[l]{Heidke\\Skill\\Score$^6$} & \makecell[l]{Peirce\\Skill\\Score$^7$} & \makecell[l]{Gerrity\\Score$^8$}\\
\midrule
\addlinespace[0.3em]
\multicolumn{14}{l}{\textbf{Model 1}}\\
\hspace{1em}No Injury & 53167 & 23097 & 145 &  & 69.311 & 0.923 & 0.501 & 0.260 & 0.642 & 0.304 &  &  & \\

\hspace{1em}Injury & 29642 & 65455 & 788 &  & 68.920 & 1.082 & 0.550 & 0.363 & 0.739 & 0.317 &  &  & \\

Fatality & 3 & 34 & 2 & \multirow{-3}{*}{\raggedleft\arraybackslash 68.834} & 99.437 & 0.042 & 0.002 & 0.000 & 0.002 & 0.949 & \multirow{-3}{*}{\raggedleft\arraybackslash 0.378} & \multirow{-3}{*}{\raggedleft\arraybackslash 0.375} & \multirow{-3}{*}{\raggedleft\arraybackslash 0.192}\\
\cmidrule{1-14}
\addlinespace[0.3em]
\multicolumn{14}{l}{\textbf{Model 1 Ensemble}}\\
\hspace{1em}No Injury & 35564 & 16930 & 71 &  & 69.967 & 0.815 & 0.437 & 0.192 & 0.552 & 0.323 &  &  & \\

\hspace{1em}Injury & 28891 & 70451 & 822 &  & 69.423 & 1.145 & 0.601 & 0.454 & 0.805 & 0.297 &  &  & \\

Fatality & 14 & 95 & 16 & \multirow{-3}{*}{\raggedleft\arraybackslash 69.368} & 99.344 & 0.138 & 0.016 & 0.001 & 0.018 & 0.872 & \multirow{-3}{*}{\raggedleft\arraybackslash 0.362} & \multirow{-3}{*}{\raggedleft\arraybackslash 0.351} & \multirow{-3}{*}{\raggedleft\arraybackslash 0.188}\\
\cmidrule{1-14}
\addlinespace[0.3em]
\multicolumn{14}{l}{\textbf{Model 2}}\\
\hspace{1em}No Injury & 54430 & 17615 & 81 &  & 73.262 & 0.871 & 0.542 & 0.198 & 0.657 & 0.245 &  &  & \\

\hspace{1em}Injury & 28382 & 70954 & 850 &  & 72.806 & 1.131 & 0.602 & 0.349 & 0.801 & 0.292 &  &  & \\

Fatality & 0 & 17 & 4 & \multirow{-3}{*}{\raggedleft\arraybackslash 72.759} & 99.450 & 0.022 & 0.004 & 0.000 & 0.004 & 0.810 & \multirow{-3}{*}{\raggedleft\arraybackslash 0.455} & \multirow{-3}{*}{\raggedleft\arraybackslash 0.451} & \multirow{-3}{*}{\raggedleft\arraybackslash 0.232}\\
\cmidrule{1-14}
\addlinespace[0.3em]
\multicolumn{14}{l}{\textbf{Model 2 Ensemble}}\\
\hspace{1em}No Injury & 36234 & 16840 & 73 &  & 70.463 & 0.824 & 0.445 & 0.191 & 0.562 & 0.318 &  &  & \\

\hspace{1em}Injury & 28216 & 70527 & 819 &  & 69.916 & 1.138 & 0.605 & 0.444 & 0.806 & 0.292 &  &  & \\

Fatality & 19 & 109 & 17 & \multirow{-3}{*}{\raggedleft\arraybackslash 69.856} & 99.333 & 0.160 & 0.016 & 0.001 & 0.019 & 0.883 & \multirow{-3}{*}{\raggedleft\arraybackslash 0.373} & \multirow{-3}{*}{\raggedleft\arraybackslash 0.362} & \multirow{-3}{*}{\raggedleft\arraybackslash 0.194}\\
\cmidrule{1-14}
\addlinespace[0.3em]
\multicolumn{14}{l}{\textbf{Model 3}}\\
\hspace{1em}No Injury & 53985 & 17731 & 83 &  & 72.936 & 0.867 & 0.536 & 0.199 & 0.652 & 0.248 &  &  & \\

\hspace{1em}Injury & 28827 & 70819 & 849 &  & 72.470 & 1.134 & 0.599 & 0.354 & 0.799 & 0.295 &  &  & \\

Fatality & 0 & 36 & 3 & \multirow{-3}{*}{\raggedleft\arraybackslash 72.422} & 99.438 & 0.042 & 0.003 & 0.000 & 0.003 & 0.923 & \multirow{-3}{*}{\raggedleft\arraybackslash 0.448} & \multirow{-3}{*}{\raggedleft\arraybackslash 0.444} & \multirow{-3}{*}{\raggedleft\arraybackslash 0.228}\\
\cmidrule{1-14}
\addlinespace[0.3em]
\multicolumn{14}{l}{\textbf{Model 4}}\\
\hspace{1em}No Injury & 54493 & 17843 & 79 &  & 73.168 & 0.874 & 0.541 & 0.200 & 0.658 & 0.247 &  &  & \\

\hspace{1em}Injury & 28318 & 70725 & 852 &  & 72.709 & 1.128 & 0.601 & 0.348 & 0.798 & 0.292 &  &  & \\

Fatality & 1 & 18 & 4 & \multirow{-3}{*}{\raggedleft\arraybackslash 72.663} & 99.449 & 0.025 & 0.004 & 0.000 & 0.004 & 0.826 & \multirow{-3}{*}{\raggedleft\arraybackslash 0.453} & \multirow{-3}{*}{\raggedleft\arraybackslash 0.449} & \multirow{-3}{*}{\raggedleft\arraybackslash 0.231}\\
\bottomrule
\multicolumn{14}{l}{\textit{Notes: }}\\
\multicolumn{14}{l}{\textsuperscript{1} $B>1$: class is overpredicted; $B<1$: class is underpredicted; }\\
\multicolumn{14}{l}{\textsuperscript{2} $CSI = 1$: perfect score; $CSI = 0$: no skill; }\\
\multicolumn{14}{l}{\textsuperscript{3} $F = 0$: perfect score; }\\
\multicolumn{14}{l}{\textsuperscript{4} $POD = 1$: perfect score; }\\
\multicolumn{14}{l}{\textsuperscript{5} $FAR = 0$: perfect score; }\\
\multicolumn{14}{l}{\textsuperscript{6} $HSS = 1$: perfect score; $HSS = 0$: no skill; $HSS < 0$: random is better; }\\
\multicolumn{14}{l}{\textsuperscript{7} $PSS = 1$: perfect score; $PSS = 0$: no skill; }\\
\multicolumn{14}{l}{\textsuperscript{8} $GS = 1$: perfect score; $GS = 0$: no skill.}\\
\end{tabular}}
\end{table}

\hypertarget{sec:further-considerations}{%
\section{Further considerations}\label{sec:further-considerations}}

Here I plan to discuss the applicability of the modelling strategy to
advanced modelling techniques (partial proportional odds, heterogeneity,
hierarchical models, etc.)

\hypertarget{sec:concluding-remarks}{%
\section{Concluding remarks}\label{sec:concluding-remarks}}

Words go here.

\hypertarget{references}{%
\section*{References}\label{references}}
\addcontentsline{toc}{section}{References}

\hypertarget{refs}{}
\leavevmode\hypertarget{ref-Aziz2013exploring}{}%
Aziz, H.M.A., Ukkusuri, S.V., Hasan, S., 2013. Exploring the
determinants of pedestrian-vehicle crash severity in new york city.
Accident Analysis and Prevention 50, 1298--1309.
doi:\href{https://doi.org/10.1016/j.aap.2012.09.034}{10.1016/j.aap.2012.09.034}

\leavevmode\hypertarget{ref-Bogue2017modified}{}%
Bogue, S., Paleti, R., Balan, L., 2017. A modified rank ordered logit
model to analyze injury severity of occupants in multivehicle crashes.
Analytic Methods in Accident Research 14, 22--40.
doi:\href{https://doi.org/10.1016/j.amar.2017.03.001}{10.1016/j.amar.2017.03.001}

\leavevmode\hypertarget{ref-Chang2006analysis}{}%
Chang, L.Y., Wang, H.W., 2006. Analysis of traffic injury severity: An
application of non-parametric classification tree techniques. Accident
Analysis and Prevention 38, 1019--1027.
doi:\href{https://doi.org/10.1016/j.aap.2006.04.009}{10.1016/j.aap.2006.04.009}

\leavevmode\hypertarget{ref-Chen2019investigation}{}%
Chen, F., Song, M.T., Ma, X.X., 2019. Investigation on the injury
severity of drivers in rear-end collisions between cars using a random
parameters bivariate ordered probit model. International Journal of
Environmental Research and Public Health 16.
doi:\href{https://doi.org/10.3390/ijerph16142632}{10.3390/ijerph16142632}

\leavevmode\hypertarget{ref-Chiou2013modeling}{}%
Chiou, Y.C., Hwang, C.C., Chang, C.C., Fu, C., 2013. Modeling
two-vehicle crash severity by a bivariate generalized ordered probit
approach. Accident Analysis and Prevention 51, 175--184.
doi:\href{https://doi.org/10.1016/j.aap.2012.11.008}{10.1016/j.aap.2012.11.008}

\leavevmode\hypertarget{ref-Devlin2019road}{}%
Devlin, A., Beck, B., Simpson, P.M., Ekegren, C.L., Giummarra, M.J.,
Edwards, E.R., Cameron, P.A., Liew, S., Oppy, A., Richardson, M., Page,
R., Gabbe, B.J., 2019. The road to recovery for vulnerable road users
hospitalised for orthopaedic injury following an on-road crash. Accident
Analysis and Prevention 132, 10.
doi:\href{https://doi.org/10.1016/j.aap.2019.105279}{10.1016/j.aap.2019.105279}

\leavevmode\hypertarget{ref-Effati2015geospatial}{}%
Effati, M., Thill, J.C., Shabani, S., 2015. Geospatial and machine
learning techniques for wicked social science problems: Analysis of
crash severity on a regional highway corridor. Journal of Geographical
Systems 17, 107--135.
doi:\href{https://doi.org/10.1007/s10109-015-0210-x}{10.1007/s10109-015-0210-x}

\leavevmode\hypertarget{ref-Gong2017modeling}{}%
Gong, L.F., Fan, W.D., 2017. Modeling single-vehicle run-off-road crash
severity in rural areas: Accounting for unobserved heterogeneity and age
difference. Accident Analysis and Prevention 101, 124--134.
doi:\href{https://doi.org/10.1016/j.aap.2017.02.014}{10.1016/j.aap.2017.02.014}

\leavevmode\hypertarget{ref-Haleem2013effect}{}%
Haleem, K., Gan, A., 2013. Effect of driver's age and side of impact on
crash severity along urban freeways: A mixed logit approach. Journal of
Safety Research 46, 67--76.
doi:\href{https://doi.org/10.1016/j.jsr.2013.04.002}{10.1016/j.jsr.2013.04.002}

\leavevmode\hypertarget{ref-Hanson2013severity}{}%
Hanson, C.S., Noland, R.B., Brown, C., 2013. The severity of pedestrian
crashes: An analysis using google street view imagery. Journal of
Transport Geography 33, 42--53.
doi:\href{https://doi.org/10.1016/j.jtrangeo.2013.09.002}{10.1016/j.jtrangeo.2013.09.002}

\leavevmode\hypertarget{ref-Iranitalab2017comparison}{}%
Iranitalab, A., Khattak, A., 2017. Comparison of four statistical and
machine learning methods for crash severity prediction. Accident
Analysis and Prevention 108, 27--36.
doi:\href{https://doi.org/10.1016/j.aap.2017.08.008}{10.1016/j.aap.2017.08.008}

\leavevmode\hypertarget{ref-Khan2015exploring}{}%
Khan, G., Bill, A.R., Noyce, D.A., 2015. Exploring the feasibility of
classification trees versus ordinal discrete choice models for analyzing
crash severity. Transportation Research Part C-Emerging Technologies 50,
86--96.
doi:\href{https://doi.org/10.1016/j.trc.2014.10.003}{10.1016/j.trc.2014.10.003}

\leavevmode\hypertarget{ref-Kim2013driver}{}%
Kim, J.K., Ulfarsson, G.F., Kim, S., Shankar, V.N., 2013. Driver-injury
severity in single-vehicle crashes in california: A mixed logit analysis
of heterogeneity due to age and gender. Accident Analysis and Prevention
50, 1073--1081.
doi:\href{https://doi.org/10.1016/j.aap.2012.08.011}{10.1016/j.aap.2012.08.011}

\leavevmode\hypertarget{ref-Kim1995personal}{}%
Kim, K., Nitz, L., Richardson, J., Li, L., 1995. PERSONAL and behavioral
predictors of automobile crash and injury severity. Accident Analysis
and Prevention 27, 469--481.
doi:\href{https://doi.org/10.1016/0001-4575(95)00001-g}{10.1016/0001-4575(95)00001-g}

\leavevmode\hypertarget{ref-Lee2014analysis}{}%
Lee, C., Li, X.C., 2014. Analysis of injury severity of drivers involved
in single- and two-vehicle crashes on highways in ontario. Accident
Analysis and Prevention 71, 286--295.
doi:\href{https://doi.org/10.1016/j.aap.2014.06.008}{10.1016/j.aap.2014.06.008}

\leavevmode\hypertarget{ref-Li2017interplay}{}%
Li, L., Hasnine, M.S., Habib, K.M.N., Persaud, B., Shalaby, A., 2017.
Investigating the interplay between the attributes of at-fault and
not-at-fault drivers and the associated impacts on crash injury
occurrence and severity level. Journal of Transportation Safety \&
Security 9, 439--456.
doi:\href{https://doi.org/10.1080/19439962.2016.1237602}{10.1080/19439962.2016.1237602}

\leavevmode\hypertarget{ref-Ma2008multivariate}{}%
Ma, J.M., Kockelman, K.M., Damien, P., 2008. A multivariate
poisson-lognormal regression model for prediction of crash counts by
severity, using bayesian methods. Accident Analysis and Prevention 40,
964--975.
doi:\href{https://doi.org/10.1016/j.aap.2007.11.002}{10.1016/j.aap.2007.11.002}

\leavevmode\hypertarget{ref-Maddala1986limited}{}%
Maddala, G.S., 1986. Limited-dependent and qualitative variables in
econometrics. Cambridge university press.

\leavevmode\hypertarget{ref-Mannering2016unobserved}{}%
Mannering, F., Shankar, V., Bhat, C.R., 2016. Unobserved heterogeneity
and the statistical analysis of highway accident data. Analytic Methods
in Accident Research 11, 1--16.
doi:\href{https://doi.org/10.1016/j.amar.2016.04.001}{10.1016/j.amar.2016.04.001}

\leavevmode\hypertarget{ref-McArthur2014spatial}{}%
McArthur, A., Savolainen, P.T., Gates, T.J., 2014. Spatial analysis of
child pedestrian and bicycle crashes development of safety performance
function for areas adjacent to schools. Transportation Research Record
57--63. doi:\href{https://doi.org/10.3141/2465-08}{10.3141/2465-08}

\leavevmode\hypertarget{ref-Merlin2007stress}{}%
Merlin, E.P.R., Gonzalez-Forteza, C., Lira, L.R., Tapia, J.A.J., 2007.
Post-traumatic stress disorder in patients with non intentional injuries
caused by road traffic accidents. Salud Mental 30, 43--48.

\leavevmode\hypertarget{ref-Montella2013crash}{}%
Montella, A., Andreassen, D., Tarko, A.P., Turner, S., Mauriello, F.,
Imbriani, L.L., Romero, M.A., 2013. Crash databases in australasia, the
european union, and the united states review and prospects for
improvement. Transportation Research Record 128--136.
doi:\href{https://doi.org/10.3141/2386-15}{10.3141/2386-15}

\leavevmode\hypertarget{ref-Mooradian2013analysis}{}%
Mooradian, J., Ivan, J.N., Ravishanker, N., Hu, S., 2013. Analysis of
driver and passenger crash injury severity using partial proportional
odds models. Accident Analysis and Prevention 58, 53--58.
doi:\href{https://doi.org/10.1016/j.aap.2013.04.022}{10.1016/j.aap.2013.04.022}

\leavevmode\hypertarget{ref-Mussone2017analysis}{}%
Mussone, L., Bassani, M., Masci, P., 2017. Analysis of factors affecting
the severity of crashes in urban road intersections. Accident Analysis
and Prevention 103, 112--122.
doi:\href{https://doi.org/10.1016/j.aap.2017.04.007}{10.1016/j.aap.2017.04.007}

\leavevmode\hypertarget{ref-Obeng2011gender}{}%
Obeng, K., 2011. Gender differences in injury severity risks in crashes
at signalized intersections. Accident Analysis and Prevention 43,
1521--1531.
doi:\href{https://doi.org/10.1016/j.aap.2011.03.004}{10.1016/j.aap.2011.03.004}

\leavevmode\hypertarget{ref-Peek-Asa2010teenage}{}%
Peek-Asa, C., Britton, C., Young, T., Pawlovich, M., Falb, S., 2010.
Teenage driver crash incidence and factors influencing crash injury by
rurality. Journal of Safety Research 41, 487--492.
doi:\href{https://doi.org/10.1016/j.jsr.2010.10.002}{10.1016/j.jsr.2010.10.002}

\leavevmode\hypertarget{ref-Pelissier2019medical}{}%
Pelissier, C., Fort, E., Fontana, L., Hours, M., n.d. Medical and
socio-occupational predictive factors of psychological distress 5 years
after a road accident: A prospective study. Social Psychiatry and
Psychiatric Epidemiology 13.
doi:\href{https://doi.org/10.1007/s00127-019-01780-0}{10.1007/s00127-019-01780-0}

\leavevmode\hypertarget{ref-Rakotonirainy2012older}{}%
Rakotonirainy, A., Steinhardt, D., Delhomme, P., Darvell, M., Schramm,
A., 2012. Older drivers' crashes in queensland, australia. Accident
Analysis and Prevention 48, 423--429.
doi:\href{https://doi.org/10.1016/j.aap.2012.02.016}{10.1016/j.aap.2012.02.016}

\leavevmode\hypertarget{ref-Regev2018crash}{}%
Regev, S., Rolison, J.J., Moutari, S., 2018. Crash risk by driver age,
gender, and time of day using a new exposure methodology. Journal of
Safety Research 66, 131--140.
doi:\href{https://doi.org/10.1016/j.jsr.2018.07.002}{10.1016/j.jsr.2018.07.002}

\leavevmode\hypertarget{ref-Rifaat2007accident}{}%
Rifaat, S.M., Chin, H.C., 2007. Accident severity analysis using ordered
probit model. Journal of Advanced Transportation 41, 91--114.
doi:\href{https://doi.org/10.1002/atr.5670410107}{10.1002/atr.5670410107}

\leavevmode\hypertarget{ref-Salon2018determinants}{}%
Salon, D., McIntyre, A., 2018. Determinants of pedestrian and bicyclist
crash severity by party at fault in san francisco, ca. Accident Analysis
and Prevention 110, 149--160.
doi:\href{https://doi.org/10.1016/j.aap.2017.11.007}{10.1016/j.aap.2017.11.007}

\leavevmode\hypertarget{ref-Sasidharan2014partial}{}%
Sasidharan, L., Menendez, M., 2014. Partial proportional odds model-an
alternate choice for analyzing pedestrian crash injury severities.
Accident Analysis and Prevention 72, 330--340.
doi:\href{https://doi.org/10.1016/j.aap.2014.07.025}{10.1016/j.aap.2014.07.025}

\leavevmode\hypertarget{ref-Savolainen2011statistical}{}%
Savolainen, P.T., Mannering, F., Lord, D., Quddus, M.A., 2011. The
statistical analysis of highway crash-injury severities: A review and
assessment of methodological alternatives. Accident Analysis and
Prevention 43, 1666--1676.
doi:\href{https://doi.org/10.1016/j.aap.2011.03.025}{10.1016/j.aap.2011.03.025}

\leavevmode\hypertarget{ref-Shaheed2013mixed}{}%
Shaheed, M.S.B., Gkritza, K., Zhang, W., Hans, Z., 2013. A mixed logit
analysis of two-vehicle crash seventies involving a motorcycle. Accident
Analysis and Prevention 61, 119--128.
doi:\href{https://doi.org/10.1016/j.aap.2013.05.028}{10.1016/j.aap.2013.05.028}

\leavevmode\hypertarget{ref-Symons2019reduced}{}%
Symons, J., Howard, E., Sweeny, K., Kumnick, M., Sheehan, P., 2019.
Reduced road traffic injuries for young people: A preliminary investment
analysis. Journal of Adolescent Health 65, S34--S43.
doi:\href{https://doi.org/10.1016/j.jadohealth.2019.01.009}{10.1016/j.jadohealth.2019.01.009}

\leavevmode\hypertarget{ref-Tarrao2014modeling}{}%
Tarrao, G.A., Coelho, M.C., Rouphail, N.M., 2014. Modeling the impact of
subject and opponent vehicles on crash severity in two-vehicle
collisions. Transportation Research Record 53--64.
doi:\href{https://doi.org/10.3141/2432-07}{10.3141/2432-07}

\leavevmode\hypertarget{ref-Thompson2018trends}{}%
Thompson, J.P., Baldock, M.R.J., Dutschke, J.K., 2018. Trends in the
crash involvement of older drivers in australia. Accident Analysis and
Prevention 117, 262--269.
doi:\href{https://doi.org/10.1016/j.aap.2018.04.027}{10.1016/j.aap.2018.04.027}

\leavevmode\hypertarget{ref-Train2009discrete}{}%
Train, K., 2009. Discrete choice methods with simulation, 2nd Edition.
ed. Cambridge University Press, Cambridge.

\leavevmode\hypertarget{ref-Wang2015copula}{}%
Wang, K., Yasmin, S., Konduri, K.C., Eluru, N., Ivan, J.N., 2015.
Copula-based joint model of injury severity and vehicle damage in
two-vehicle crashes. Transportation Research Record 158--166.
doi:\href{https://doi.org/10.3141/2514-17}{10.3141/2514-17}

\leavevmode\hypertarget{ref-White1972effects}{}%
White, S., Clayton, S., 1972. Some effects of alcohol, age of driver,
and estimated speed on the likelihood of driver injury. Accident
Analysis \& Prevention 4.

\leavevmode\hypertarget{ref-Wijnen2019analysis}{}%
Wijnen, W., Weijermars, W., Schoeters, A., Berghe, W. van den, Bauer,
R., Carnis, L., Elvik, R., Martensen, H., 2019. An analysis of official
road crash cost estimates in european countries. Safety Science 113,
318--327.
doi:\href{https://doi.org/10.1016/j.ssci.2018.12.004}{10.1016/j.ssci.2018.12.004}

\leavevmode\hypertarget{ref-WHO2019global}{}%
World Health Organization, 2019. Global status report on road safety
2018 (2018). Geneva.

\leavevmode\hypertarget{ref-Wu2014mixed}{}%
Wu, Q., Chen, F., Zhang, G.H., Liu, X.Y.C., Wang, H., Bogus, S.M., 2014.
Mixed logit model-based driver injury severity investigations in single-
and multi-vehicle crashes on rural two-lane highways. Accident Analysis
and Prevention 72, 105--115.
doi:\href{https://doi.org/10.1016/j.aap.2014.06.014}{10.1016/j.aap.2014.06.014}

\leavevmode\hypertarget{ref-Shamsunnahar2013evaluating}{}%
Yasmin, S., Eluru, N., 2013. Evaluating alternate discrete outcome
frameworks for modeling crash injury severity. Accident Analysis \&
Prevention 59, 506--521.
doi:\href{https://doi.org/https://doi.org/10.1016/j.aap.2013.06.040}{https://doi.org/10.1016/j.aap.2013.06.040}


\end{document}


