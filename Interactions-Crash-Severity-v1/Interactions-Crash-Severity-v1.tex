\documentclass[]{elsarticle} %review=doublespace preprint=single 5p=2 column
%%% Begin My package additions %%%%%%%%%%%%%%%%%%%
\usepackage[hyphens]{url}

  \journal{Some Journal} % Sets Journal name


\usepackage{lineno} % add
\providecommand{\tightlist}{%
  \setlength{\itemsep}{0pt}\setlength{\parskip}{0pt}}

\usepackage{graphicx}
\usepackage{booktabs} % book-quality tables
%%%%%%%%%%%%%%%% end my additions to header

\usepackage[T1]{fontenc}
\usepackage{lmodern}
\usepackage{amssymb,amsmath}
\usepackage{ifxetex,ifluatex}
\usepackage{fixltx2e} % provides \textsubscript
% use upquote if available, for straight quotes in verbatim environments
\IfFileExists{upquote.sty}{\usepackage{upquote}}{}
\ifnum 0\ifxetex 1\fi\ifluatex 1\fi=0 % if pdftex
  \usepackage[utf8]{inputenc}
\else % if luatex or xelatex
  \usepackage{fontspec}
  \ifxetex
    \usepackage{xltxtra,xunicode}
  \fi
  \defaultfontfeatures{Mapping=tex-text,Scale=MatchLowercase}
  \newcommand{\euro}{€}
\fi
% use microtype if available
\IfFileExists{microtype.sty}{\usepackage{microtype}}{}
\bibliographystyle{elsarticle-harv}
\ifxetex
  \usepackage[setpagesize=false, % page size defined by xetex
              unicode=false, % unicode breaks when used with xetex
              xetex]{hyperref}
\else
  \usepackage[unicode=true]{hyperref}
\fi
\hypersetup{breaklinks=true,
            bookmarks=true,
            pdfauthor={},
            pdftitle={An empirical assessment of strategies to model opponent effects in crash severity analysis},
            colorlinks=false,
            urlcolor=blue,
            linkcolor=magenta,
            pdfborder={0 0 0}}
\urlstyle{same}  % don't use monospace font for urls

\setcounter{secnumdepth}{5}
% Pandoc toggle for numbering sections (defaults to be off)


% Pandoc header
\usepackage[margin=1in]{geometry}
\usepackage{lineno}
\linenumbers
\usepackage{booktabs}
\usepackage{longtable}
\usepackage{array}
\usepackage{multirow}
\usepackage{wrapfig}
\usepackage{float}
\usepackage{colortbl}
\usepackage{pdflscape}
\usepackage{tabu}
\usepackage{threeparttable}
\usepackage{threeparttablex}
\usepackage[normalem]{ulem}
\usepackage{makecell}
\usepackage{xcolor}



\begin{document}
\begin{frontmatter}

  \title{An empirical assessment of strategies to model opponent effects in crash
severity analysis}
    \author[McMaster University]{Antonio Paez\corref{Corresponding Author}}
   \ead{paezha@mcmaster.ca} 
    \author[Louisiana State University]{Hany Hassan}
   \ead{hassan1@lsu.edu} 
    \author[McMaster University]{Mark Ferguson}
   \ead{fergumr@mcmaster.ca} 
    \author[McMaster University]{Saiedeh Razavi}
   \ead{razavi@mcmaster.ca} 
      \address[McMaster University]{McMaster Institute for Transportation and Logistics, 1280 Main Street
West, Hamilton, Ontario, Canada L8S 4K1}
    \address[Louisiana State University]{Department of Civil and Environmental Engineering, Baton Rouge,
Louisiana, USA 70803}
    
  \begin{abstract}
  Road accidents impose an important burden on health and the economy.
  Numerous efforts to understand the factors that affect road collisions
  have been undertaken. One stream of research focus on modelling the
  severity of crashes. Crash severity research is useful to clarify the
  way different factors can influence the outcome of an event. The
  objective of this paper is to assess different strategies to model the
  interactions between participants in a crash in the context of crashes
  involving two parties. Towards this objective, a series of models are
  estimated using data from Canada's National Collision Database. Three
  levels of crash severity (no injury/injury/fatality) are analyzed using
  ordered logit models and covariates for the participants in the crash
  and the conditions of the crash. Modelling strategies include different
  ways of introducing the covariates (e.g., in a single-level or
  multi-level form), as well as by subsetting the dataset. The models are
  assessed using predicted shares and classes of outcomes, and the results
  highlight the importance of considering opponent effects in crash
  severity analysis. The study also suggests that hierarchical (i.e.,
  multi-level) specifications and subsetting do not necessarily perform
  better than a relatively simple single-level model with opponent
  effects. The results of this study provide insights regarding the
  performace of different modelling strategies, and should be informative
  to researchers working with crash severity models.
  \end{abstract}
  
 \end{frontmatter}

\hypertarget{introduction}{%
\section{Introduction}\label{introduction}}

Road safety continues to be a world-wide concern. According to a recent
report from the World Health Organization (2019), road accidents are the
8th leading cause of death for all ages, and the number one cause of
death for children and young people between the ages of 5 to 29. Of all
leading causes of death, road accidents are the only cause of death
unrelated to disease, health disorder, or infection. For this reason,
road accidents impose a heavy burden on individuals and society as a
whole. Gobally, the rate of road collision-related deaths per 100,000
population and 100,000 vehicles have both fallen, even as the number of
vehicles has grown (World Health Organization, 2019, Figs. 1 and 2).
These gains, although they are to be celebrated, cannot distract from
the crushing economic cost of premature death (e.g., Symons et al.,
2019; Wijnen et al., 2019), not to mention the long-term consequences
for survivors, measured in sometimes crippling emotional and physical
pain (e.g., Merlin et al., 2007; Devlin et al., 2019; Pelissier et al.,
n.d.).

Evidence from across the world suggests that the burden of road
accidents is not borne evenly. There are important disparities at the
international level, where the odds of death due to road crashes are
three times higher in low-income countries compared to high-income
countries; in fact, no reductions in road accident-related fatalities
were appreciated in low-income countries between 2013 and 2016 (World
Health Organization, 2019). In the case of high-income countries, where
substantial gains in road safety have been observed for years, said
gains have also been unevenly distributed; thus, while fatal crashes
involving older adults in the United States and Great Britain declined
between 1997 and 2010 (despite the graying of the population), the trend
remained stable or increased slightly in Australia in roughly the same
period (Thompson et al., 2018). There are also systematic differences in
the impact of road accidents. For example, in a study in the United
States, Obeng (2011) reported that the impact of covariates of crash
severity varied substantially in magnitude by gender. More recently,
Regev et al.~(2018) used adjusted crash risk to find that the risk of
crashes in Great Britain peaked for people 21 to 29 years of age; on the
other hand, the risk of fatal injuries for older drivers was constant,
irrespective of the seriousness of the crash - which highlights the
perils of accidents at older ages. Other studies have concentrated on
the consequences of road accidents for the young (e.g., Peek-Asa et al.,
2010), the old (e.g., Rakotonirainy et al., 2012), as well as
pedestrians and cyclists (e.g., Hanson et al., 2013; McArthur et al.,
2014).

Given the relevance and cost of this matter, as well as the important
variations of the impacts among different population segments, numerous
efforts have been conducted to better understand the factors that affect
road safety - including the probable consequences of crashes. Along
these lines, a stream of research in the analysis of road accidents is
concerned with the severity of crashes. In particular, multivariate
analysis of crash severity is a useful way to clarify the way various
factors can affect the outcome of an incident, and to discriminate
between various levels of injury, from no injury (i.e., property damage
only), to different degrees of injury up to and including death. This is
an active area of research, and one where methodological developments
have aimed at improving the reliability, accuracy, and precision of
models (e.g., Savolainen et al., 2011; Bogue et al., 2017; Shamsunnahar
and Eluru, 2013).

This paper aspires to contribute to the literature on crash severity by
assessing different modelling strategies useful to incorporate opponent
effects in crash analysis, in the context of incidents involving two
parties. The importance of these interactions has been recognized in the
existing literature (e.g., Chiou et al., 2013; Lee and Li, 2014; Li et
al., 2017; Tarrao et al., 2014), and a number of different modelling
strategies have been proposed. In this paper we present a systematic
assessment of several relevant modelling strategies, ranging from the
way variables are defined in single-level models, in multi-level models
(i.e., hierarchical models), as well as using data subsetting
approaches. For the assessment we use data from Canada's National
Collision Database, a database that collects all police-reported
collisions in the country. Using the most recent version of the dataset
(2017), three levels of crash severity (no injury/injury/fatality) are
analyzed using ordered logit models and covariates for the participants
in the crash and the conditions of the crash. For model assessment, we
conduct an in-sample prediction exercise using the estimation sample
(i.e., \emph{nowcasting}), and also an out-of-sample prediction exercise
using the dataset corresponding to 2016 (i.e., \emph{backcasting}). The
models are assessed using predicted shares and predicted classes of
outcomes at the individual level, using an extensive array of
verification statistics. The results highlight the importance of
considering opponent effects in crash severity analysis to improve the
goodness-of-fit and predictive performance o. On the other hand, the
study suggests that hierarchical variable specifications and subsetting
do not perform necessarily better than a relatively simple single-level
model with opponent effects.

The rest of this paper is structured as follows. In Section
\ref{sec:review-of-methods} we present a concise review of the methods
used to analyze crash severity, with a particular focus on techniques
that consider the effect of opponents in a crash. Section
\ref{sec:application} describes the data requirements, data
preprocessing, and the modelling strategies, along with the results of
model estimation. The results of assessing the models and the discussion
of these results is found in Section \ref{sec:assessment}. We then
present some additional thoughts about the applicability of this
approach in Section \ref{sec:further-considerations} before offering
some concluding remarks in Section \ref{sec:concluding-remarks}.

\hypertarget{sec:review-of-methods}{%
\section{Methodological approaches in crash severity
analysis}\label{sec:review-of-methods}}

\hypertarget{general-considerations}{%
\subsection{General considerations}\label{general-considerations}}

Modelling the outcomes of crashes in terms of the severity of injuries
to participants has been a preoccupation of transportation researchers,
planners, auto insurance companies, governments, and the general public
for decades. One of the earliests studies to investigate the severity of
injuries conditional on an accident having occurred was by White and
Clayton (1972). Kim et al.~(1995) later stated that the ``linkages
between severity of injury and driver characteristics and behaviors have
not been thoroughly investigated'' (p.~470). Nowadays, there is a
burgeoning literature on this subject, including methodological
developments, case studies, and more niche research with a focus on
particular situations (e.g., crashes at intersections, Mussone et al.,
2017; crashes in rural roads, Gong and Fan, 2017), and crashes involving
special population groups (e.g., crashes involving motorcyclists or
active travelers; see Shaheed et al., 2013; Salon and McIntyre, 2018).

Crash severity is often modelled using models for discrete outcomes. An
analyst interested in crash severity has at their disposal an ample menu
of models to choose from, including classification techniques from
machine learning (e.g., Iranitalab and Khattak, 2017; Chang and Wang,
2006; Effati et al., 2015; Khan et al., 2015), Poisson models for counts
(e.g., Ma et al., 2008), unordered logit/probit models (e.g., Tay et
al., 2011), as well as ordered logit/probit models (e.g., Rifaat and
Chin, 2007), with numerous variants, such as random parameters/mixed
logit (e.g., Aziz et al., 2013; Haleem and Gan, 2013), partial
proportional odds models (e.g., Mooradian et al., 2013; Sasidharan and
Menendez, 2014), and the use of copulas (e.g., Wang et al., 2015).
Recent reviews of methods include Savolainen et al.~(2011), Shamsunnahar
and Eluru (2013), and Mannering et al.~(2016).

Irrespective of the modelling framework employed, models of crash
severity often include variables in several categories, as shown with
examples in Table \ref{tab:variable-categories} (also see Montella et
al., 2013). Many crash databases and analyses also account for the
multievent nature of many crashes. Participants may have had different
roles in a crash depending on their context, with some acting as
operators of a vehicle (i.e., drivers, bicyclists), while others were
passengers. They also may differ depending on what type of traffic unit
they were, for example occupants of a light duty vehicle or a truck,
motorcyclists, or pedestrians. The multiplicity of roles makes for
complicated modelling decisions when trying to understand the severity
of injuries; for example, what is the unit of analysis, the person, the
traffic unit, or the collision? Not surprisingly, it is possible to find
examples of studies that adopt different perspectives. A common
simplifying strategy in model specification is to consider only
\emph{drivers} and/or only \emph{single-vehicle} crashes (e.g., Kim et
al., 2013; Gong and Fan, 2017; Lee and Li, 2014; Osman et al., 2018).
This strategy reduces the dimensions of the event, and it becomes
possible, for example, to equate the traffic unit to the person for
modelling purposes.

The situation becomes more complex when dealing with events that involve
two traffic units (e.g., Tarrao et al., 2014; Wang et al., 2015) and
multi-traffic unit crashes (e.g., Wu et al., 2014; Bogue et al., 2017).
Different strategies have been developed to study these, more complex
cases. A number of studies advocate the estimation of separate models
for different participants and/or situations. In this way, Wang and
Kockelman (2005) estimated models for single-vehicle and two-vehicle
crashes, while Savolainen and Mannering (2007) estimated models for
single-vehicle and multi-vehicle crashes. More recently, Duddu et
al.~(2018) and Penmetsa et al.~(2017) presented research that estimated
separate models for at-fault and not-at-fault drivers. The strategy of
estimating separate models also relies on subsetting the dataset,
although it is possible to link the relevant models more tightly by
means of a common covariance structure, as is the case of bivariate
models (e.g., Chiou et al., 2013; Chen et al., 2019) or models with
copulas (e.g., Rana et al., 2010; Shamsunnahar et al., 2014; Wang et
al., 2015).

A related strategy to specify a crash severity model is to organize the
data in such a way that it is possible to model the influence of the
attributes of the opponent in a crash. There are numerous examples of
studies that consider at least some characteristics of the oponents in
two- or multi-vehicle crashes. For example, Wang and Kockelman (2005)
considered the type of the opposing vehicle in their model for
two-vehicle collisions. Similarly, Tarrao et al.~(2014) included in
their analysis the age, wheelbase, weight, and engine size of the
opposing vehicle, while Bogue et al.~(2017) used the body type of the
opposing vehicle. Penmetsa et al.~(2017) and Duddu et al.~(2018) are two
of the most comprehensive examples of using opponent's information, as
they used individual attributes of opponents (their physical condition,
sex, and age), as well as characteristics of the opposing traffic unit
(the vehicle type of the opponent). The twin strategies of subsetting
the sample and using the attributes of the opponent are not mutually
exclusive, but neither are they used consistently together, as a scan of
the literature reveals.

\begin{table}

\caption{\label{tab:table-variable-categories}\label{tab:variable-categories}Categories of variables used in the analysis of crash severity with examples}
\centering
\fontsize{7}{9}\selectfont
\begin{tabular}[t]{l>{\raggedright\arraybackslash}p{22em}}
\toprule
Category & Examples\\
\midrule
Person-related & Attributes of participants in the crash, e.g., injury status, age, gender, licensing status, professional driver status\\
Traffic unit-related & Attributes of the traffic unit, e.g., type of traffic unit (car, motorcycle, etc.), maneouver, etc.\\
Crash-related & Attributes of the crash, e.g., location, weather conditions, light conditions, number of parties, etc.\\
Road-related & Attributes of the road, e.g., surface condition, grade, geometry, etc.\\
Opponent-related & Attributes of the opponent, e.g., age of opponent, gender of opponent, opponent vehicle type, etc.\\
\bottomrule
\end{tabular}
\end{table}

\hypertarget{modelling-techniques}{%
\subsection{Modelling techniques}\label{modelling-techniques}}

With respect to model structures, Shamsunnahar and Eluru (2013)
conducted an extensive comparison of models for discrete outcomes and
found only small differences in the performance of unordered models and
ordered models; however, ordered models are usually more parsimonious
since only one latent functions needs to be estimated for all outcomes,
as opposed to one for each outcome in unordered modelling mechanisms.
Bogue et al.~(2017) also compared unordered and ordered models in the
form of the mixed multinomial logit and a modified rank ordered logit,
respectively, and found that the ordered model performed best. To keep
the empirical assessment managable we will consider only the ordinal
logit model, and will comment on potential extensions in Section
\ref{sec:further-considerations}.

The ordinal model is a latent-variable approach, whereby the severity of
the crash (observed) is linked to an underlying latent variable that is
a function of the variables of interest, as follows:

\begin{equation}
\label{eq:latent-function}
y_{itk}^*=\sum_{l=1}^L\alpha_lp_{itkl} + \sum_{m=1}^M\beta_mu_{tkm} + \sum_{q=1}^Q\kappa_qc_{kq} + \epsilon_{itk}
\end{equation}

The left-hand side of the expression above (\(y_{itk}^*\)) is a latent
(unobservable) variable that is associated with the severity of crash
\(k\) (\(k=1,\cdots,K\)) for participant \(i\) in traffic unit \(t\).
The right-hand side of the expression is split in four parts. The first
part collects \(l=1,\cdots,L\) individual attributes \(p\) for
participant \(i\) in traffic unit \(t\) and crash \(k\); these could
relate to the person (e.g., age, gender, and road user class). The
second part collects \(m=1,\cdots,M\) attributes \(u\) related to
traffic unit \(t\) in crash \(k\); these could be items such as maneuver
or vehicle type. The third part collects \(q=1,\cdots,Q\) attributes
\(c\) related to the crash \(k\), including crash-related and
road-related data, such as weather conditions, road alignment, and type
of surface. Lastly, the fourth element is a random term specific to
participant \(i\) in traffic unit \(t\) and crash \(k\). The function
consists of a total of \(Z=L+M+Q\) covariates and associated parameters.

When opponent-related variables are included, the function is augmented
as follows:

\begin{equation}
\label{eq:latent-function-with-opponent-variables}
y_{itk}^*=\sum_{l=1}^L\alpha_lp_{itkl} + \sum_{m=1}^M\beta_mu_{tkm} + \sum_{q=1}^Q\kappa_qc_{kq} + \sum_{r=1}^R\delta_ro_{jvkr} + \epsilon_{itk}
\end{equation}

The additional part collects \(r=1,\cdots,R\) attributes \(o\) related
to individual \(j\) in traffic unit \(v\) and crash \(k\) that opposed
individual \(i\) in traffic unit \(t\) and crash \(k\). These could be
individual characteristics of the opponent (such as age and gender)
and/or characteristics of the opposing vehicle (such as vehicle type or
weight). To qualify as an opponent, individual \(j\) must have been a
participant in crash \(k\) but operating traffic unit \(v\ne t\).
Sometimes the person \emph{is} the traffic unit, as is the case of a
pedestrian. And we exclude passengers of vehicles as opponents, since
they do not operate the traffic unit. In case the opponent attributes
include only characteristics of the traffic unit, the condition for the
traffic unit to be an opponent is that it participated in crash \(k\)
and was different from \(t\). After introducing this new set of terms,
the latent function now consists of a total of \(Z=L+M+Q+R\) covariates
and associated parameters.

For conciseness, in what follows we will abbreviate the function as
follows:

\begin{equation}
\label{eq:latent-function-compact}
y_{itk}^*=\sum_{z=1}^Z\theta_zx_{itkz} + \epsilon_{itk}
\end{equation}

The latent variable is not observed directly, but it is possible to
posit a probabilistic relationship with the outcome \(y_{itk}\) (the
severity of crash \(k\) for participant \(i\) in traffic unit \(t\)).
Depending on the characteristics of the data and the assumptions made
about the random component of the latent function different models can
be obtained. For example, if crash severity is coded as a binary
variable (e.g., non-fatal/fatal), we can relate the latent variable to
the outcome as follows:

\begin{equation}
\label{eq:latent-function-2-outcomes}
y_{itk} = 
\begin{cases}
\text{fatal} & \text{if } y_{itk}^*>0\\
\text{non-fatal} & \text{if } y_{itk}^*\leq0
\end{cases}
\end{equation}

Due to the stochastic nature of the latent function, the outcome of the
crash is not fully determined. However, we can make the following
probability statement:

\begin{equation}
\label{eq:probability-2-outcomes}
P(y_{itk} = \text{fatal}) = P(y_{itk}^* > 0)
\end{equation}

In other words, the probability that individual \(i\) in traffic unit
\(t\) and crash \(k\) was a fatality equals the probability that the
latent variable is greater than zero. This implies (see Maddala, 1986,
p. 22):

\begin{equation}
\label{eq:probability-2-outcomes-2}
\begin{array}{rl}\
P(y_{itk} = \text{fatal}) &= P(\sum_{z=1}\theta_zp_{itkz} + \epsilon_{itk} > 0)\\ 
&=P(\epsilon_{itk} > -\sum_{z=1}\theta_zp_{itkz})
\end{array}
\end{equation}

If the random terms \(\epsilon_{itk}\) are assumed to follow the
logistic distribution, then the binary logit model is obtained; if they
are assumed to follow the normal distribution, the binary probit model
is obtained. More often, though, the outcome is recorded using more
categories, for example property damage only (PDO)/injury/fatality. A
similar approach can be adopted, with a latent variable that relates to
the outcome as follows:

\begin{equation}
\label{eq:latent-function-ordered-outcomes}
y_{itk} = 
\begin{cases}
\text{fatality} & \text{if } y_{itk}^*> \mu_2\\
\text{injury} & \text{if } \mu_1< y_{itk}^*< \mu_2\\
\text{PDO} & \text{if } y_{itk}^*< \mu_1
\end{cases}
\end{equation}

\noindent where \(\mu_1\) and \(\mu_2\) are estimable thresholds. In
this case, the associated probability statements are as follows:

\begin{equation}
\label{eq:probability-ordered-outcomes}
\begin{array}{rcl}\
P(y_{itk} = \text{PDO}) &=& 1 - P(y_{itk} = \text{injury}) - P(y_{itk} = \text{fatality})\\ 
P(y_{itk} = \text{injury}) &=& P(\mu_1 - \sum_{z=1}\theta_zp_{itkz} < \epsilon_{itk} < \mu_2 - \sum_{z=1}\theta_zp_{itkz})\\
P(y_{itk} = \text{fatality}) &=& P(\epsilon_{itk} < \mu_1 - \sum_{z=1}\theta_zp_{itkz})
\end{array}
\end{equation}

If the random terms are assumed to follow the logistic distribution, the
ordered logit model is obtained; if the normal distribution, then the
ordered probit model. Estimation methods for these models are very
well-established (e.g., Maddala, 1986; Train, 2009). There are numerous
variations of the basic modelling framework above, including
hierarchical models, bivariate models, multinomial models, and Bayesian
models, among others (see Savolainen et al., 2011 for a review of
methods).

\hypertarget{model-specification-strategies}{%
\subsection{Model specification
strategies}\label{model-specification-strategies}}

In this paper we consider three general model specification strategies,
as follows:

\begin{itemize}
\tightlist
\item
  Strategy 1. Introducing opponent-related variables
\item
  Strategy 2. Single-level model and multi-level (hierarchical) model
  specifications
\item
  Strategy 3. Full sample and sample subsetting
\end{itemize}

Introduction of opponent related-variables was explained in the
preceding subsection. In this way, a base model that ignores opponent
effects is given by Equation \ref{eq:latent-function}. Strategy 1, in
contrast, is Equation \ref{eq:latent-function-with-opponent-variables},
which includes opponent-related variables. These two equations are also
examples of single-level models. Next we describe Strategies 2 and 3.

\hypertarget{strategy-2-hierarchical-model-specification}{%
\subsubsection{Strategy 2: hierarchical model
specification}\label{strategy-2-hierarchical-model-specification}}

We can choose to conceptualize the event leading to the outcome as a
hierarchical process. There are a few different ways of doing this. For
example, the hierarchy could be based on individuals in traffic units.
In this case, we can rewrite the latent function as follows:

\begin{equation}
\label{eq:latent-function-with-hierarchical-traffic-unit}
y_{itk}^*=\sum_{m=1}^M\beta_mu_{tkm} + \sum_{q=1}^Q\kappa_qc_{kq} + \sum_{r=1}^R\delta_ro_{jvkr} + \epsilon_{itk}
\end{equation}

The coefficients of the traffic unit nest the individual attributes as
follows. For any given coefficient \(q\):

\begin{equation}
\label{eq:hierarchical-traffic-unit-coefficients}
\beta_{m}=\sum_{l=1}^L\beta_{ml}p_{itkl} 
\end{equation}

Therefore, the corresponding term in the latent function becomes
(assuming that \(p_{itk1} = 1\), i.e., it is a constant term):

\begin{equation}
\label{eq:hierarchical-traffic-unit-coefficients}
\begin{array}{rcl}\
\beta_{m}u_{tkm} &=& \big( \beta_{m1} + \beta_{m2}p_{itk2} + \cdots + \beta_{mL}p_{itkL}\big)u_{tkm}\\ 
&=& \beta_{m1}u_{tkm} + \beta_{m2}p_{itk2}u_{tkm} + \cdots + \beta_{mL}p_{itkL}u_{tkm}
\end{array}
\end{equation}

As an alternative, the nesting unit could be the interaction
person-opponent, in which case the opponent-level attributes are nested
in the following fashion:

\begin{equation}
\label{eq:latent-function-with-opponent-variables}
y_{itk}^*=\sum_{l=1}^L\alpha_lp_{itkl} + \sum_{m=1}^M\beta_mu_{tkm} + \sum_{q=1}^Q\kappa_qc_{kq} + \epsilon_{itk}
\end{equation}

\noindent with any person-level coefficient \(l\) that we wish to expand
defined as follows:

\begin{equation}
\label{eq:hierarchical-traffic-unit-coefficients}
\alpha_{l}=\sum_{r=1}^R\alpha_{lr}o_{jvkr}
\end{equation}

\noindent with the same conditions as before, that \(j\ne i\) is the
operator of traffic unit \(v\ne t\). The corresponding term in the
latent function is now (assuming that \(o_{jvk1}=1\), i.e., it is a
constant term):

\begin{equation}
\label{eq:hierarchical-traffic-unit-coefficients}
\begin{array}{rcl}\
\alpha_{l}p_{itkl} &=& \big(\alpha_{l1} + \alpha_{l2}o_{jvk2} + \cdots + \alpha_{lR}o_{jvkR} \big)p_{itkl}\\
&=& \alpha_{l1}p_{itkl} + \alpha_{l2}o_{jvk2}p_{itkl} + \cdots + \alpha_{lR}o_{jvkR}p_{itkl}
\end{array} 
\end{equation}

Alterted readers will identify this model specification strategy as
Casetti's expansion method (Casetti, 1972; Roorda et al., 2010). This is
a deterministic strategy for modelling contextual effects which, when
augmented with random components, becomes the well-known multi-level
modelling method (Hedeker and Gibbons, 1994, more on this in Section
\ref{sec:further-considerations}). It is worthwhile to note that
higher-order hierarchical effects are possible; for instance, individual
attributes nested within traffic units, which in turn are nested within
collisions. We do not explore higher-level hierarchies further in the
current paper.

\hypertarget{strategy-3-sample-subsetting}{%
\subsubsection{Strategy 3: sample
subsetting}\label{strategy-3-sample-subsetting}}

The third model specification strategy that we will consider is
subsetting the sample. This is applicable in conjunction with any of the
other strategies discussed above. In essence, we define the latent
function with restrictions as follows. Consider a continuous variable,
e.g., age of person, and imagine that we wish to concentrate the
analysis on older adults (e.g., Dissanayake and Lu, 2002). The latent
function is defined as desired (see above), however, the following
restriction is applied to the sample:

\begin{equation}
\label{eq:sampling-age}
\text{Age of individual } i \text{ in traffic unit } t \text{ in crash } k = 
\begin{cases}
\ge 65 & \text{use record } itk\\
< 65 & \text{do not use record } itk
\end{cases}
\end{equation}

Suppose instead that we are interested in crashes by or against a
specific type of traffic unit (e.g., pedestrians, Amoh-Gyimah et al.,
2017):

\begin{equation}
\label{eq:sampling-pedestrian}
\text{Road user class of individual } i \text{ in traffic unit } t \text{ in crash } k = 
\begin{cases}
\text{Pedestrian} & \text{use record } itk\\
\text{Not pedestrian} & \text{do not use record } itk
\end{cases}
\end{equation}

\noindent or:

\begin{equation}
\label{eq:sampling-pedestrian-opponent}
\text{Road user class of individual } j \text{ in traffic unit } v \text{ in crash } k = 
\begin{cases}
\text{Pedestrian} & \text{use record } jvk\\
\text{Not pedestrian} & \text{do not use record } jvk
\end{cases}
\end{equation}

More generally, for any variable \(x\) of interest:

\begin{equation}
\label{eq:sampling-general}
x_{itk} = 
\begin{cases}
\text{Condition: TRUE} & \text{use record } itk\\
\text{Condition: FALSE} & \text{do not use record } itk
\end{cases}
\end{equation}

Several conditions can be imposed to subset the sample in any way that
the analyst deems appropriate or suitable.

\hypertarget{sec:application}{%
\section{Setting for empirical assessment}\label{sec:application}}

In this section we present the setting for the empirical assessment of
the modelling strategies discussed in Section
\ref{sec:review-of-methods}, namely matters related to data and model
estimation.

\begin{quote}
\textbf{Note:} this paper presents reproducible research. The source
file is an R Markdown document. All code and data necessary to reproduce
the analysis are available from the following anonymous Drive folder:
\end{quote}

\begin{quote}
https://drive.google.com/open?id=12aJtVBaQ4Zj0xa7mtfqxh0E48hKCb\_XV
\end{quote}

\begin{quote}
The source files, code, and data will be publicly available in a GitHub
repository upon acceptance of the paper for publication
\end{quote}

\hypertarget{data-for-empirical-assessment}{%
\subsection{Data for empirical
assessment}\label{data-for-empirical-assessment}}

To assess the performance of the various modelling strategies we use
data from Canada's National Collision Database (NCDB). This database
contains all police-reported motor vehicle collisions on public roads in
Canada. Data are originally collected by provinces and territories, and
shared with the federal government, that proceeds to combine, track, and
analyze them for reporting deaths, injuries, and collisions in Canada at
the national level. The NCDB is provided by Transport Canada, the agency
of the federal government of Canada in charge of transportation policies
and programs, under the Open Government License - Canada version 2.0
{[}https://open.canada.ca/en/open-government-licence-canada{]}.

The NCDB is available from 1999. For the purpose of this paper, we use
the data corresponding to 2017, which is the most recent year available
as of this writing. Furthermore, for assessment we also use the data
corresponding to 2016. Similar to databases in other jurisdictions (see
Montella et al., 2013), the NCDB contains information pertaining to the
collision, the traffic unit(s), and the person(s) involved in a crash,
as shown in Tables \ref{tab:ncdb-descriptives-collision},
\ref{tab:ncdb-descriptives-vehicle}, and
\ref{tab:ncdb-descriptives-person}. Notice that, compared to Table
\ref{tab:variable-categories}, crash-related variables and road-related
variables are collected under a single variable class, namely
collision-related, since they are unique for each crash.

Data are organized by person; in other words, there is one record per
participant in a collision, be they drivers, passengers, pedestrians,
etc. The only variable directly available with respect to opponents in a
collision is the number of vehicles involved (see models in Bogue et
al., 2017). Therefore, the data needs to be processed to obtain
attributes of opponents for each participant in a collision. The
protocol to do this is described next.

\begingroup\fontsize{7}{9}\selectfont

\begin{longtable}[t]{ll>{\raggedright\arraybackslash}p{32em}}
\caption{\label{tab:ncdb-descriptives-collision}\label{tab:ncdb-descriptives-collision}Contents of National Collision Database: Collision-level variables}\\
\toprule
Variable & Description & Notes\\
\midrule
\endfirsthead
\caption[]{\label{tab:ncdb-descriptives-collision}Contents of National Collision Database: Collision-level variables \textit{(continued)}}\\
\toprule
Variable & Description & Notes\\
\midrule
\endhead
\
\endfoot
\bottomrule
\multicolumn{3}{l}{\textit{Note: }}\\
\multicolumn{3}{l}{Source NCDB available from https://open.canada.ca/data/en/dataset/1eb9eba7-71d1-4b30-9fb1-30cbdab7e63a}\\
\multicolumn{3}{l}{Source data files for analysis also available from https://drive.google.com/open?id=12aJtVBaQ4Zj0xa7mtfqxh0E48hKCb\_XV}\\
\endlastfoot
\rowcolor{gray!6}  C\_CASE & Unique collision identifier & Unique identifier for collisions\\
C\_YEAR & Year & Last two digits of year.\\
\rowcolor{gray!6}  C\_MNTH & Month & 14 levels: January - December; unknown; not reported by jurisdiction.\\
C\_WDAY & Day of week & 9 levels: Monday - Sunday; unknown; not reported by jurisdiction.\\
\rowcolor{gray!6}  C\_HOUR & Collision hour & 25 levels: hourly intervals; unknown; not reported by jurisdiction.\\
\addlinespace
C\_SEV & Collision severity & 4 levels: collision producing at least one fatality; collision producing non-fatal injury; unknown; not reported by jurisdiction.\\
\rowcolor{gray!6}  C\_VEHS & Number of vehicles involved in collision & Number of vehicles: 1-98 vehicles involved; 99 or more vehicles involved; unknown; not reported by jurisdiction.\\
C\_CONF & Collision configuration & 21 levels: SINGLE VEHICLE: Hit a moving object (e.g. a person or an animal); Hit a stationary object (e.g. a tree); Ran off left shoulder; Ran off right shoulder; Rollover on roadway; Any other single vehicle collision configuration; TWO-VEHICLES SAME DIRECTION OF TRAVEL: Rear-end collision; Side swipe; One vehicle passing to the left of the other, or left turn conflict; One vehicle passing to the right of the other, or right turn conflict; Any other two vehicle - same direction of travel configuration; TWO-VEHICLES DIFFERENT DIRECTION OF TRAVEL: Head-on collision; Approaching side-swipe; Left turn across opposing traffic; Right turn, including turning conflicts; Right angle collision; Any other two-vehicle - different direction of travel configuration; TWO-VEHICLES, HIT A PARKED VEHICLE: Hit a parked motor vehicle; Choice is other than the preceding values; unknown;not reported by jurisdiction.\\
\rowcolor{gray!6}  C\_RCFG & Roadway configuration & 15 levels: Non-intersection; At an intersection of at least two public roadways; Intersection with parking lot entrance/exit, private driveway or laneway; Railroad level crossing; Bridge, overpass, viaduct; Tunnel or underpass; Passing or climbing lane; Ramp; Traffic circle; Express lane of a freeway system; Collector lane of a freeway system; Transfer lane of a freeway system; Choice is other than the preceding values; unknown;not reported by jurisdiction.\\
C\_WTHR & Weather condition & 10 levels: Clear and sunny; Overcast, cloudy but no precipitation; Raining; Snowing, not including drifting snow; Freezing rain, sleet, hail; Visibility limitation; Strong wind; Choice is other than the preceding values; unknown;not reported by jurisdiction.\\
\addlinespace
\rowcolor{gray!6}  C\_RSUR & Road surface & 12 levels: Dry, normal; Wet; Snow (fresh, loose snow); Slush, wet snow; Icy, packed snow; Debris on road (e.g., sand/gravel/dirt); Muddy; Oil; Flooded; Choice is other than the preceding values; unknown;not reported by jurisdiction.\\
C\_RALN & Road alignment & 9 levels: Straight and level; Straight with gradient; Curved and level; Curved with gradient; Top of hill or gradient; Bottom of hill or gradient; Choice is other than the preceding values; unknown;not reported by jurisdiction.\\
\rowcolor{gray!6}  C\_TRAF & Traffic control & 21 levels: Traffic signals fully operational; Traffic signals in flashing mode; Stop sign; Yield sign; Warning sign; Pedestrian crosswalk; Police officer; School guard, flagman; School crossing; Reduced speed zone; No passing zone sign; Markings on the road; School bus stopped with school bus signal lights flashing; School bus stopped with school bus signal lights not flashing; Railway crossing with signals, or signals and gates; Railway crossing with signs only; Control device not specified; No control present; Choice is other than the preceding values; unknown; not reported by jurisdiction.\\*
\end{longtable}
\endgroup{}

\begingroup\fontsize{7}{9}\selectfont

\begin{longtable}[t]{ll>{\raggedright\arraybackslash}p{32em}}
\caption{\label{tab:ncdb-descriptives-vehicle}\label{tab:ncdb-descriptives-vehicle}Contents of National Collision Database: Traffic unit-level variables}\\
\toprule
Variable & Description & Notes\\
\midrule
\endfirsthead
\caption[]{\label{tab:ncdb-descriptives-vehicle}Contents of National Collision Database: Traffic unit-level variables \textit{(continued)}}\\
\toprule
Variable & Description & Notes\\
\midrule
\endhead
\
\endfoot
\bottomrule
\multicolumn{3}{l}{\textit{Note: }}\\
\multicolumn{3}{l}{Source NCDB available from https://open.canada.ca/data/en/dataset/1eb9eba7-71d1-4b30-9fb1-30cbdab7e63a}\\
\multicolumn{3}{l}{Source data files for analysis also available from https://drive.google.com/open?id=12aJtVBaQ4Zj0xa7mtfqxh0E48hKCb\_XV}\\
\endlastfoot
\rowcolor{gray!6}  V\_ID & Vehicle sequence number & Number of vehicles: 1-98; Pedestrian sequence number: 99; unknown.\\
V\_TYPE & Vehicle type & 21 levels: Light Duty Vehicle (Passenger car, Passenger van, Light utility vehicles and light duty pick up trucks); Panel/cargo van (<= 4536 KG GVWR   Panel or window type of van designed primarily for carrying goods); Other trucks and vans (<= 4536 KG GVWR); Unit trucks (> 4536 KG GVWR); Road tractor; School bus; Smaller school bus (< 25 passengers); Urban and Intercity Bus; Motorcycle and moped; Off road vehicles; Bicycle; Purpose-built motorhome; Farm equipment; Construction equipment; Fire engine; Snowmobile; Street car; Data element is not applicable  (e.g. dummy vehicle record created for pedestrian); Choice is other than the preceding values; unknown; not reported by jurisdiction.\\
\rowcolor{gray!6}  V\_YEAR & Vehicle model year & Model year; dummy for pedestrians; unknown; not reported by jurisdiction.\\*
\end{longtable}
\endgroup{}

\begingroup\fontsize{7}{9}\selectfont

\begin{longtable}[t]{ll>{\raggedright\arraybackslash}p{32em}}
\caption{\label{tab:ncdb-descriptives-person}\label{tab:ncdb-descriptives-person}Contents of National Collision Database: Personal-level variables}\\
\toprule
Variable & Description & Notes\\
\midrule
\endfirsthead
\caption[]{\label{tab:ncdb-descriptives-person}Contents of National Collision Database: Personal-level variables \textit{(continued)}}\\
\toprule
Variable & Description & Notes\\
\midrule
\endhead
\
\endfoot
\bottomrule
\multicolumn{3}{l}{\textit{Note: }}\\
\multicolumn{3}{l}{Source NCDB available from https://open.canada.ca/data/en/dataset/1eb9eba7-71d1-4b30-9fb1-30cbdab7e63a}\\
\multicolumn{3}{l}{Preprocessed data for analysis available from https://drive.google.com/open?id=12aJtVBaQ4Zj0xa7mtfqxh0E48hKCb\_XV}\\
\endlastfoot
\rowcolor{gray!6}  P\_ID & Person sequence number & Sequence number: 1-99; Not applicable (dummy for parked vehicles); not reported by jurisdiction.\\
P\_SEX & Person sex & 5 levels: Male; Female; Not applicable (dummy for parked vehicles); unknown (runaway vehicle); not reported by jurisdiction.\\
\rowcolor{gray!6}  P\_AGE & Person age & Age: less than 1 year; 1-98 years old; 99 years or older; Not applicable (dummy for parked vehicles); unknown (runaway vehicle); not reported by jurisdiction.\\
P\_PSN & Person position & Person position: Driver; Passenger front row, center; Passenger front row, right outboard (including motorcycle passenger in sidecar); Passenger second row, left outboard, including motorcycle passenger; Passenger second row, center; Passenger second row, right outboard; Passenger third row, left outboard;...; Position unknown, but the person was definitely an occupant; Sitting on someone’s lap; Outside passenger compartment; Pedestrian; Not applicable (dummy for parked vehicles); Choice is other than the preceding values; unknown (runaway vehicle); not reported by jurisdiction.\\
\rowcolor{gray!6}  P\_ISEV & Medical treatment required & 6 levels: No Injury; Injury; Fatality; Not applicable (dummy for parked vehicles); Choice is other than the preceding values; unknown (runaway vehicle); not reported by jurisdiction.\\
\addlinespace
P\_SAFE & Safety device used & 11 levels: No safety device used; Safety device used; Helmet worn; Reflective clothing worn; Both helmet and reflective clothing used; Other safety device used; No safety device equipped   (e.g. buses); Not applicable (dummy for parked vehicles); Choice is other than the preceding values; unknown (runaway vehicle); not reported by jurisdiction.\\
\rowcolor{gray!6}  P\_USER & Road user class & 6 levels: Motor Vehicle Driver; Motor Vehicle Passenger; Pedestrian; Bicyclist; Motorcyclist; Not stated/Other/Unknown.\\*
\end{longtable}
\endgroup{}

\hypertarget{data-preprocessing-initial-filter}{%
\subsubsection{Data preprocessing: initial
filter}\label{data-preprocessing-initial-filter}}

We apply an initial filter, whereby we scan the database to remove all
records that are not a person (including parked cars and other objects)
or that are missing information (as is the case when a participant in a
crash is a runaway vehicle). Next, records missing at least one of the
next variables are removed: P\_USER (the road user class), P\_SEX (sex),
P\_AGE (age), and P\_ISEV (individual-level severity of crash). This
initial filter ensures that all records are complete from the
perspective of key information for analysis.

\hypertarget{data-preprocessing-filter-two-vehicle-collisions}{%
\subsubsection{Data preprocessing: filter two-vehicle
collisions}\label{data-preprocessing-filter-two-vehicle-collisions}}

After the initial filter, the database is summarized to find the number
of individual-level records that correspond to each collision (C\_CASE).
At this point, there are 32,298 collisions, involving only one (known)
participant, there are 46,483 collisions involving two participants,
19,433 collisions with three participants, 8,250 collisions involving
four participants, 3,783 collisions with five participants, 1,789
collisions with six participants, and 1,491 collisions involving seven
or more participants. These participants were possibly occupants in
different vehicles or were otherwise their own traffic units.
Accordingly, the sample includes 174,741 drivers, 61,403 passengers,
10,798 pedestrians, 5,286 bicyclists, and 6,564 motorcyclists.

The next step in our data preprocessing protocol is to remove all
collisions that involve only one participant. This still leaves numerous
cases where multiple participants could have been in a single vehicle,
for instance in a collision against a stationary object. Therefore, we
proceed to use the vehicle sequence number to find the number of
vehicles involved in each collision. This reveals that there are 20,732
collisions that involve only one vehicle but possibly multiple
participants (i.e., driver and one or more passengers). In addition,
there are 165,520 collisions involving two vehicles (and possibly
multiple participants). Finally, there are 40,242 collisions with three
or more vehicles.

Once we have identified the number of vehicles in each collision, we
proceed to select all cases that involve only two vehicles. The most
common cases in two-vehicle collisions are those that include drivers
(40,297 collisions; this is reflective of the prevalence of
single-occupant vehicles). This is followed by cases with passengers
(14,120 collisions), pedestrians (5,204 collisions), byciclists (2,238
collisions), and motorcyclists (1,016 collisions). The distribution of
individuals per traffic unit is as follows: 80,382 individuals are coded
as being in V\_ID = 1, 76,523 individuals are coded as being in V\_ID =
2, and 7,932 individuals are coded as pedestrians. In addition, 683
individuals are coded as being in vehicles 3 through 9, despite our
earlier filter to retain only collisions with two vehicles. We therefore
proceed to select only individuals assigned to vehicles 1 or 2, as well
as pedestrians. As a result of this filter a number of cases with only
one known participant need to be removed.

\hypertarget{data-preprocessing-extract-opponent-information-and-join-to-individual-records}{%
\subsubsection{Data preprocessing: extract opponent information and join
to individual
records}\label{data-preprocessing-extract-opponent-information-and-join-to-individual-records}}

Up to this point, the goal of data preprocessing has been to obtain a
complete, workable sample of individual records of participants in
two-vehicle collisions. There are two possible cases for the collisions,
depending on the traffic units involved: 1) vehicle vs vehicle
collisions (``vehicle'' is all motorized vehicles, including
motorcycles/mopeds, as well as bicycles); and 2) vehicle vs pedestrian
collisions. To identify opponents in a collision, it is convenient to
classify collisions by pedestrian involvement. In this way, we find that
the database includes 16,636 collisions that are vehicle vs pedestrian
(possibly multiple pedestrians), and 147,594 collisions that involve two
vehicles. After splitting the database according to pedestrian
involvement, we can now extract relevant information about participants
in the collision. This involves renaming the person-level variables so
that we can distinguish each individual by their role as an individual
or an opponent in a given record. Notice that when working with
individuals in vehicles, only the driver is considered a legitimate
opponent in a collision (i.e., passengers are never considered
opponents).

Once the personal attributes of individuals that can be considered
opponents in a given collision have been extracted, their information is
joined to the individual records by means of the collision unique
identifier. As a result of this process, a new set of variables are now
available for analysis: the age, sex, and road user class of the
opponent, as well as the type of the opposing vehicle. A summary of
opponent interactions and outcomes can be found in Table
\ref{tab:summary-opponent-information}. The information there shows that
the most commont type of opponent for drivers is other drivers, followed
by pedestrians. The only opponents of pedestrians, on the other hand,
are drivers. Bicyclists and motorcyclists, on the other hand, are mostly
opposed by drivers, but occasionally by other road users as well. In
terms of outcomes, we observe that virtually all fatalities occur when
the opponent is a driver, and only very rarely when the opponent is a
motorcyclist. Injuries are also more common when the opponent is a
driver, whereas no injuries are relatively more frequent when the
opponent is a pedestrian or a bicyclist.

\begin{table}

\caption{\label{tab:summary-opponent-information}\label{tab:summary-opponent-information}Summary of opponent interactions and outcomes by road user class}
\centering
\resizebox{\linewidth}{!}{
\fontsize{7}{9}\selectfont
\begin{tabular}[t]{lrrrrrrrlll}
\toprule
\multicolumn{1}{c}{} & \multicolumn{4}{c}{Road User Class of Opponent} & \multicolumn{3}{c}{Outcome} & \multicolumn{3}{c}{Proportion by Road User Class} \\
\cmidrule(l{3pt}r{3pt}){2-5} \cmidrule(l{3pt}r{3pt}){6-8} \cmidrule(l{3pt}r{3pt}){9-11}
Road User Class & Driver & Pedestrian & Bicyclist & Motorcyclist & No Injury & Injury & Fatality & No Injury & Injury & Fatality\\
\midrule
\rowcolor{gray!6}  \addlinespace[0.3em]
\multicolumn{11}{l}{\textbf{All opponents}}\\
\hspace{1em}Driver & 97582 & 7880 & 3799 & 2498 & 59180 & 52143 & 436 & 0.52953 & 0.46657 & 0.003901\\
\hspace{1em}Passenger & 35359 & 1282 & 667 & 818 & 19308 & 18667 & 151 & 0.50643 & 0.48961 & 0.003961\\
\rowcolor{gray!6}  \hspace{1em}Pedestrian & 7880 & 0 & 0 & 0 & 145 & 7507 & 228 & 0.01840 & 0.95266 & \vphantom{1} 0.028934\\
\hspace{1em}Bicyclist & 3799 & 1 & 0 & 40 & 49 & 3760 & 31 & 0.01276 & 0.97917 & 0.008073\\
\rowcolor{gray!6}  \hspace{1em}Motorcyclist & 2498 & 30 & 40 & 338 & 204 & 2598 & 104 & 0.07020 & 0.89401 & 0.035788\\
\addlinespace[0.3em]
\multicolumn{11}{l}{\textbf{Opponent: Driver}}\\
\hspace{1em}Driver & 97582 & 0 & 0 & 0 & 45493 & 51657 & 432 & 0.46620 & 0.52937 & 0.004427\\
\rowcolor{gray!6}  \hspace{1em}Passenger & 35359 & 0 & 0 & 0 & 16672 & 18536 & 151 & 0.47151 & 0.52422 & 0.004270\\
Pedestrian & 7880 & 0 & 0 & 0 & 145 & 7507 & 228 & 0.01840 & 0.95266 & \vphantom{1} 0.028934\\
\rowcolor{gray!6}  \hspace{1em}Bicyclist & 3799 & 0 & 0 & 0 & 43 & 3725 & 31 & 0.01132 & 0.98052 & 0.008160\\
\hspace{1em}Motorcyclist & 2498 & 0 & 0 & 0 & 98 & 2299 & 101 & 0.03923 & 0.92034 & 0.040432\\
\rowcolor{gray!6}  \addlinespace[0.3em]
\multicolumn{11}{l}{\textbf{Opponent: Pedestrian}}\\
\hspace{1em}Driver & 0 & 7880 & 0 & 0 & 7693 & 187 & 0 & 0.97627 & 0.02373 & 0.000000\\
\hspace{1em}Passenger & 0 & 1282 & 0 & 0 & 1246 & 36 & 0 & 0.97192 & 0.02808 & 0.000000\\
\rowcolor{gray!6}  \hspace{1em}Pedestrian & 0 & 0 & 0 & 0 & 0 & 0 & 0 & - & - & \vphantom{2} -\\
\hspace{1em}Bicyclist & 0 & 1 & 0 & 0 & 0 & 1 & 0 & 0.00000 & 1.00000 & 0.000000\\
\rowcolor{gray!6}  \hspace{1em}Motorcyclist & 0 & 30 & 0 & 0 & 11 & 19 & 0 & 0.36667 & 0.63333 & 0.000000\\
\addlinespace[0.3em]
\multicolumn{11}{l}{\textbf{Opponent: Bicyclist}}\\
\hspace{1em}Driver & 0 & 0 & 3799 & 0 & 3706 & 93 & 0 & 0.97552 & 0.02448 & 0.000000\\
\rowcolor{gray!6}  \hspace{1em}Passenger & 0 & 0 & 667 & 0 & 649 & 18 & 0 & 0.97301 & 0.02699 & 0.000000\\
\hspace{1em}Pedestrian & 0 & 0 & 0 & 0 & 0 & 0 & 0 & - & - & \vphantom{1} -\\
\rowcolor{gray!6}  \hspace{1em}Bicyclist & 0 & 0 & 0 & 0 & 0 & 0 & 0 & - & - & -\\
\hspace{1em}Motorcyclist & 0 & 0 & 40 & 0 & 16 & 24 & 0 & 0.40000 & 0.60000 & 0.000000\\
\rowcolor{gray!6}  \addlinespace[0.3em]
\multicolumn{11}{l}{\textbf{Opponent: Motorcyclist}}\\
\hspace{1em}Driver & 0 & 0 & 0 & 2498 & 2288 & 206 & 4 & 0.91593 & 0.08247 & 0.001601\\
\hspace{1em}Passenger & 0 & 0 & 0 & 818 & 741 & 77 & 0 & 0.90587 & 0.09413 & 0.000000\\
\rowcolor{gray!6}  Pedestrian & 0 & 0 & 0 & 0 & 0 & 0 & 0 & - & - & \vphantom{2} -\\
\hspace{1em}Bicyclist & 0 & 0 & 0 & 40 & 6 & 34 & 0 & 0.15000 & 0.85000 & 0.000000\\
\rowcolor{gray!6}  \hspace{1em}Motorcyclist & 0 & 0 & 0 & 338 & 79 & 256 & 3 & 0.23373 & 0.75740 & 0.008876\\
\bottomrule
\end{tabular}}
\end{table}

\hypertarget{model-estimation}{%
\subsection{Model estimation}\label{model-estimation}}

Before model estimation, the variables are prepared as follows. First,
age is scaled from years to decades. Secondly, new variables are defined
to describe the vehicle type. Three classes of vehicle types are
considered: 1) light duty vehicles (which in Canada include passenger
cars, passenger vans, light utility vehicles, and light duty pick up
trucks); 2) light trucks (all other vehicles \(\le\) 4536 kg in gross
vehicle weight rating); and heavy vehicles (all other vehicles \(\ge\)
4536 in gross vehicle weight rating). Furthermore, this typology of
vehicle is combined with the road user class of the individual to
distiguish between drivers and passengers of light duty vehicles, light
trucks, and heavy vehicles, in addition to pedestrians, bicyclists, and
motorcyclists. This is done for both the individual and the opponent.
Variable interactions are calculated to produce hierarchical variables.
For example, for a hierarchical definition of traffic unit-level
variables, age (and the square of age to account for possible
non-monotonic effects) are interacted with gender, road user class, and
vehicle type. For hierarchical opponent variables, age (and the square
of age) are interacted with the age of opponent (and the corresponding
square). The variables thus obtained are shown in Table
\ref{tab:model-specification-summary}. As seen in the table, Models 1
and 2 are single-level models, and the difference between them is that
Model 2 includes opponent variables. Models 3 and 4, in contrast, are
hierarchical models. Model 3 considers the hierarchy on the basis of the
traffic unit, while Model 4 considers the hierarchy on the basis of the
opponent.

Models 1 through 4 are estimated using the full sample. As discussed
above, a related modelling strategy is to subset the sample (e.g., Islam
et al., 2014; Lee and Li, 2014; Tarrao et al., 2014; Wu et al., 2014).
In this case we subset by a combination of traffic unit type of the
individual (i.e., light duty vehicle, light truck, heavy vehicle,
pedestrian, bicyclist, and motorcyclist) and vehicle type of the oponent
(i.e., light duty vehicle, light truck, heavy vehicle). This leads to an
ensemble of eighteen models to be estimated using subsets of data (see
Table \ref{tab:model-specification-summary}). By subsetting the sample,
at least \emph{some} opponent effects are incorporated implicitly.
Models 1 and 2 are re-estimated using this strategy, dropping variables
as necessary whenever they become irrelevant (for instance, after
filtering for pedestrians, no other traffic unit types are present in
the subset of data). In addition to variables that are no longer
relevant in some subsamples, it is important to note that when using
some subsamples a few variables had to be occasionally dropped to avoid
convergence issues. This tended to happen particularly with smaller
subsamples where some particular combination of attributes was rare as a
result of subsampling (e.g., in 2017 there were few or no collisions
that involved a motorcyclist and a heavy vehicle in a bridge, or
overpass, or viaduct). The process of estimation carefully paid
attention to convergence issues to ensure the validity of the models
reported here.

The resuls of model estimation are discussed in the following section.

\begin{table}

\caption{\label{tab:model-specification-summary}\label{tab:model-specification-summary}Summary of variables and model specification}
\centering
\resizebox{\linewidth}{!}{
\fontsize{7}{9}\selectfont
\begin{tabular}[t]{llllll}
\toprule
Variable & Notes & Model 1 & Model 2 & Model 3 & Model 4\\
\midrule
\rowcolor{gray!6}  \addlinespace[0.3em]
\multicolumn{6}{l}{\textbf{Individual-level variables}}\\
\hspace{1em}Age & In decades & $\checkmark$ & $\checkmark$ & $\checkmark$ & $\checkmark$\\
\hspace{1em}Age Squared &  & $\checkmark$ & $\checkmark$ & $\checkmark$ & $\checkmark$\\
\rowcolor{gray!6}  \hspace{1em}Sex & Reference: Female & $\checkmark$ & $\checkmark$ & $\checkmark$ & $\checkmark$\\
\hspace{1em}Use of Safety Devices & 7 levels; Reference: No Safety Device & $\checkmark$ & $\checkmark$ & $\checkmark$ & $\checkmark$\\
\rowcolor{gray!6}  \addlinespace[0.3em]
\multicolumn{6}{l}{\textbf{Traffic unit-level variables}}\\
\hspace{1em}Passenger & Reference: Driver & $\checkmark$ & $\checkmark$ &  & $\checkmark$\\
\hspace{1em}Pedestrian & Reference: Driver & $\checkmark$ & $\checkmark$ &  & $\checkmark$\\
\rowcolor{gray!6}  \hspace{1em}Bicyclist & Reference: Driver & $\checkmark$ & $\checkmark$ &  & $\checkmark$\\
\hspace{1em}Motorcyclist & Reference: Driver & $\checkmark$ & $\checkmark$ &  & $\checkmark$\\
\rowcolor{gray!6}  \hspace{1em}Light Truck & Reference: Light Duty Vehicle & $\checkmark$ & $\checkmark$ &  & $\checkmark$\\
\hspace{1em}Heavy Vehicle & Reference: Light Duty Vehicle & $\checkmark$ & $\checkmark$ &  & $\checkmark$\\
\rowcolor{gray!6}  \addlinespace[0.3em]
\multicolumn{6}{l}{\textbf{Opponent variables}}\\
\hspace{1em}Age of Opponent & In decades &  & $\checkmark$ & $\checkmark$ & \\
\hspace{1em}Age of Opponent Squared &  &  & $\checkmark$ & $\checkmark$ & \\
\rowcolor{gray!6}  \hspace{1em}Sex of Opponent & Reference: Female &  & $\checkmark$ & $\checkmark$ & \\
\hspace{1em}Opponent: Light Duty Vehicle & Reference: Pedestrian/Bicyclist/Motorcyclist &  & $\checkmark$ & $\checkmark$ & $\checkmark$\\
\rowcolor{gray!6}  \hspace{1em}Opponent: Light Truck & Reference: Pedestrian/Bicyclist/Motorcyclist &  & $\checkmark$ & $\checkmark$ & $\checkmark$\\
\hspace{1em}Opponent: Heavy Vehicle & Reference: Pedestrian/Bicyclist/Motorcyclist &  & $\checkmark$ & $\checkmark$ & $\checkmark$\\
\rowcolor{gray!6}  \addlinespace[0.3em]
\multicolumn{6}{l}{\textbf{Hierarchical traffic unit variables}}\\
\hspace{1em}Light Truck Driver:Age &  &  &  & $\checkmark$ & \\
\hspace{1em}Light Truck Driver:Age Squared &  &  &  & $\checkmark$ & \\
\rowcolor{gray!6}  \hspace{1em}Heavy Vehicle Driver:Age &  &  &  & $\checkmark$ & \\
\hspace{1em}Heavy Vehicle Driver:Age Squared &  &  &  & $\checkmark$ & \\
\rowcolor{gray!6}  \hspace{1em}Light Truck Passenger:Age &  &  &  & $\checkmark$ & \\
\hspace{1em}Light Truck Passenger:Age Squared: &  &  &  & $\checkmark$ & \\
\rowcolor{gray!6}  \hspace{1em}Heavy Vehicle Passenger:Age &  &  &  & $\checkmark$ & \\
\hspace{1em}Heavy Vehicle Passenger:Age Squared &  &  &  & $\checkmark$ & \\
\rowcolor{gray!6}  \hspace{1em}Pedestrian:Age &  &  &  & $\checkmark$ & \\
\hspace{1em}Pedestrian:Age Squared &  &  &  & $\checkmark$ & \\
\rowcolor{gray!6}  \hspace{1em}Bicyclist:Age &  &  &  & $\checkmark$ & \\
\hspace{1em}Bicyclist:Age Squared &  &  &  & $\checkmark$ & \\
\rowcolor{gray!6}  \hspace{1em}Motorcyclist:Age &  &  &  & $\checkmark$ & \\
\hspace{1em}Motorcyclist:Age Squared &  &  &  & $\checkmark$ & \\
\rowcolor{gray!6}  \addlinespace[0.3em]
\multicolumn{6}{l}{\textbf{Hierarchical opponent variables}}\\
\hspace{1em}Age:Age of Opponent &  &  &  &  & $\checkmark$\\
\hspace{1em}Age:Age of Female Opponent &  &  &  &  & $\checkmark$\\
\rowcolor{gray!6}  \hspace{1em}Age:Age of Male Opponent Squared &  &  &  &  & $\checkmark$\\
\hspace{1em}Age:Age of Female Opponent Squared &  &  &  &  & $\checkmark$\\
\rowcolor{gray!6}  \hspace{1em}Age Squared:Age of Male Opponent &  &  &  &  & $\checkmark$\\
\hspace{1em}Age Squared:Age of Female Opponent &  &  &  &  & $\checkmark$\\
\rowcolor{gray!6}  \addlinespace[0.3em]
\multicolumn{6}{l}{\textbf{Collision-level variables}}\\
\hspace{1em}Crash Configuration & 19 levels; Reference: Hit a moving object & $\checkmark$ & $\checkmark$ & $\checkmark$ & $\checkmark$\\
\hspace{1em}Road Configuration & 12 levels; Reference: Non-intersection & $\checkmark$ & $\checkmark$ & $\checkmark$ & $\checkmark$\\
\rowcolor{gray!6}  \hspace{1em}Weather & 9 levels; Reference: Clear and sunny & $\checkmark$ & $\checkmark$ & $\checkmark$ & $\checkmark$\\
\hspace{1em}Surface & 11 levels; Reference: Dry & $\checkmark$ & $\checkmark$ & $\checkmark$ & $\checkmark$\\
\rowcolor{gray!6}  \hspace{1em}Road Alignment & 8 levels; Reference: Straight and level & $\checkmark$ & $\checkmark$ & $\checkmark$ & $\checkmark$\\
\hspace{1em}Traffic Controls & 19 levels; Reference: Operational traffic signals & $\checkmark$ & $\checkmark$ & $\checkmark$ & $\checkmark$\\
\rowcolor{gray!6}  \hspace{1em}Month & 12 levels; Reference: January & $\checkmark$ & $\checkmark$ & $\checkmark$ & $\checkmark$\\
\bottomrule
\end{tabular}}
\end{table}

\hypertarget{sec:assessment}{%
\section{Model assessment}\label{sec:assessment}}

In this section we report an in-depth examination of the performance of
the models. To this end, first inspect the goodness of fit of the
models. Next, we use the models to conduct in-sample predictions (i.e.,
\emph{nowcasting}), using the same sample that was used to estimate the
models. In addition, we also conduct out-of-sample predictions, using
the dataset corresponding to the year 2016, processed in identical way
as the estimation sample. This is an example of \emph{backcasting}.
Using these predictions we evaluate the models in two ways: first, we
compute the estimated shares of each outcome based on the predicted
probabilities; and secondly, we evaluate the classes of outcomes of the
individual-level predictions.

\hypertarget{goodness-of-fit-of-models}{%
\subsection{Goodness of fit of models}\label{goodness-of-fit-of-models}}

We begin our empirical assessment by examining the results of estimating
the models described above. Tables \ref{tab:model-full-sample-summary}
and \ref{tab:model-subsample-summary} collect some key summary
statistics of the estimated models. Of interest is the goodness of fit
of the models, which in the case is measured with Akaike's Information
Criterion (\(AIC\)). This criterion is calculated as follows:

\begin{equation}
\label{eq:aic}
AIC = 2Z - 2\ln{\hat{L}}
\end{equation}

\noindent where \(Z\) is the number of coefficients estimated by the
model, and \(\hat{L}\) the maximized likelihood of the model. Since
\(AIC\) penalizes the model fit by means of the number of coefficients,
this criterion gives preference to more parsimonious models. The
objective is to minimize the \(AIC\), and therefore smaller values of
this criterion represent better model fits. Model comparison can be
conducted using the relative likelihood. Suppose that we have two
models, say Model \(a\) and Model \(b\), with \(AIC_{a} \le AIC_{b}\).
The relative likelihood is calculated as:

\begin{equation}
\label{eq:relative-likelihood}
e^{(AIC_b - AIC_b)/2}
\end{equation}

The relative likelihood is interpreted as the probability that Model
\(b\) minimizes the information loss as well as Model \(a\).

Turning our attention to the models estimated using the full sample
(Table \ref{tab:model-full-sample-summary}), it is possible to see that,
compared to the base (single-level) model without opponent variables
(Model 1), there are large and significant improvements in goodness of
fit to be gained by introducing opponent effects. However, the gains are
not as large when hierarchical specifications are used, even when the
number of additional coefficients that need to be estimated is not
substantially larger (recall that the penalty per coefficient in \(AIC\)
is 2). The best model according to this measure of goodness of fit is
Model 2 (single-level with opponent effects), followed by Model 4
(hierarchical opponent variables), Model 3 (hierarchical traffic unit
variables with opponent effects), and finally Model 1 (single-level
without opponent effects).

It is important to note that the likelihood function of a model, and
therefore the value of its \(AIC\), both depend on the size of the
sample, which is why \(AIC\) is not comparable across models estimated
with different sample sizes. For this reason, the full sample models
cannot be compared directly to the models estimated with subsets of
data. The models in the ensembles, however, can be compared to each
other (Table \ref{tab:model-subsample-summary}). As seen in the table,
introducing opponent variables leads to a better fit in the case of
most, but not all models. The simplest model (single-level without
opponent effects) is clearly the best fitting candidate in the case of
bicycle vs light truck collisions, bicycle vs heavy vehicle collisions,
motorcyclist vs light duty vehicle collisions, and motorcyclist vs heavy
vehicle collisions. Model 1 is a statistical toss for best performance
with two competing models in the case of pedestrian vs heavy vehicle
collisions. The relative likelihood of Model 1 compared to Models 2 and
3 in this case is 0.56, which means that these two models are 0.56 times
as probable as Model 1 to minimize the information loss.

Model 2 is the best fit in the case of light truck vs heavy vehicle
collisions. This model is also tied for best fit with Model 2 in the
case of pedestrian vs light duty vehicle and pedestrian vs light truck
collisions, and is a statistical toss with Model 4 in the case of heavy
vehicle vs heavy vehicle collisions (relative likelihood is 0.592).
Model 3 is the best fit in the case of light duty vehicle vs light duty
vehicle collisions and heavy vehicle vs light duty vehicle. Model 4 is
the best fit in the case of light duty vehicle vs light truck
collisions, light duty vehicle vs heavy vehicle collisions, light truck
vs light duty vehicle collisions, heavy vehicle vs light truck
collisions, and motorcycle vs light truck collisions. This model is a
statistical toss with Model 2 in the case of light truck vs light truck
collisions, with a relative likelihood of 0.791.

These results give some preliminary ideas about the relative performance
of the different modelling strategies. In the next subsection we delve
more deeply into this question by examining the predictive performance
of the various modelling strategies. The results up to this point
indicate that different model specification strategies might work best
when combined with subsampling strategies. For space reasons, from this
point onwards, we will consider the ensembles of models for predictions
and will not compare individual models within the ensembles; this we
suggest is a matter for future research.

\begin{table}

\caption{\label{tab:model-full-sample-summary}\label{tab:model-full-sample-summary}Summary of model estimation results: Full sample models}
\centering
\fontsize{7}{9}\selectfont
\begin{tabular}[t]{lccc}
\toprule
Model & \makecell[l]{Number of\\observations} & \makecell[l]{Number of\\coefficients} & AIC\\
\midrule
\rowcolor{gray!6}  Model 1 & 164,511 & 102 & 195,215\\
Model 2 & 164,511 & 108 & 178,943\\
\rowcolor{gray!6}  Model 3 & 164,511 & 118 & 181,333\\
Model 4 & 164,511 & 111 & 179,018\\
\bottomrule
\end{tabular}
\end{table}

\begin{landscape}\begin{table}

\caption{\label{tab:model-subsample-summary}\label{tab:model-summary}Summary of model estimation results: Subsample Models}
\centering
\resizebox{\linewidth}{!}{
\fontsize{7}{9}\selectfont
\begin{tabular}[t]{lccccccccc}
\toprule
\multicolumn{1}{c}{ } & \multicolumn{1}{c}{ } & \multicolumn{2}{c}{Model 1} & \multicolumn{2}{c}{Model 2} & \multicolumn{2}{c}{Model 3} & \multicolumn{2}{c}{Model 4} \\
\cmidrule(l{3pt}r{3pt}){3-4} \cmidrule(l{3pt}r{3pt}){5-6} \cmidrule(l{3pt}r{3pt}){7-8} \cmidrule(l{3pt}r{3pt}){9-10}
Model & \makecell[l]{Number of\\observations} & \makecell[l]{Number of\\coefficients} & AIC & \makecell[l]{Number of\\coefficients} & AIC & \makecell[l]{Number of\\coefficients} & AIC & \makecell[l]{Number of\\coefficients} & AIC\\
\midrule
\rowcolor{gray!6}  Light duty vehicle vs light duty vehicle & 114,841 & 94 & 145,390 & 97 & 143,903 & 100 & 143,896 & 100 & 144,004\\
Light duty vehicle vs light truck & 3,237 & 79 & 3,943 & 82 & 3,927 & 85 & 3,937 & 85 & 3,922\\
\rowcolor{gray!6}  Light duty vehicle vs heavy vehicle & 5,013 & 88 & 5,895 & 91 & 5,878 & 94 & 5,888 & 94 & 5,864\\
Light truck vs light duty vehicle & 3,121 & 79 & 3,885 & 82 & 3,877 & 85 & 3,881 & 85 & 3,875\\
\rowcolor{gray!6}  Light truck vs light truck & 809 & 67 & 1,170 & 70 & 1,156 & 73 & 1,162 & 73 & 1,155\\
\addlinespace
Light truck vs heavy vehicle & 198 & 64 & 288 & 67 & 281 & 70 & 287 & 70 & 286\\
\rowcolor{gray!6}  Heavy vehicle vs light duty vehicle & 4,726 & 79 & 4,326 & 84 & 4,283 & 86 & 4,268 & 87 & 4,287\\
Heavy vehicle vs light truck & 180 & 64 & 225 & 65 & 205 & 67 & 207 & 66 & 187\\
\rowcolor{gray!6}  Heavy vehicle vs heavy vehicle & 779 & 74 & 1,147 & 77 & 1,136 & 80 & 1,141 & 80 & 1,137\\
Pedestrian vs light duty vehicle & 7,176 & 88 & 2,826 & 91 & 2,821 & 91 & 2,821 & 93 & 2,827\\
\addlinespace
\rowcolor{gray!6}  Pedestrian vs light truck & 328 & 62 & 202 & 65 & 200 & 65 & 200 & 68 & 206\\
Pedestrian vs heavy vehicle & 376 & 64 & 409 & 67 & 410 & 67 & 410 & 70 & 417\\
\rowcolor{gray!6}  Bicyclist vs light duty vehicle & 3,521 & 80 & 654 & 83 & 659 & 83 & 659 & 85 & 657\\
Bicyclist vs light truck & 116 & 42 & 84 & 57 & 114 & 57 & 114 & 54 & 108\\
\rowcolor{gray!6}  Bicyclist vs heavy vehicle & - & - & - & - & - & - & - & - & -\\
\addlinespace
Motorcyclist vs light duty vehicle & 2,298 & 78 & 1,367 & 81 & 1,373 & 81 & 1,373 & 84 & 1,373\\
\rowcolor{gray!6}  Motorcyclist vs light truck & 127 & 56 & 153 & 59 & 153 & 59 & 153 & 47 & 94\\
Motorcyclist vs heavy vehicle & 62 & 43 & 88 & 45 & 90 & 46 & 92 & 51 & 102\\
\bottomrule
\multicolumn{10}{l}{\textit{Note: }}\\
\multicolumn{10}{l}{There are zero cases of Bicyclist vs heavy vehicle in the sample}\\
\end{tabular}}
\end{table}
\end{landscape}

\hypertarget{outcome-shares-based-on-predicted-probabilities}{%
\subsection{Outcome shares based on predicted
probabilities}\label{outcome-shares-based-on-predicted-probabilities}}

In this, and the following section, \emph{backcasting} refers to the
prediction of probabilities and classes of outcomes using the 2016
dataset. When conducting backcasting, the dataset is preprocessed in
identical manner as the 2017 dataset. In addition, the variables used in
backcasting match exactly those in the models. This means that some
variables were dropped when they were present in the 2016 dataset but
not in the models. This tended to happen in the case of relatively rare
outcomes (e.g., in 2016, there was at least one collision between a
heavy vehicle and a light duty vehicle in a school crossing zone; no
such event was observed in 2017).

The shares of each outcome are calculated as the sum of the estimated
probabilities for each observation:

\begin{equation}
\label{eq:predicted-shares}
\begin{array}{rcl}
\hat{S}_{\text{PDO}} &=& \sum_{itk}\hat{P}(y_{itk}=\text{PDO})\\
\hat{S}_{\text{injury}} &=& \sum_{itk}\hat{P}(y_{itk}=\text{injury})\\
\hat{S}_{\text{fatality}} &=& \sum_{itk}\hat{P}(y_{itk}=\text{fatality})
\end{array}
\end{equation}

\noindent where \(\hat{P}(y_{itk}=h_w)\) is the estimated probability of
outcome \(h_w\) for individual \(i\) in traffic unit \(t\) and crash
\(k\). The estimated share of outcome \(h\) is \(\hat{S}_{h_w}\).

The estimated shares can be used to assess the ability of the model to
forecast for the population the total number of cases of each outcome. A
summary statistic useful to evaluate the performance is the Average
Percentage Error, or \(APE\) (see Bogue et al., 2017, p. 31), which is
calculated for each outcome as follows:

\begin{equation}
\label{eq:APE}
APE_{h_w} = \Bigg|\frac{\hat{S}_{h_w} - S_{h_w}}{S_{h_w}}\Bigg|\times 100
\end{equation}

The Weighted Average Percentage Error (\(WAPE\)) aggregates the \(APE\)
as follows:

\begin{equation}
\label{eq:WAPE}
WAPE = \frac{\sum_w^WAPE_{h_w}\times S_{h_w}}{\sum_w^WS_{h_w}}
\end{equation}

The results of this exercise are reported in Table
\ref{tab:ape-results}. Of the four full-sample models (Models 1-4), the
\(APE\) of Model 2 is lowest in the nowcasting exercise for every
outcome, with the exception of Fatality, where Model 4 produces a
considerably lower \(APE\). When the results are aggregated by means of
the \(WAPE\), Model 2 gives marginally better results than Model 4. It
is interesting to see that the four ensemble models have lower \(APE\)
values across the board in the nowcasting exercise, and much better
\(WAPE\) than the full sample models. However, once we turn to the
results of the backcasting exercise, these results do not hold, and it
is possible to see that the Average Percentage Errors of the ensemble
worsen considerably, particularly in the case of Fatality. The Weighted
Average Prediction Error of the ensemble models in the backcasting
exercise is also worse than for any of the full sample models. Excellent
in-sample predictions but mediocre out-of-sample predictions are often
evidence of overfitting, as in the case of the ensemble models here.

In terms of backcasting, full sample Model 1 is marginally better than
full sample Models 2 and 3, and better than full sample Model 4. The
reason for this is the lower \(APE\) of Model 1 when predicting Injury,
the most frequent outcome. However, the performance of Model 1 with
respect to Fatality (the least frequent outcome) is the worst of all
models. Whereas Model 4 has the best performance predicting Fatality,
its performance with respect to other classes of outcomes is less
impressive. Model 3 does better than Model 2 with respect to Injury, but
performs relatively poorly when backcasting Fatality. Overall, Model 2
appears to be the most balanced, with good in-sample performance and
competitive out-of-sample performance that is also balanced with respect
to the various classes of outcomes.

\begin{table}

\caption{\label{tab:table-ape-results}\label{tab:ape-results}Predicted shares and average prediction errors (APE) by model (percentages)}
\centering
\fontsize{7}{9}\selectfont
\begin{tabular}[t]{lrrrrrrrrrr}
\toprule
\multicolumn{1}{c}{} & \multicolumn{3}{c}{No Injury} & \multicolumn{3}{c}{Injury} & \multicolumn{3}{c}{Fatality} & \multicolumn{1}{c}{} \\
\cmidrule(l{3pt}r{3pt}){2-4} \cmidrule(l{3pt}r{3pt}){5-7} \cmidrule(l{3pt}r{3pt}){8-10}
Model & Observed & Predicted & APE & Observed & Predicted & APE & Observed & Predicted & APE & WAPE\\
\rowcolor{gray!15}
\midrule
\addlinespace[0.3em]
\multicolumn{11}{l}{\textbf{In-sample (nowcasting using 2017 dataset, i.e., estimation dataset)}}\\
\hspace{1em}Model 1 & 78886 & 79029.00 & 0.18 & 84675 & 84533.74 & 0.17 & 950 & 948.26 & 0.18 & 0.17\\
\rowcolor{gray!15}
\hspace{1em}Model 2 & 78886 & 78928.98 & 0.05 & 84675 & 84641.94 & 0.04 & 950 & 940.08 & 1.04 & 0.05\\
\hspace{1em}Model 3 & 78886 & 79027.29 & 0.18 & 84675 & 84512.50 & 0.19 & 950 & 971.21 & 2.23 & 0.20\\
\rowcolor{gray!15}
\hspace{1em}Model 4 & 78886 & 78939.18 & 0.07 & 84675 & 84622.54 & 0.06 & 950 & 949.28 & 0.08 & 0.06\\
\hspace{1em}Model 1 Ensemble & 62413 & 62402.78 & 0.02 & 83564 & 83573.58 & 0.01 & 931 & 931.64 & 0.07 & 0.01\\
\rowcolor{gray!15}
\hspace{1em}Model 2 Ensemble & 62417 & 62407.00 & 0.02 & 83595 & 83604.14 & 0.01 & 931 & 931.86 & 0.09 & 0.01\\
\rowcolor{gray!15}
\hspace{1em}Model 3 Ensemble & 62411 & 62401.23 & 0.02 & 83596 & 83604.71 & 0.01 & 933 & 934.06 & 0.11 & 0.01\\
\hspace{1em}Model 4 Ensemble & 62405 & 62395.28 & 0.02 & 83578 & 83586.75 & 0.01 & 932 & 932.97 & 0.10 & 0.01\\
\rowcolor{gray!15}
\addlinespace[0.3em]
\multicolumn{11}{l}{\textbf{Out-of-sample (backcasting using 2016 dataset)}}\\
\hspace{1em}Model 1 & 96860 & 96364.67 & 0.51 & 101605 & 102002.59 & 0.39 & 1109 & 1206.74 & 8.81 & 0.50\\
\rowcolor{gray!15}
\hspace{1em}Model 2 & 96860 & 96361.41 & 0.51 & 101605 & 102112.08 & 0.50 & 1109 & 1100.51 & 0.77 & 0.51\\
\hspace{1em}Model 3 & 96860 & 96354.01 & 0.52 & 101605 & 102086.18 & 0.47 & 1109 & 1133.82 & 2.24 & 0.51\\
\rowcolor{gray!15}
\hspace{1em}Model 4 & 96860 & 96325.85 & 0.55 & 101605 & 102136.72 & 0.52 & 1109 & 1111.43 & 0.22 & 0.54\\
\hspace{1em}Model 1 Ensemble & 77457 & 76822.49 & 0.82 & 100013 & 100580.60 & 0.57 & 1072 & 1138.91 & 6.24 & 0.71\\
\rowcolor{gray!15}
\hspace{1em}Model 2 Ensemble & 77459 & 76799.11 & 0.85 & 100049 & 100630.48 & 0.58 & 1071 & 1149.41 & 7.32 & 0.74\\
\hspace{1em}Model 3 Ensemble & 77459 & 76786.76 & 0.87 & 100050 & 100644.29 & 0.59 & 1072 & 1149.95 & 7.27 & 0.75\\
\rowcolor{gray!15}
Model 4 Ensemble & 77461 & 76766.08 & 0.90 & 100029 & 100630.21 & 0.60 & 1070 & 1163.71 & 8.76 & 0.78\\
\bottomrule
\end{tabular}
\end{table}

\hypertarget{outcome-frequency-based-on-predicted-classes}{%
\subsection{Outcome frequency based on predicted
classes}\label{outcome-frequency-based-on-predicted-classes}}

\(APE\) and \(WAPE\) are summary measures of the performance of models
at the aggregated level. Aggregate-level predictions (i.e., shares of
outcomes) are of interest from a population health perspective. In other
cases, an analyst might be interested in the predicted outcomes at the
individual level. In this section we examine the frequency of outcomes
based on predicted classes, using the same two settings as above:
nowcasting and backcasting.

The individual-level outcomes are examined using a battery of
verification statistics. Verification statistics are widely used in the
evaluation of predective approaches were the outcomes are categorical,
and are often based on the analysis of \emph{confusion matrices} (e.g.,
Provost and Kohavi, 1998; Beguería, 2006). Confusion matrices are
cross-tabulations of \emph{observed} and \emph{predicted} classes. In a
two-by-two confusion matrix there are four possible combinations of
observed to predicted classes: hits, misses, false alarms, and correct
non-events, as shown in Table \ref{tab:confusion-matrix}. When the
outcome has more than two classes, the confusion matrix is converted to
a two-by-two table to calculate verification statistics.

\begin{table}[!h]

\caption{\label{tab:example-confusion-matrix}\label{tab:confusion-matrix}Example of a two-by-two confusion matrix}
\centering
\fontsize{7}{9}\selectfont
\begin{tabular}[t]{lccc}
\toprule
\multicolumn{1}{c}{ } & \multicolumn{2}{c}{Observed} & \multicolumn{1}{c}{ } \\
\cmidrule(l{3pt}r{3pt}){2-3}
Predicted & Yes & No & Marginal Total\\
\midrule
\rowcolor{gray!6}  Yes & Hit & False Alarm & Predicted Yes\\
No & Miss & Correct Non-event & Predicted No\\
\rowcolor{gray!6}  Marginal Total & Observed Yes & Observed No & \\
\bottomrule
\end{tabular}
\end{table}

The statistics used in our assessment are summarized in Table
\ref{tab:verification-statistics}, including brief descriptions of their
interpretation. The statistics evaluate the performance of models from
different perspectives. Take as examples Percent Correct (\(PC\)) is the
sum of hits and correct rejections divided by the number of cases; Bias
(\(B\)), which measures whether a class has been over- or
under-predicted; and the probability of detection (\(POD\)) is the
fraction of hits relative to the total observed for the corresponding
class. The results of calculating the battery of verification statistics
are shown in Table \ref{tab:nowcast-outcomes-assessment} for the
nowcasting exercise, and Table \ref{tab:backcast-outcomes-assessment}
for the backcasting exercise.

\begin{landscape}\begin{table}

\caption{\label{tab:table-verification-statistics}\label{tab:verification-statistics}Verification statistics}
\centering
\resizebox{\linewidth}{!}{
\fontsize{7}{9}\selectfont
\begin{tabular}[t]{l>{\raggedright\arraybackslash}p{30em}>{\raggedright\arraybackslash}p{30em}}
\toprule
Statistic & Description & Notes\\
\midrule
\rowcolor{gray!6}  Percent Correct ($PC$) & Total hits and correct rejections divided by number of cases & Strongly influenced by most common category\\
Percent Correct by Class ($PC_c$) & Same as Percent Correct but by category & Strongly influenced by most common category\\
\rowcolor{gray!6}  Bias ($B$) & Total predicted by category, divided by total observed by category & $B>1$: class is overpredicted; $B<1$: class is underpredicted\\
Critical Success Index ($CSI$) & Total hits divided by total hits + false alarms + misses & $CSI = 1$: perfect score; $CSI = 0$: no skill\\
\rowcolor{gray!6}  Probability of False Detection ($F$) & Proportion of no events forecast as yes; sensitive to false alarms but ignores misses & $F = 0$: perfect score\\
\addlinespace
Probability of Detection ($POD$) & Total hits divided by total observed by class & $POD = 1$: perfect score\\
\rowcolor{gray!6}  False Alarm Ratio ($FAR$) & Total false alarms divided by total forecast yes by class; measures fraction of predicted yes that did not occur & $FAR = 0$: perfect score\\
Heidke Skill Score ($HSS$) & Fraction of correct predictions after removing predictions attributable to chance; measures fractional improvement over random; tends to reward conservative forecasts & $HSS = 1$: perfect score; $HSS = 0$: no skill; $HSS < 0$: random is better\\
\rowcolor{gray!6}  Peirce Skill Score ($PSS$) & Combines $POD$ and $F$; measures ability to separate yes events from no events; tends to reward conservative forecasts & $PSS = 1$: perfect score; $PSS = 0$: no skill\\
Gerrity Score ($GS$) & Measures accuracy of predicting the correct category, relative to random; tends to reward correct forecasts of less likely category & $GS = 1$: perfect score; $GS = 0$: no skill\\
\bottomrule
\end{tabular}}
\end{table}
\end{landscape}

\hypertarget{nowcasting}{%
\subsubsection{Nowcasting}\label{nowcasting}}

First, some general remarks regarding the results are in order. It is
clear that no model performs consistently well from every perspective.
Recalling Box's maxim, all models are wrong - in this case it just so
happens that the models are wrong in subtly different ways.

The model that performs best in terms of Percent Correct and is Model 2,
followed by Model 4. The worst performer from this perspective is Model
1, with a \(PC\) score several percentage points below the top models.
The second score is Percent Correct by Class (\(PC_c\)). This score is
calculated individually for each outcome class. Model 2, again, has the
best performance for outcomes No Injury and Injury, and the second best
score for Fatality. Model 4 has the best score for Fatality, and is
second best for No Injury and Injury. Model 1 (full sample) has worst
scores for No Injury and Injury whereas its ensemble version has the
worst score for Fatality. It is important to note that Percent Correct
and Percent Correct by Class are heavily influenced by the most common
category. This can be appreciated in the scores, particularly for
Fatality, which has generally high values of the score, despite the fact
that the number of hits in that class are relatively low; the high
values in this case are due to the high ocurrence of correct rejections
elsewhere in the table.

Bias, in contrast, measures for each outcome class, the proportion of
total predictions by category (e.g., hits as well as false alarms)
relative to the total observed for that class. Overall, predictions can
have low bias (values closer to 1) but still do poorly in terms of hits.
The models with the best performance in terms of Bias are Model 1 for No
Injury and Injury, and Model 3 (ensemble) for Fatality. Model 4 is the
second best performer for No Injury and Injury, and Model 4 (ensemble)
is second best performer for Fatality. Model 1 (ensemble) has the worst
bias for No Injury and Injury, whereas Model 2 has the worst bias for
Fatality.

Critical Success Index (\(CSI\)) is a verification statistic that
assumes that the number of correct non-events is inconsequential for
determining forecasting skill. For this reason, the statistic is
calculated as the proportion of hits relative to the sum of hits plus
false alarms plus misses. No model performs uniformly best from this
perspective. Model 1 has the best \(CSI\) for No Injury, Model 2
(ensemble) has the best score for Injury, and Model 4 (ensemble) the
best score for Fatality. Model 2 has the worst score for Fatality - this
indicates that Model 4 is not particularly skilled at predicting
Fatalities correctly, given the frequency with which it gives false
alarms for this class, or misses it.

The next statistic is Probability of False Detection (\(F\)). This is
the proportion of false alarms relative to the total observed no events.
This statistic measure the frequency with which the model incorrectly
predicts an event, but not when it incorrectly misses it. The

\begin{table}[!h]

\caption{\label{tab:nowcast-outcomes-assessment}\label{tab:nowcast-outcomes-assessment}Assessment of in-sample outcomes (nowcasting using 2017 dataset, i.e., estimation dataset)}
\centering
\resizebox{\linewidth}{!}{
\fontsize{7}{9}\selectfont
\begin{tabular}[t]{lrrrllllllllll}
\toprule
\multicolumn{1}{c}{Predicted} & \multicolumn{3}{c}{Observed Outcome} & \multicolumn{10}{c}{Verification Statistics} \\
\cmidrule(l{3pt}r{3pt}){1-1} \cmidrule(l{3pt}r{3pt}){2-4} \cmidrule(l{3pt}r{3pt}){5-14}
Outcome & No Injury & Injury & Fatality & \makecell[l]{Percent\\Correct} & \makecell[l]{Percent\\Correct\\by Class} & Bias$^1$ & \makecell[l]{Critical\\Success\\Index$^2$} & \makecell[l]{Probability of\\False\\Detection$^3$} & \makecell[l]{Probability\\of\\Detection$^4$} & \makecell[l]{False\\Alarm\\Ratio$^5$} & \makecell[l]{Heidke\\Skill\\Score$^6$} & \makecell[l]{Peirce\\Skill\\Score$^7$} & \makecell[l]{Gerrity\\Score$^8$}\\
\midrule
\addlinespace[0.3em]
\multicolumn{14}{l}{\textbf{Model 1}}\\
\hspace{1em}No Injury & 50652 & 22503 & 150 &  & \textcolor{red}{69.07} & \textcolor{black}{\textbf{0.9293}} & \textcolor{black}{0.4988} & \textcolor{red}{0.2646} & \textcolor{black}{0.6421} & \textcolor{black}{0.309} &  &  & \\

\hspace{1em}Injury & 28232 & 62121 & 797 &  & \textcolor{red}{68.64} & \textcolor{black}{\textbf{1.0765}} & \textcolor{red}{0.5463} & \textcolor{black}{0.3636} & \textcolor{red}{0.7336} & \textcolor{red}{0.3185} &  &  & \\

Fatality & 2 & 51 & 3 & \multirow{-3}{*}{\raggedright\arraybackslash \textcolor{red}{68.55}} & \textcolor{black}{99.39} & \textcolor{black}{0.0589} & \textcolor{black}{0.003} & \textcolor{red}{3e-04} & \textcolor{black}{0.0032} & \textcolor{black}{0.9464} & \multirow{-3}{*}{\raggedright\arraybackslash \textcolor{black}{0.3725}} & \multirow{-3}{*}{\raggedright\arraybackslash \textcolor{black}{0.3696}} & \multirow{-3}{*}{\raggedright\arraybackslash \textcolor{red}{0.1902}}\\
\cmidrule{1-14}
\addlinespace[0.3em]
\multicolumn{14}{l}{\textbf{Model 2}}\\
\hspace{1em}No Injury & 51531 & 17137 & 85 &  & \textcolor{black}{\textbf{72.9}} & \textcolor{black}{0.8715} & \textcolor{black}{\textbf{0.5362}} & \textcolor{black}{0.2011} & \textcolor{black}{\underline{0.6532}} & \textcolor{black}{\textbf{0.2505}} &  &  & \\

\hspace{1em}Injury & 27355 & 67514 & 864 &  & \textcolor{black}{\textbf{72.42}} & \textcolor{black}{1.1306} & \textcolor{black}{0.598} & \textcolor{black}{\underline{0.3535}} & \textcolor{black}{0.7973} & \textcolor{black}{0.2948} &  &  & \\

Fatality & 0 & 24 & 1 & \multirow{-3}{*}{\raggedright\arraybackslash \textcolor{black}{\textbf{72.36}}} & \textcolor{black}{\underline{99.41}} & \textcolor{red}{0.0263} & \textcolor{red}{0.001} & \textcolor{black}{1e-04} & \textcolor{red}{0.0011} & \textcolor{red}{0.96} & \multirow{-3}{*}{\raggedright\arraybackslash \textcolor{black}{\textbf{0.4474}}} & \multirow{-3}{*}{\raggedright\arraybackslash \textcolor{black}{\textbf{0.4429}}} & \multirow{-3}{*}{\raggedright\arraybackslash \textcolor{black}{\underline{0.2265}}}\\
\cmidrule{1-14}
\addlinespace[0.3em]
\multicolumn{14}{l}{\textbf{Model 3}}\\
\hspace{1em}No Injury & 51101 & 17296 & 79 &  & \textcolor{black}{72.55} & \textcolor{black}{0.868} & \textcolor{black}{0.5309} & \textcolor{black}{0.2029} & \textcolor{black}{0.6478} & \textcolor{black}{0.2537} &  &  & \\

\hspace{1em}Injury & 27785 & 67338 & 868 &  & \textcolor{black}{72.04} & \textcolor{black}{1.1336} & \textcolor{black}{0.5942} & \textcolor{black}{0.3589} & \textcolor{black}{0.7953} & \textcolor{black}{0.2985} &  &  & \\

Fatality & 0 & 41 & 3 & \multirow{-3}{*}{\raggedright\arraybackslash \textcolor{black}{72}} & \textcolor{black}{99.4} & \textcolor{black}{0.0463} & \textcolor{black}{0.003} & \textcolor{black}{3e-04} & \textcolor{black}{0.0032} & \textcolor{black}{0.9318} & \multirow{-3}{*}{\raggedright\arraybackslash \textcolor{black}{0.44}} & \multirow{-3}{*}{\raggedright\arraybackslash \textcolor{black}{0.4356}} & \multirow{-3}{*}{\raggedright\arraybackslash \textcolor{black}{0.2239}}\\
\cmidrule{1-14}
\addlinespace[0.3em]
\multicolumn{14}{l}{\textbf{Model 4}}\\
\hspace{1em}No Injury & 51575 & 17318 & 84 &  & \textcolor{black}{\underline{72.82}} & \textcolor{black}{\underline{0.8744}} & \textcolor{black}{\underline{0.5356}} & \textcolor{black}{0.2032} & \textcolor{black}{\textbf{0.6538}} & \textcolor{black}{\underline{0.2523}} &  &  & \\

\hspace{1em}Injury & 27311 & 67334 & 863 &  & \textcolor{black}{\underline{72.33}} & \textcolor{black}{\underline{1.1279}} & \textcolor{black}{0.5967} & \textcolor{black}{\textbf{0.3529}} & \textcolor{black}{0.7952} & \textcolor{black}{0.295} &  &  & \\

Fatality & 0 & 23 & 3 & \multirow{-3}{*}{\raggedright\arraybackslash \textcolor{black}{\underline{72.28}}} & \textcolor{black}{\textbf{99.41}} & \textcolor{black}{0.0274} & \textcolor{black}{0.0031} & \textcolor{black}{1e-04} & \textcolor{black}{0.0032} & \textcolor{black}{0.8846} & \multirow{-3}{*}{\raggedright\arraybackslash \textcolor{black}{\underline{0.4458}}} & \multirow{-3}{*}{\raggedright\arraybackslash \textcolor{black}{\underline{0.4414}}} & \multirow{-3}{*}{\raggedright\arraybackslash \textcolor{black}{\textbf{0.2268}}}\\
\cmidrule{1-14}
\addlinespace[0.3em]
\multicolumn{14}{l}{\textbf{Model 1 Ensemble}}\\
\hspace{1em}No Injury & 34664 & 16434 & 63 &  & \textcolor{black}{69.88} & \textcolor{red}{0.8197} & \textcolor{red}{0.4393} & \textcolor{black}{0.1952} & \textcolor{red}{0.5554} & \textcolor{red}{0.3225} &  &  & \\

\hspace{1em}Injury & 27749 & 67120 & 829 &  & \textcolor{black}{69.35} & \textcolor{red}{1.1452} & \textcolor{black}{0.5985} & \textcolor{red}{0.4512} & \textcolor{black}{0.8032} & \textcolor{black}{0.2986} &  &  & \\

Fatality & 0 & 10 & 39 & \multirow{-3}{*}{\raggedright\arraybackslash \textcolor{black}{69.31}} & \textcolor{red}{99.39} & \textcolor{black}{0.0526} & \textcolor{black}{0.0414} & \textcolor{black}{\underline{1e-04}} & \textcolor{black}{0.0419} & \textcolor{black}{\underline{0.2041}} & \multirow{-3}{*}{\raggedright\arraybackslash \textcolor{red}{0.3626}} & \multirow{-3}{*}{\raggedright\arraybackslash \textcolor{red}{0.3521}} & \multirow{-3}{*}{\raggedright\arraybackslash \textcolor{black}{0.201}}\\
\cmidrule{1-14}
\addlinespace[0.3em]
\multicolumn{14}{l}{\textbf{Model 2 Ensemble}}\\
\hspace{1em}No Injury & 35443 & 16145 & 60 &  & \textcolor{black}{70.62} & \textcolor{black}{0.8275} & \textcolor{black}{0.4508} & \textcolor{black}{\textbf{0.1917}} & \textcolor{black}{0.5678} & \textcolor{black}{0.3138} &  &  & \\

\hspace{1em}Injury & 26974 & 67437 & 829 &  & \textcolor{black}{70.08} & \textcolor{black}{1.1393} & \textcolor{black}{\textbf{0.6054}} & \textcolor{black}{0.4389} & \textcolor{black}{\textbf{0.8067}} & \textcolor{black}{0.2919} &  &  & \\

Fatality & 0 & 13 & 42 &  & \textcolor{black}{99.39} & \textcolor{black}{0.0591} & \textcolor{black}{0.0445} & \textcolor{black}{1e-04} & \textcolor{black}{0.0451} & \textcolor{black}{0.2364} & \multirow{-3}{*}{\raggedright\arraybackslash \textcolor{black}{0.3784}} & \multirow{-3}{*}{\raggedright\arraybackslash \textcolor{black}{0.3678}} & \multirow{-3}{*}{\raggedright\arraybackslash \textcolor{black}{0.2106}}\\

\addlinespace[0.3em]
\multicolumn{14}{l}{\textbf{Model 3 Ensemble}}\\
\hspace{1em}No Injury & 35498 & 16204 & 60 &  & \textcolor{black}{70.62} & \textcolor{black}{0.8294} & \textcolor{black}{0.4512} & \textcolor{black}{\underline{0.1924}} & \textcolor{black}{0.5688} & \textcolor{black}{0.3142} &  &  & \\

\hspace{1em}Injury & 26913 & 67379 & 828 &  & \textcolor{black}{70.08} & \textcolor{black}{1.1379} & \textcolor{black}{\underline{0.6052}} & \textcolor{black}{0.4379} & \textcolor{black}{\underline{0.806}} & \textcolor{black}{\underline{0.2916}} &  &  & \\

Fatality & 0 & 13 & 45 & \multirow{-6}{*}{\raggedright\arraybackslash \textcolor{black}{70.04}} & \textcolor{black}{99.39} & \textcolor{black}{\textbf{0.0622}} & \textcolor{black}{\underline{0.0476}} & \textcolor{black}{1e-04} & \textcolor{black}{\underline{0.0482}} & \textcolor{black}{0.2241} & \multirow{-3}{*}{\raggedright\arraybackslash \textcolor{black}{0.3786}} & \multirow{-3}{*}{\raggedright\arraybackslash \textcolor{black}{0.3681}} & \multirow{-3}{*}{\raggedright\arraybackslash \textcolor{black}{0.2123}}\\
\cmidrule{1-14}
\addlinespace[0.3em]
\multicolumn{14}{l}{\textbf{Model 4 Ensemble}}\\
\hspace{1em}No Injury & 35553 & 16297 & 59 &  & \textcolor{black}{70.59} & \textcolor{black}{0.8318} & \textcolor{black}{0.4514} & \textcolor{black}{0.1935} & \textcolor{black}{0.5697} & \textcolor{black}{0.3151} &  &  & \\

\hspace{1em}Injury & 26852 & 67271 & 827 &  & \textcolor{black}{70.06} & \textcolor{black}{1.1361} & \textcolor{black}{0.6046} & \textcolor{black}{0.437} & \textcolor{black}{0.8049} & \textcolor{black}{\textbf{0.2915}} &  &  & \\

Fatality & 0 & 10 & 46 & \multirow{-3}{*}{\raggedright\arraybackslash \textcolor{black}{70.02}} & \textcolor{black}{99.39} & \textcolor{black}{\underline{0.0601}} & \textcolor{black}{\textbf{0.0488}} & \textcolor{black}{\textbf{1e-04}} & \textcolor{black}{\textbf{0.0494}} & \textcolor{black}{\textbf{0.1786}} & \multirow{-3}{*}{\raggedright\arraybackslash \textcolor{black}{0.3783}} & \multirow{-3}{*}{\raggedright\arraybackslash \textcolor{black}{0.3679}} & \multirow{-3}{*}{\raggedright\arraybackslash \textcolor{black}{0.2127}}\\
\bottomrule
\multicolumn{14}{l}{\textit{Note: }}\\
\multicolumn{14}{l}{Bold numbers: best scores; underlined numbers: second best scores; red numbers: worst scores}\\
\multicolumn{14}{l}{\textsuperscript{1} $B>1$: class is overpredicted; $B<1$: class is underpredicted; }\\
\multicolumn{14}{l}{\textsuperscript{2} $CSI = 1$: perfect score; $CSI = 0$: no skill; }\\
\multicolumn{14}{l}{\textsuperscript{3} $F = 0$: perfect score; }\\
\multicolumn{14}{l}{\textsuperscript{4} $POD = 1$: perfect score; }\\
\multicolumn{14}{l}{\textsuperscript{5} $FAR = 0$: perfect score; }\\
\multicolumn{14}{l}{\textsuperscript{6} $HSS = 1$: perfect score; $HSS = 0$: no skill; $HSS < 0$: random is better; }\\
\multicolumn{14}{l}{\textsuperscript{7} $PSS = 1$: perfect score; $PSS = 0$: no skill; }\\
\multicolumn{14}{l}{\textsuperscript{8} $GS = 1$: perfect score; $GS = 0$: no skill.}\\
\end{tabular}}
\end{table}

\begin{table}[!h]

\caption{\label{tab:backcast-outcomes-assessment}\label{tab:nowcast-outcomes-assessment}Assessment of out-of-sample outcomes (backcasting using 2016 dataset)}
\centering
\resizebox{\linewidth}{!}{
\fontsize{7}{9}\selectfont
\resizebox{\linewidth}{!}{
\begin{tabular}[t]{lrrrllllllllll}
\toprule
\multicolumn{1}{c}{Predicted} & \multicolumn{3}{c}{Observed Outcome} & \multicolumn{10}{c}{Verification Statistics} \\
\cmidrule(l{3pt}r{3pt}){1-1} \cmidrule(l{3pt}r{3pt}){2-4} \cmidrule(l{3pt}r{3pt}){5-14}
Outcome & No Injury & Injury & Fatality & \makecell[l]{Percent\\Correct} & \makecell[l]{Percent\\Correct\\by Class} & Bias$^1$ & \makecell[l]{Critical\\Success\\Index$^2$} & \makecell[l]{Probability of\\False\\Detection$^3$} & \makecell[l]{Probability\\of\\Detection$^4$} & \makecell[l]{False\\Alarm\\Ratio$^5$} & \makecell[l]{Heidke\\Skill\\Score$^6$} & \makecell[l]{Peirce\\Skill\\Score$^7$} & \makecell[l]{Gerrity\\Score$^8$}\\
\midrule
\addlinespace[0.3em]
\multicolumn{14}{l}{\textbf{Model 1}}\\
\hspace{1em}No Injury & 61684 & 27447 & 184 &  & \textcolor{red}{68.53} & \textcolor{black}{\textbf{0.9221}} & \textcolor{black}{0.4955} & \textcolor{red}{0.269} & \textcolor{black}{0.6368} & \textcolor{black}{0.3094} &  &  & \\

\hspace{1em}Injury & 35171 & 74073 & 915 &  & \textcolor{red}{68.12} & \textcolor{black}{\textbf{1.0842}} & \textcolor{red}{0.538} & \textcolor{black}{0.3683} & \textcolor{red}{0.729} & \textcolor{red}{0.3276} &  &  & \\

Fatality & 5 & 85 & 10 & \multirow{-3}{*}{\raggedright\arraybackslash \textcolor{red}{68.03}} & \textcolor{black}{99.4} & \textcolor{black}{0.0902} & \textcolor{black}{0.0083} & \textcolor{black}{5e-04} & \textcolor{black}{0.009} & \textcolor{red}{0.9} & \multirow{-3}{*}{\raggedright\arraybackslash \textcolor{black}{0.3628}} & \multirow{-3}{*}{\raggedright\arraybackslash \textcolor{black}{0.3604}} & \multirow{-3}{*}{\raggedright\arraybackslash \textcolor{black}{0.1882}}\\
\cmidrule{1-14}
\addlinespace[0.3em]
\multicolumn{14}{l}{\textbf{Model 2}}\\
\hspace{1em}No Injury & 62735 & 21013 & 106 &  & \textcolor{black}{\textbf{72.32}} & \textcolor{black}{0.8657} & \textcolor{black}{\textbf{0.5317}} & \textcolor{black}{0.2056} & \textcolor{black}{\underline{0.6477}} & \textcolor{black}{\textbf{0.2519}} &  &  & \\

\hspace{1em}Injury & 34125 & 80569 & 996 &  & \textcolor{black}{\textbf{71.86}} & \textcolor{black}{1.1386} & \textcolor{black}{0.5893} & \textcolor{black}{\underline{0.3585}} & \textcolor{black}{0.793} & \textcolor{black}{\textbf{0.3036}} &  &  & \\

Fatality & 0 & 23 & 7 & \multirow{-3}{*}{\raggedright\arraybackslash \textcolor{black}{\textbf{71.81}}} & \textcolor{black}{\textbf{99.44}} & \textcolor{red}{0.0271} & \textcolor{black}{0.0062} & \textcolor{black}{\textbf{1e-04}} & \textcolor{black}{0.0063} & \textcolor{black}{\textbf{0.7667}} & \multirow{-3}{*}{\raggedright\arraybackslash \textcolor{black}{\textbf{0.4372}}} & \multirow{-3}{*}{\raggedright\arraybackslash \textcolor{black}{\textbf{0.4335}}} & \multirow{-3}{*}{\raggedright\arraybackslash \textcolor{black}{\textbf{0.2241}}}\\
\cmidrule{1-14}
\addlinespace[0.3em]
\multicolumn{14}{l}{\textbf{Model 3}}\\
\hspace{1em}No Injury & 62248 & 21133 & 107 &  & \textcolor{black}{72.01} & \textcolor{black}{0.8619} & \textcolor{black}{0.5271} & \textcolor{black}{0.2068} & \textcolor{black}{0.6427} & \textcolor{black}{0.2544} &  &  & \\

\hspace{1em}Injury & 34610 & 80433 & 996 &  & \textcolor{black}{71.55} & \textcolor{black}{1.1421} & \textcolor{black}{0.5862} & \textcolor{black}{0.3634} & \textcolor{black}{0.7916} & \textcolor{black}{0.3068} &  &  & \\

Fatality & 2 & 39 & 6 & \multirow{-3}{*}{\raggedright\arraybackslash \textcolor{black}{71.5}} & \textcolor{black}{99.43} & \textcolor{black}{0.0424} & \textcolor{red}{0.0052} & \textcolor{black}{2e-04} & \textcolor{red}{0.0054} & \textcolor{black}{0.8723} & \multirow{-3}{*}{\raggedright\arraybackslash \textcolor{black}{0.431}} & \multirow{-3}{*}{\raggedright\arraybackslash \textcolor{black}{0.4274}} & \multirow{-3}{*}{\raggedright\arraybackslash \textcolor{black}{0.2205}}\\
\cmidrule{1-14}
\addlinespace[0.3em]
\multicolumn{14}{l}{\textbf{Model 4}}\\
\hspace{1em}No Injury & 62788 & 21246 & 102 &  & \textcolor{black}{\underline{72.23}} & \textcolor{black}{\underline{0.8686}} & \textcolor{black}{\underline{0.5312}} & \textcolor{black}{0.2078} & \textcolor{black}{\textbf{0.6482}} & \textcolor{black}{\underline{0.2537}} &  &  & \\

\hspace{1em}Injury & 34071 & 80332 & 1000 &  & \textcolor{black}{\underline{71.77}} & \textcolor{black}{\underline{1.1358}} & \textcolor{black}{0.5878} & \textcolor{black}{\textbf{0.358}} & \textcolor{black}{0.7906} & \textcolor{black}{0.3039} &  &  & \\

Fatality & 1 & 27 & 7 & \multirow{-3}{*}{\raggedright\arraybackslash \textcolor{black}{\underline{71.72}}} & \textcolor{black}{\underline{99.43}} & \textcolor{black}{0.0316} & \textcolor{black}{0.0062} & \textcolor{black}{\underline{1e-04}} & \textcolor{black}{0.0063} & \textcolor{black}{\underline{0.8}} & \multirow{-3}{*}{\raggedright\arraybackslash \textcolor{black}{\underline{0.4355}}} & \multirow{-3}{*}{\raggedright\arraybackslash \textcolor{black}{\underline{0.4318}}} & \multirow{-3}{*}{\raggedright\arraybackslash \textcolor{black}{\underline{0.2233}}}\\
\cmidrule{1-14}
\addlinespace[0.3em]
\multicolumn{14}{l}{\textbf{Model 1 Ensemble}}\\
\hspace{1em}No Injury & 42896 & 20230 & 95 &  & \textcolor{black}{69.26} & \textcolor{red}{0.8162} & \textcolor{red}{0.4387} & \textcolor{black}{0.2011} & \textcolor{red}{0.5538} & \textcolor{red}{0.3215} &  &  & \\

\hspace{1em}Injury & 34546 & 79692 & 962 &  & \textcolor{black}{68.73} & \textcolor{red}{1.1519} & \textcolor{black}{0.588} & \textcolor{red}{0.4522} & \textcolor{black}{0.7968} & \textcolor{black}{0.3082} &  &  & \\

Fatality & 15 & 91 & 15 & \multirow{-3}{*}{\raggedright\arraybackslash \textcolor{black}{68.67}} & \textcolor{black}{99.35} & \textcolor{black}{0.1129} & \textcolor{black}{0.0127} & \textcolor{black}{6e-04} & \textcolor{black}{0.014} & \textcolor{black}{0.876} & \multirow{-3}{*}{\raggedright\arraybackslash \textcolor{red}{0.3539}} & \multirow{-3}{*}{\raggedright\arraybackslash \textcolor{red}{0.3447}} & \multirow{-3}{*}{\raggedright\arraybackslash \textcolor{red}{0.1831}}\\
\cmidrule{1-14}
\addlinespace[0.3em]
\multicolumn{14}{l}{\textbf{Model 2 Ensemble}}\\
\hspace{1em}No Injury & 43486 & 19937 & 95 &  & \textcolor{black}{69.76} & \textcolor{black}{0.82} & \textcolor{black}{0.4461} & \textcolor{black}{\textbf{0.1981}} & \textcolor{black}{0.5614} & \textcolor{black}{0.3154} &  &  & \\

\hspace{1em}Injury & 33953 & 80009 & 961 &  & \textcolor{black}{69.23} & \textcolor{black}{1.1487} & \textcolor{black}{\textbf{0.5928}} & \textcolor{black}{0.4446} & \textcolor{black}{\textbf{0.7997}} & \textcolor{black}{\underline{0.3038}} &  &  & \\

Fatality & 20 & 103 & 15 & \multirow{-3}{*}{\raggedright\arraybackslash \textcolor{black}{69.16}} & \textcolor{black}{99.34} & \textcolor{black}{\underline{0.1289}} & \textcolor{black}{0.0126} & \textcolor{black}{7e-04} & \textcolor{black}{0.014} & \textcolor{black}{0.8913} & \multirow{-3}{*}{\raggedright\arraybackslash \textcolor{black}{0.3644}} & \multirow{-3}{*}{\raggedright\arraybackslash \textcolor{black}{0.3551}} & \multirow{-3}{*}{\raggedright\arraybackslash \textcolor{black}{0.1883}}\\
\cmidrule{1-14}
\addlinespace[0.3em]
\multicolumn{14}{l}{\textbf{Model 3 Ensemble}}\\
\hspace{1em}No Injury & 43526 & 20033 & 92 &  & \textcolor{black}{69.73} & \textcolor{black}{0.8217} & \textcolor{black}{0.446} & \textcolor{black}{\underline{0.199}} & \textcolor{black}{0.5619} & \textcolor{black}{0.3162} &  &  & \\

\hspace{1em}Injury & 33915 & 79915 & 964 &  & \textcolor{black}{69.19} & \textcolor{black}{1.1474} & \textcolor{black}{\underline{0.5923}} & \textcolor{black}{0.4441} & \textcolor{black}{\underline{0.7988}} & \textcolor{black}{0.3038} &  &  & \\

Fatality & 18 & 102 & 16 & \multirow{-3}{*}{\raggedright\arraybackslash \textcolor{black}{69.13}} & \textcolor{black}{99.34} & \textcolor{black}{0.1269} & \textcolor{black}{\underline{0.0134}} & \textcolor{black}{7e-04} & \textcolor{black}{\underline{0.0149}} & \textcolor{black}{0.8824} & \multirow{-3}{*}{\raggedright\arraybackslash \textcolor{black}{0.3639}} & \multirow{-3}{*}{\raggedright\arraybackslash \textcolor{black}{0.3546}} & \multirow{-3}{*}{\raggedright\arraybackslash \textcolor{black}{0.1885}}\\
\cmidrule{1-14}
\addlinespace[0.3em]
\multicolumn{14}{l}{\textbf{Model 4 Ensemble}}\\
\hspace{1em}No Injury & 43561 & 20160 & 94 &  & \textcolor{black}{69.67} & \textcolor{black}{0.8238} & \textcolor{black}{0.4458} & \textcolor{black}{0.2003} & \textcolor{black}{0.5624} & \textcolor{black}{0.3174} &  &  & \\

\hspace{1em}Injury & 33875 & 79762 & 959 &  & \textcolor{black}{69.14} & \textcolor{black}{1.1456} & \textcolor{black}{0.5914} & \textcolor{black}{0.4436} & \textcolor{black}{0.7974} & \textcolor{black}{0.304} &  &  & \\

Fatality & 25 & 107 & 17 & \multirow{-3}{*}{\raggedright\arraybackslash \textcolor{black}{69.07}} & \textcolor{red}{99.34} & \textcolor{black}{\textbf{0.1393}} & \textcolor{black}{\textbf{0.0141}} & \textcolor{red}{7e-04} & \textcolor{black}{\textbf{0.0159}} & \textcolor{black}{0.8859} & \multirow{-3}{*}{\raggedright\arraybackslash \textcolor{black}{0.3629}} & \multirow{-3}{*}{\raggedright\arraybackslash \textcolor{black}{0.3538}} & \multirow{-3}{*}{\raggedright\arraybackslash \textcolor{black}{0.1886}}\\
\bottomrule
\multicolumn{14}{l}{\textit{Note: }}\\
\multicolumn{14}{l}{Bold numbers: best scores; underlined numbers: second best scores; red numbers: worst scores}\\
\multicolumn{14}{l}{\textsuperscript{1} $B>1$: class is overpredicted; $B<1$: class is underpredicted; }\\
\multicolumn{14}{l}{\textsuperscript{2} $CSI = 1$: perfect score; $CSI = 0$: no skill; }\\
\multicolumn{14}{l}{\textsuperscript{3} $F = 0$: perfect score; }\\
\multicolumn{14}{l}{\textsuperscript{4} $POD = 1$: perfect score; }\\
\multicolumn{14}{l}{\textsuperscript{5} $FAR = 0$: perfect score; }\\
\multicolumn{14}{l}{\textsuperscript{6} $HSS = 1$: perfect score; $HSS = 0$: no skill; $HSS < 0$: random is better; }\\
\multicolumn{14}{l}{\textsuperscript{7} $PSS = 1$: perfect score; $PSS = 0$: no skill; }\\
\multicolumn{14}{l}{\textsuperscript{8} $GS = 1$: perfect score; $GS = 0$: no skill.}\\
\end{tabular}}}
\end{table}

\hypertarget{sec:further-considerations}{%
\section{Further considerations}\label{sec:further-considerations}}

Here I plan to discuss the applicability of the modelling strategy to
advanced modelling techniques (partial proportional odds, heterogeneity,
hierarchical models, etc.)

\hypertarget{sec:concluding-remarks}{%
\section{Concluding remarks}\label{sec:concluding-remarks}}

Different modelling strategies can be used to model complex
hierarchical, multievent outcomes such as the severity of injuries
following a collision. The objective of this paper was to assess the
performance of different strategies to model opponent effects in
two-vehicle crashes. In broad terms, three strategies were considered:
1) incorporating opponent-level variables in the model; 2) single-
versus multi-level model specifications; and 3) sample subsetting and
estimation of separate models for different types of individual-opponent
interactions.

The results of the empirical assessment strongly suggest that
incorporating opponent effects can greatly improve the fit and
predictive performance of a model. There is some evidence that
subsetting the sample can improve the results in some isolated
situations (e.g., when modelling the severity of crashes involving
active travellers or motorcyclists), possibly at the risk of
overfitting. In this paper we did not compare individual models in our
Ensemble approach, but we suggest that this is an avenue for future
research.

In terms of the full sample models, the evidence was not conclusive in
favor of a single-level model with opponent variables (Model 2), or a
hierarchical model with individual-opponent interactions (Model 4). On
the one hand, the \(AIC\) tended to favor the

\hypertarget{references}{%
\section*{References}\label{references}}
\addcontentsline{toc}{section}{References}

\hypertarget{refs}{}
\leavevmode\hypertarget{ref-Amoh2017effect}{}%
Amoh-Gyimah, R., Aidoo, E.N., Akaateba, M.A., Appiah, S.K., 2017. The
effect of natural and built environmental characteristics on
pedestrian-vehicle crash severity in ghana. International Journal of
Injury Control and Safety Promotion 24, 459--468.
doi:\href{https://doi.org/10.1080/17457300.2016.1232274}{10.1080/17457300.2016.1232274}

\leavevmode\hypertarget{ref-Aziz2013exploring}{}%
Aziz, H.M.A., Ukkusuri, S.V., Hasan, S., 2013. Exploring the
determinants of pedestrian-vehicle crash severity in new york city.
Accident Analysis and Prevention 50, 1298--1309.
doi:\href{https://doi.org/10.1016/j.aap.2012.09.034}{10.1016/j.aap.2012.09.034}

\leavevmode\hypertarget{ref-Begueria2006validation}{}%
Beguería, S., 2006. Validation and evaluation of predictive models in
hazard assessment and risk management. Natural Hazards 37, 315--329.
doi:\href{https://doi.org/10.1007/s11069-005-5182-6}{10.1007/s11069-005-5182-6}

\leavevmode\hypertarget{ref-Bogue2017modified}{}%
Bogue, S., Paleti, R., Balan, L., 2017. A modified rank ordered logit
model to analyze injury severity of occupants in multivehicle crashes.
Analytic Methods in Accident Research 14, 22--40.
doi:\href{https://doi.org/10.1016/j.amar.2017.03.001}{10.1016/j.amar.2017.03.001}

\leavevmode\hypertarget{ref-Casetti1972generating}{}%
Casetti, E., 1972. Generating models by the expansion method:
Applications to geographic research. Geographical Analysis 4, 81--91.

\leavevmode\hypertarget{ref-Chang2006analysis}{}%
Chang, L.Y., Wang, H.W., 2006. Analysis of traffic injury severity: An
application of non-parametric classification tree techniques. Accident
Analysis and Prevention 38, 1019--1027.
doi:\href{https://doi.org/10.1016/j.aap.2006.04.009}{10.1016/j.aap.2006.04.009}

\leavevmode\hypertarget{ref-Chen2019investigation}{}%
Chen, F., Song, M.T., Ma, X.X., 2019. Investigation on the injury
severity of drivers in rear-end collisions between cars using a random
parameters bivariate ordered probit model. International Journal of
Environmental Research and Public Health 16.
doi:\href{https://doi.org/10.3390/ijerph16142632}{10.3390/ijerph16142632}

\leavevmode\hypertarget{ref-Chiou2013modeling}{}%
Chiou, Y.C., Hwang, C.C., Chang, C.C., Fu, C., 2013. Modeling
two-vehicle crash severity by a bivariate generalized ordered probit
approach. Accident Analysis and Prevention 51, 175--184.
doi:\href{https://doi.org/10.1016/j.aap.2012.11.008}{10.1016/j.aap.2012.11.008}

\leavevmode\hypertarget{ref-Devlin2019road}{}%
Devlin, A., Beck, B., Simpson, P.M., Ekegren, C.L., Giummarra, M.J.,
Edwards, E.R., Cameron, P.A., Liew, S., Oppy, A., Richardson, M., Page,
R., Gabbe, B.J., 2019. The road to recovery for vulnerable road users
hospitalised for orthopaedic injury following an on-road crash. Accident
Analysis and Prevention 132, 10.
doi:\href{https://doi.org/10.1016/j.aap.2019.105279}{10.1016/j.aap.2019.105279}

\leavevmode\hypertarget{ref-Dissanayake2002factors}{}%
Dissanayake, S., Lu, J.J., 2002. Factors influential in making an injury
severity difference to older drivers involved in fixed object-passenger
car crashes. Accident Analysis and Prevention 34, 609--618.
doi:\href{https://doi.org/10.1016/s0001-4575(01)00060-4}{10.1016/s0001-4575(01)00060-4}

\leavevmode\hypertarget{ref-Duddu2018modeling}{}%
Duddu, V.R., Penmetsa, P., Pulugurtha, S.S., 2018. Modeling and
comparing injury severity of at-fault and not at-fault drivers in
crashes. Accident Analysis and Prevention 120, 55--63.
doi:\href{https://doi.org/10.1016/j.aap.2018.07.036}{10.1016/j.aap.2018.07.036}

\leavevmode\hypertarget{ref-Effati2015geospatial}{}%
Effati, M., Thill, J.C., Shabani, S., 2015. Geospatial and machine
learning techniques for wicked social science problems: Analysis of
crash severity on a regional highway corridor. Journal of Geographical
Systems 17, 107--135.
doi:\href{https://doi.org/10.1007/s10109-015-0210-x}{10.1007/s10109-015-0210-x}

\leavevmode\hypertarget{ref-Gong2017modeling}{}%
Gong, L.F., Fan, W.D., 2017. Modeling single-vehicle run-off-road crash
severity in rural areas: Accounting for unobserved heterogeneity and age
difference. Accident Analysis and Prevention 101, 124--134.
doi:\href{https://doi.org/10.1016/j.aap.2017.02.014}{10.1016/j.aap.2017.02.014}

\leavevmode\hypertarget{ref-Haleem2013effect}{}%
Haleem, K., Gan, A., 2013. Effect of driver's age and side of impact on
crash severity along urban freeways: A mixed logit approach. Journal of
Safety Research 46, 67--76.
doi:\href{https://doi.org/10.1016/j.jsr.2013.04.002}{10.1016/j.jsr.2013.04.002}

\leavevmode\hypertarget{ref-Hanson2013severity}{}%
Hanson, C.S., Noland, R.B., Brown, C., 2013. The severity of pedestrian
crashes: An analysis using google street view imagery. Journal of
Transport Geography 33, 42--53.
doi:\href{https://doi.org/10.1016/j.jtrangeo.2013.09.002}{10.1016/j.jtrangeo.2013.09.002}

\leavevmode\hypertarget{ref-Hedeker1994random}{}%
Hedeker, D., Gibbons, R.D., 1994. A random-effects ordinal
regression-model for multilevel analysis. Biometrics 50, 933--944.

\leavevmode\hypertarget{ref-Iranitalab2017comparison}{}%
Iranitalab, A., Khattak, A., 2017. Comparison of four statistical and
machine learning methods for crash severity prediction. Accident
Analysis and Prevention 108, 27--36.
doi:\href{https://doi.org/10.1016/j.aap.2017.08.008}{10.1016/j.aap.2017.08.008}

\leavevmode\hypertarget{ref-Islam2014comprehensive}{}%
Islam, S., Jones, S.L., Dye, D., 2014. Comprehensive analysis of single-
and multi-vehicle large truck at-fault crashes on rural and urban
roadways in alabama. Accident Analysis \& Prevention 67, 148--158.
doi:\href{https://doi.org/https://doi.org/10.1016/j.aap.2014.02.014}{https://doi.org/10.1016/j.aap.2014.02.014}

\leavevmode\hypertarget{ref-Khan2015exploring}{}%
Khan, G., Bill, A.R., Noyce, D.A., 2015. Exploring the feasibility of
classification trees versus ordinal discrete choice models for analyzing
crash severity. Transportation Research Part C-Emerging Technologies 50,
86--96.
doi:\href{https://doi.org/10.1016/j.trc.2014.10.003}{10.1016/j.trc.2014.10.003}

\leavevmode\hypertarget{ref-Kim2013driver}{}%
Kim, J.K., Ulfarsson, G.F., Kim, S., Shankar, V.N., 2013. Driver-injury
severity in single-vehicle crashes in california: A mixed logit analysis
of heterogeneity due to age and gender. Accident Analysis and Prevention
50, 1073--1081.
doi:\href{https://doi.org/10.1016/j.aap.2012.08.011}{10.1016/j.aap.2012.08.011}

\leavevmode\hypertarget{ref-Kim1995personal}{}%
Kim, K., Nitz, L., Richardson, J., Li, L., 1995. PERSONAL and behavioral
predictors of automobile crash and injury severity. Accident Analysis
and Prevention 27, 469--481.
doi:\href{https://doi.org/10.1016/0001-4575(95)00001-g}{10.1016/0001-4575(95)00001-g}

\leavevmode\hypertarget{ref-Lee2014analysis}{}%
Lee, C., Li, X.C., 2014. Analysis of injury severity of drivers involved
in single- and two-vehicle crashes on highways in ontario. Accident
Analysis and Prevention 71, 286--295.
doi:\href{https://doi.org/10.1016/j.aap.2014.06.008}{10.1016/j.aap.2014.06.008}

\leavevmode\hypertarget{ref-Li2017interplay}{}%
Li, L., Hasnine, M.S., Habib, K.M.N., Persaud, B., Shalaby, A., 2017.
Investigating the interplay between the attributes of at-fault and
not-at-fault drivers and the associated impacts on crash injury
occurrence and severity level. Journal of Transportation Safety \&
Security 9, 439--456.
doi:\href{https://doi.org/10.1080/19439962.2016.1237602}{10.1080/19439962.2016.1237602}

\leavevmode\hypertarget{ref-Ma2008multivariate}{}%
Ma, J.M., Kockelman, K.M., Damien, P., 2008. A multivariate
poisson-lognormal regression model for prediction of crash counts by
severity, using bayesian methods. Accident Analysis and Prevention 40,
964--975.
doi:\href{https://doi.org/10.1016/j.aap.2007.11.002}{10.1016/j.aap.2007.11.002}

\leavevmode\hypertarget{ref-Maddala1986limited}{}%
Maddala, G.S., 1986. Limited-dependent and qualitative variables in
econometrics. Cambridge university press.

\leavevmode\hypertarget{ref-Mannering2016unobserved}{}%
Mannering, F., Shankar, V., Bhat, C.R., 2016. Unobserved heterogeneity
and the statistical analysis of highway accident data. Analytic Methods
in Accident Research 11, 1--16.
doi:\href{https://doi.org/10.1016/j.amar.2016.04.001}{10.1016/j.amar.2016.04.001}

\leavevmode\hypertarget{ref-McArthur2014spatial}{}%
McArthur, A., Savolainen, P.T., Gates, T.J., 2014. Spatial analysis of
child pedestrian and bicycle crashes development of safety performance
function for areas adjacent to schools. Transportation Research Record
57--63. doi:\href{https://doi.org/10.3141/2465-08}{10.3141/2465-08}

\leavevmode\hypertarget{ref-Merlin2007stress}{}%
Merlin, E.P.R., Gonzalez-Forteza, C., Lira, L.R., Tapia, J.A.J., 2007.
Post-traumatic stress disorder in patients with non intentional injuries
caused by road traffic accidents. Salud Mental 30, 43--48.

\leavevmode\hypertarget{ref-Montella2013crash}{}%
Montella, A., Andreassen, D., Tarko, A.P., Turner, S., Mauriello, F.,
Imbriani, L.L., Romero, M.A., 2013. Crash databases in australasia, the
european union, and the united states review and prospects for
improvement. Transportation Research Record 128--136.
doi:\href{https://doi.org/10.3141/2386-15}{10.3141/2386-15}

\leavevmode\hypertarget{ref-Mooradian2013analysis}{}%
Mooradian, J., Ivan, J.N., Ravishanker, N., Hu, S., 2013. Analysis of
driver and passenger crash injury severity using partial proportional
odds models. Accident Analysis and Prevention 58, 53--58.
doi:\href{https://doi.org/10.1016/j.aap.2013.04.022}{10.1016/j.aap.2013.04.022}

\leavevmode\hypertarget{ref-Mussone2017analysis}{}%
Mussone, L., Bassani, M., Masci, P., 2017. Analysis of factors affecting
the severity of crashes in urban road intersections. Accident Analysis
and Prevention 103, 112--122.
doi:\href{https://doi.org/10.1016/j.aap.2017.04.007}{10.1016/j.aap.2017.04.007}

\leavevmode\hypertarget{ref-Obeng2011gender}{}%
Obeng, K., 2011. Gender differences in injury severity risks in crashes
at signalized intersections. Accident Analysis and Prevention 43,
1521--1531.
doi:\href{https://doi.org/10.1016/j.aap.2011.03.004}{10.1016/j.aap.2011.03.004}

\leavevmode\hypertarget{ref-Osman2018injury}{}%
Osman, M., Mishra, S., Paleti, R., 2018. Injury severity analysis of
commercially-licensed drivers in single-vehicle crashes: Accounting for
unobserved heterogeneity and age group differences. Accident Analysis
and Prevention 118, 289--300.
doi:\href{https://doi.org/10.1016/j.aap.2018.05.004}{10.1016/j.aap.2018.05.004}

\leavevmode\hypertarget{ref-Peek-Asa2010teenage}{}%
Peek-Asa, C., Britton, C., Young, T., Pawlovich, M., Falb, S., 2010.
Teenage driver crash incidence and factors influencing crash injury by
rurality. Journal of Safety Research 41, 487--492.
doi:\href{https://doi.org/10.1016/j.jsr.2010.10.002}{10.1016/j.jsr.2010.10.002}

\leavevmode\hypertarget{ref-Pelissier2019medical}{}%
Pelissier, C., Fort, E., Fontana, L., Hours, M., n.d. Medical and
socio-occupational predictive factors of psychological distress 5 years
after a road accident: A prospective study. Social Psychiatry and
Psychiatric Epidemiology 13.
doi:\href{https://doi.org/10.1007/s00127-019-01780-0}{10.1007/s00127-019-01780-0}

\leavevmode\hypertarget{ref-Penmetsa2017examining}{}%
Penmetsa, P., Pulugurtha, S.S., Duddu, V.R., 2017. Examining injury
severity of not-at-fault drivers in two-vehicle crashes. Transportation
Research Record 164--173.
doi:\href{https://doi.org/10.3141/2659-18}{10.3141/2659-18}

\leavevmode\hypertarget{ref-Provost1998glossary}{}%
Provost, F., Kohavi, R., 1998. Glossary of terms. Journal of Machine
Learning 30, 271--274.

\leavevmode\hypertarget{ref-Rakotonirainy2012older}{}%
Rakotonirainy, A., Steinhardt, D., Delhomme, P., Darvell, M., Schramm,
A., 2012. Older drivers' crashes in queensland, australia. Accident
Analysis and Prevention 48, 423--429.
doi:\href{https://doi.org/10.1016/j.aap.2012.02.016}{10.1016/j.aap.2012.02.016}

\leavevmode\hypertarget{ref-Rana2010copula}{}%
Rana, T.A., Sikder, S., Pinjari, A.R., 2010. Copula-based method for
addressing endogeneity in models of severity of traffic crash injuries
application to two-vehicle crashes. Transportation Research Record
75--87. doi:\href{https://doi.org/10.3141/2147-10}{10.3141/2147-10}

\leavevmode\hypertarget{ref-Regev2018crash}{}%
Regev, S., Rolison, J.J., Moutari, S., 2018. Crash risk by driver age,
gender, and time of day using a new exposure methodology. Journal of
Safety Research 66, 131--140.
doi:\href{https://doi.org/10.1016/j.jsr.2018.07.002}{10.1016/j.jsr.2018.07.002}

\leavevmode\hypertarget{ref-Rifaat2007accident}{}%
Rifaat, S.M., Chin, H.C., 2007. Accident severity analysis using ordered
probit model. Journal of Advanced Transportation 41, 91--114.
doi:\href{https://doi.org/10.1002/atr.5670410107}{10.1002/atr.5670410107}

\leavevmode\hypertarget{ref-Roorda2010trip}{}%
Roorda, M.J., Paez, A., Morency, C., Mercado, R., Farber, S., 2010. Trip
generation of vulnerable populations in three canadian cities: A spatial
ordered probit approach. Transportation 37, 525--548.
doi:\href{https://doi.org/10.1007/s11116-010-9263-3}{10.1007/s11116-010-9263-3}

\leavevmode\hypertarget{ref-Salon2018determinants}{}%
Salon, D., McIntyre, A., 2018. Determinants of pedestrian and bicyclist
crash severity by party at fault in san francisco, ca. Accident Analysis
and Prevention 110, 149--160.
doi:\href{https://doi.org/10.1016/j.aap.2017.11.007}{10.1016/j.aap.2017.11.007}

\leavevmode\hypertarget{ref-Sasidharan2014partial}{}%
Sasidharan, L., Menendez, M., 2014. Partial proportional odds model-an
alternate choice for analyzing pedestrian crash injury severities.
Accident Analysis and Prevention 72, 330--340.
doi:\href{https://doi.org/10.1016/j.aap.2014.07.025}{10.1016/j.aap.2014.07.025}

\leavevmode\hypertarget{ref-Savolainen2007probabilistic}{}%
Savolainen, P., Mannering, F., 2007. Probabilistic models of
motorcyclists' injury severities in single- and multi-vehicle crashes.
Accident Analysis and Prevention 39, 955--963.
doi:\href{https://doi.org/10.1016/j.aap.2006.12.016}{10.1016/j.aap.2006.12.016}

\leavevmode\hypertarget{ref-Savolainen2011statistical}{}%
Savolainen, P.T., Mannering, F., Lord, D., Quddus, M.A., 2011. The
statistical analysis of highway crash-injury severities: A review and
assessment of methodological alternatives. Accident Analysis and
Prevention 43, 1666--1676.
doi:\href{https://doi.org/10.1016/j.aap.2011.03.025}{10.1016/j.aap.2011.03.025}

\leavevmode\hypertarget{ref-Shaheed2013mixed}{}%
Shaheed, M.S.B., Gkritza, K., Zhang, W., Hans, Z., 2013. A mixed logit
analysis of two-vehicle crash seventies involving a motorcycle. Accident
Analysis and Prevention 61, 119--128.
doi:\href{https://doi.org/10.1016/j.aap.2013.05.028}{10.1016/j.aap.2013.05.028}

\leavevmode\hypertarget{ref-Shamsunnahar2013evaluating}{}%
Shamsunnahar, Y., Eluru, N., 2013. Evaluating alternate discrete outcome
frameworks for modeling crash injury severity. Accident Analysis \&
Prevention 59, 506--521.
doi:\href{https://doi.org/https://doi.org/10.1016/j.aap.2013.06.040}{https://doi.org/10.1016/j.aap.2013.06.040}

\leavevmode\hypertarget{ref-Shamsunnahar2014examining}{}%
Shamsunnahar, Y., Eluru, N., Pinjari, A.R., Tay, R., 2014. Examining
driver injury severity in two vehicle crashes - a copula based approach.
Accident Analysis and Prevention 66, 120--135.
doi:\href{https://doi.org/10.1016/j.aap.2014.01.018}{10.1016/j.aap.2014.01.018}

\leavevmode\hypertarget{ref-Symons2019reduced}{}%
Symons, J., Howard, E., Sweeny, K., Kumnick, M., Sheehan, P., 2019.
Reduced road traffic injuries for young people: A preliminary investment
analysis. Journal of Adolescent Health 65, S34--S43.
doi:\href{https://doi.org/10.1016/j.jadohealth.2019.01.009}{10.1016/j.jadohealth.2019.01.009}

\leavevmode\hypertarget{ref-Tarrao2014modeling}{}%
Tarrao, G.A., Coelho, M.C., Rouphail, N.M., 2014. Modeling the impact of
subject and opponent vehicles on crash severity in two-vehicle
collisions. Transportation Research Record 53--64.
doi:\href{https://doi.org/10.3141/2432-07}{10.3141/2432-07}

\leavevmode\hypertarget{ref-Tay2011multinomial}{}%
Tay, R., Choi, J., Kattan, L., Khan, A., 2011. A multinomial logit model
of pedestrian-vehicle crash severity. International Journal of
Sustainable Transportation 5, 233--249.
doi:\href{https://doi.org/10.1080/15568318.2010.497547}{10.1080/15568318.2010.497547}

\leavevmode\hypertarget{ref-Thompson2018trends}{}%
Thompson, J.P., Baldock, M.R.J., Dutschke, J.K., 2018. Trends in the
crash involvement of older drivers in australia. Accident Analysis and
Prevention 117, 262--269.
doi:\href{https://doi.org/10.1016/j.aap.2018.04.027}{10.1016/j.aap.2018.04.027}

\leavevmode\hypertarget{ref-Train2009discrete}{}%
Train, K., 2009. Discrete choice methods with simulation, 2nd Edition.
ed. Cambridge University Press, Cambridge.

\leavevmode\hypertarget{ref-Wang2015copula}{}%
Wang, K., Yasmin, S., Konduri, K.C., Eluru, N., Ivan, J.N., 2015.
Copula-based joint model of injury severity and vehicle damage in
two-vehicle crashes. Transportation Research Record 158--166.
doi:\href{https://doi.org/10.3141/2514-17}{10.3141/2514-17}

\leavevmode\hypertarget{ref-Wang2005use}{}%
Wang, X.K., Kockelman, K.M., 2005. Use of heteroscedastic ordered logit
model to study severity of occupant injury - distinguishing effects of
vehicle weight and type, in: Statistical Methods; Highway Safety Data,
Analysis, and Evaluation; Occupant Protection; Systematic Reviews and
Meta-Analysis, Transportation Research Record. pp. 195--204.

\leavevmode\hypertarget{ref-White1972effects}{}%
White, S., Clayton, S., 1972. Some effects of alcohol, age of driver,
and estimated speed on the likelihood of driver injury. Accident
Analysis \& Prevention 4.

\leavevmode\hypertarget{ref-Wijnen2019analysis}{}%
Wijnen, W., Weijermars, W., Schoeters, A., Berghe, W. van den, Bauer,
R., Carnis, L., Elvik, R., Martensen, H., 2019. An analysis of official
road crash cost estimates in european countries. Safety Science 113,
318--327.
doi:\href{https://doi.org/10.1016/j.ssci.2018.12.004}{10.1016/j.ssci.2018.12.004}

\leavevmode\hypertarget{ref-WHO2019global}{}%
World Health Organization, 2019. Global status report on road safety
2018 (2018). Geneva.

\leavevmode\hypertarget{ref-Wu2014mixed}{}%
Wu, Q., Chen, F., Zhang, G.H., Liu, X.Y.C., Wang, H., Bogus, S.M., 2014.
Mixed logit model-based driver injury severity investigations in single-
and multi-vehicle crashes on rural two-lane highways. Accident Analysis
and Prevention 72, 105--115.
doi:\href{https://doi.org/10.1016/j.aap.2014.06.014}{10.1016/j.aap.2014.06.014}


\end{document}


